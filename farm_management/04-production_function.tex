\PassOptionsToPackage{unicode=true}{hyperref} % options for packages loaded elsewhere
\PassOptionsToPackage{hyphens}{url}
\documentclass[12pt,ignorenonframetext,aspectratio=169]{beamer}
\IfFileExists{pgfpages.sty}{\usepackage{pgfpages}}{}
\setbeamertemplate{caption}[numbered]
\setbeamertemplate{caption label separator}{: }
\setbeamercolor{caption name}{fg=normal text.fg}
\beamertemplatenavigationsymbolsempty
\usepackage{lmodern}
\usepackage{amssymb}
\usepackage{amsmath}
\usepackage{ifxetex,ifluatex}
\usepackage{fixltx2e} % provides \textsubscript
\ifnum 0\ifxetex 1\fi\ifluatex 1\fi=0 % if pdftex
  \usepackage[T1]{fontenc}
  \usepackage[utf8]{inputenc}
\else % if luatex or xelatex
  \ifxetex
    \usepackage{mathspec}
  \else
    \usepackage{fontspec}
\fi
\defaultfontfeatures{Ligatures=TeX,Scale=MatchLowercase}






%
\fi

  \usetheme[]{iqss}






% use upquote if available, for straight quotes in verbatim environments
\IfFileExists{upquote.sty}{\usepackage{upquote}}{}
% use microtype if available
\IfFileExists{microtype.sty}{%
  \usepackage{microtype}
  \UseMicrotypeSet[protrusion]{basicmath} % disable protrusion for tt fonts
}{}


\newif\ifbibliography


\hypersetup{
      pdftitle={Production function},
        pdfauthor={Deependra Dhakal},
          pdfborder={0 0 0},
    breaklinks=true}
%\urlstyle{same}  % Use monospace font for urls







% Prevent slide breaks in the middle of a paragraph:
\widowpenalties 1 10000
\raggedbottom

  \AtBeginPart{
    \let\insertpartnumber\relax
    \let\partname\relax
    \frame{\partpage}
  }
  \AtBeginSection{
    \ifbibliography
    \else
      \let\insertsectionnumber\relax
      \let\sectionname\relax
      \frame{\sectionpage}
    \fi
  }
  \AtBeginSubsection{
    \let\insertsubsectionnumber\relax
    \let\subsectionname\relax
    \frame{\subsectionpage}
  }



\setlength{\parindent}{0pt}
\setlength{\parskip}{6pt plus 2pt minus 1pt}
\setlength{\emergencystretch}{3em}  % prevent overfull lines
\providecommand{\tightlist}{%
  \setlength{\itemsep}{0pt}\setlength{\parskip}{0pt}}

  \setcounter{secnumdepth}{0}


  \usepackage{booktabs}
  \usepackage{longtable}
  \usepackage{emptypage}
  \usepackage{array}
  \usepackage{multirow}
  \usepackage{wrapfig}
  \usepackage{float}
  \usepackage{colortbl}
  \usepackage{pdflscape}
  \usepackage{tabu}
  \usepackage{threeparttable}
  \usepackage{threeparttablex}
  \usepackage[normalem]{ulem}
  \usepackage{rotating}
  \usepackage{makecell}
  \usepackage{xcolor}
  \usepackage{tikz} % required for image opacity change
  \usepackage[absolute,overlay]{textpos} % for text formatting
  \usepackage[utf8]{inputenc}
  \usetikzlibrary{mindmap,arrows,shapes,positioning,shadows,trees}

  % this font option is amenable for beamer
  \setbeamerfont{caption}{size=\tiny}


%% IQSS overrides
\iqsssectiontitle{Outline}

\AtBeginSection[]{
  \title{\insertsectionhead}
  {
    \definecolor{white}{rgb}{0.776,0.357,0.157}
    \definecolor{iqss@orange}{rgb}{1,1,1}
    \ifnum \insertmainframenumber > \insertframenumber
    \frame{
      \frametitle{\iqsssectiontitleheader}
      \tableofcontents[currentsection]
    }
    \else
    \frame{
      \frametitle{Backup Slides}
      \tableofcontents[sectionstyle=shaded/shaded,subsectionstyle=shaded/shaded/shaded]
    }
    \fi
  }
}

\AtBeginSubsection[]{}

%%


  \title[]{Production function}



  \author[
        Deependra Dhakal
    ]{Deependra Dhakal}

  \institute[
    ]{
    GAASC, Baitadi \and Tribhuwan University
    }

\date[
      \today
  ]{
      \today
        }

\begin{document}

% Hide progress bar and footline on titlepage
  \begin{frame}[plain]
  \titlepage
  \end{frame}



\hypertarget{factor-product-relationship}{%
\section{Factor-product
relationship}\label{factor-product-relationship}}

\begin{frame}{Background}
\protect\hypertarget{background}{}
\begin{itemize}
\tightlist
\item
  When concerned with resource allocation for production optimization,
  an understanding of input-output or factor-product relationship is
  important.
\item
  First, study of physical or technical relationship is important.
  Second, for decision making, application of economic choice indicators
  such as price ratio is required.
\item
  In a simple scenario, we details the physical factor-product
  relationship of a single variable resoruce and single product.
\item
  Many time resources or capacities of technical units, such as a ropani
  of land or a cow, are fixed and choice is to vary the input of only
  one factor -- such as fertilizer \alertc{OR} labor.
\item
  Other inputs such as fixed capital, buildings, implements and
  technical knowhow remain the same.
\end{itemize}
\end{frame}

\begin{frame}{}
\protect\hypertarget{section}{}
\begin{itemize}
\tightlist
\item
  Under such situation question of how much of certain input (amount of
  fertilizer or feed to a cow) to apply arises ?
\item
  This situtation is dealt by single factor-product relationships. a.k.a
  single variable production function (in a production function various
  levels of input are involved with corresponding output of the
  product).
\end{itemize}

\begin{block}{Inputs have several different names:}
\protect\hypertarget{inputs-have-several-different-names}{}
Inputs = factors = factors of production = resources = A, L, K, M

\begin{quote}
\textbf{A}: Land (Natural and biological resources, climate.)
\newline \textbf{L}: Labor (Human resources.) \newline \textbf{K}:
Capital (Manufactured resources, which include buildings, machines,
tools, and equipment.) \newline \textbf{M}: Management (The
entrepreneur, or individual, who combines the other resources into
inputs.)
\end{quote}
\end{block}
\end{frame}

\hypertarget{types-of-factor-product-relationships-production-functions}{%
\section{Types of factor-product relationships (production
functions)}\label{types-of-factor-product-relationships-production-functions}}

\begin{frame}{Types of factor-product relationships (production
functions)}
\begin{itemize}
\tightlist
\item
  There can be three types of input-output relationships in the
  production of a commodity where one input is varied and the quantities
  of all other inputs are fixed.

  \begin{enumerate}
  \tightlist
  \item
    Constant marginal rate of returns (Constant productivity)
  \item
    Increasing marginal rate of returns (Increasing productivity)
  \item
    Decreasing marginal rate of returns (Decreasing productivity)
  \end{enumerate}
\end{itemize}
\end{frame}

\begin{frame}{Contant marginal rate of returns}
\protect\hypertarget{contant-marginal-rate-of-returns}{}
\begin{itemize}
\tightlist
\item
  Each additional unit of the variable input when applied to fixed
  factors, produces an equal amount of additional product. The amount of
  product increases by the same magnitude for each additional unit of
  input.
\item
  Not a very common relationship in agriculture and holds true only for
  limited range.
\item
  Example:

  \begin{enumerate}
  \tightlist
  \item
    Addition of one acre of land (technology and other factors being
    same) will add the same amount of product.
  \item
    An addition of one tractor plus driver will do the same amount of
    work as previous tractor driver unit did.
  \end{enumerate}
\end{itemize}
\end{frame}

\begin{frame}{}
\protect\hypertarget{section-1}{}
\begin{figure}
\includegraphics[width=0.7\linewidth]{04-production_function_files/figure-beamer/tractor-driver-unit-cmr-fig-1} \caption{Constant marginal rate of returns for a input-output relationship between number of tractor plus driver unit recruits and hectares of land ploughed.}\label{fig:tractor-driver-unit-cmr-fig}
\end{figure}
\end{frame}

\begin{frame}{}
\protect\hypertarget{section-2}{}
\begin{table}

\caption{\label{tab:tractor-driver-unit-cmr-tab}Constant marginal rate of returns for a input-output relationship between number of tractor plus driver unit recruits and hectares of land ploughed.}
\centering
\fontsize{6}{8}\selectfont
\begin{tabular}[t]{rrrrr}
\toprule
tractor driver unit & field ploughed & marginal tractor driver unit & marginal field ploughed & marginal rate returns\\
\midrule
1 & 2 &  &  & 2\\
2 & 4 & 1 & 2 & 2\\
5 & 10 & 3 & 6 & 2\\
6 & 12 & 1 & 2 & 2\\
\bottomrule
\end{tabular}
\end{table}
\end{frame}

\begin{frame}{Increasing marginal rate of returns}
\protect\hypertarget{increasing-marginal-rate-of-returns}{}
\begin{itemize}
\tightlist
\item
  Every additional or marginal unit of input adds more to the total
  product than the previous unit, i.e., addition to total product is at
  an increasing rate.
\item
  In actual practice, the cases of purely increasing returns are rarely
  available except, again, in very limited range.
\item
  This relationship is possible when the fixed factors of production are
  in excess capacity and addition of the small units of a variable
  resource makes more and more efficient use of fixed resources.
\item
  Example:

  \begin{enumerate}
  \tightlist
  \item
    Small quanity of wheat seed applied when other factors of production
    such as fertilizer, irrigation and other cultural practices can be
    used at high levels will give low returns.
  \end{enumerate}
\end{itemize}
\end{frame}

\begin{frame}{}
\protect\hypertarget{section-3}{}
\begin{figure}
\includegraphics[width=0.7\linewidth]{04-production_function_files/figure-beamer/nitrogen-wheat-imr-fig-1} \caption{Increasing marginal rate of returns for hypothetical wheat production scenario.}\label{fig:nitrogen-wheat-imr-fig}
\end{figure}
\end{frame}

\begin{frame}{}
\protect\hypertarget{section-4}{}
\begin{table}

\caption{\label{tab:nitrogen-wheat-imr-tab}Increasing marginal rate of return for hypothetical wheat production scenario.}
\centering
\fontsize{6}{8}\selectfont
\begin{tabular}[t]{rrrrr}
\toprule
wheat seed & marginal wheat seed & wheat production & marginal wheat production & marginal rate returns\\
\midrule
10 &  & 1000 &  & \\
15 & 5 & 1025 & 25 & 5\\
20 & 5 & 1075 & 50 & 10\\
25 & 5 & 1150 & 75 & 15\\
30 & 5 & 1250 & 100 & 20\\
\addlinespace
35 & 5 & 1375 & 125 & 25\\
\bottomrule
\end{tabular}
\end{table}
\end{frame}

\begin{frame}{Decreasing marginal rate of returns}
\protect\hypertarget{decreasing-marginal-rate-of-returns}{}
\begin{itemize}
\tightlist
\item
  Each additional unit of input adds less to the total product than the
  previous unit did.
\item
  This relationship exists in almost every practical situation in
  agriculture.
\item
  Example:

  \begin{enumerate}
  \tightlist
  \item
    Response to fertilizers, insecticides, seeds, irrigation, feeds,
    etc. all show diminishing returns.
  \end{enumerate}
\end{itemize}
\end{frame}

\begin{frame}{}
\protect\hypertarget{section-5}{}
\begin{figure}
\includegraphics[width=0.7\linewidth]{04-production_function_files/figure-beamer/nitrogen-wheat-dmr-fig-1} \caption{Decreasing marginal rate of returns for hypothetical wheat production scenario.}\label{fig:nitrogen-wheat-dmr-fig}
\end{figure}
\end{frame}

\begin{frame}{}
\protect\hypertarget{section-6}{}
\begin{table}

\caption{\label{tab:nitrogen-wheat-dmr-tab}Decreasing marginal rate of return for hypothetical wheat production scenario.}
\centering
\fontsize{6}{8}\selectfont
\begin{tabular}[t]{rrrrr}
\toprule
wheat fertilizer & marginal wheat fertilizer & wheat production & marginal wheat production & marginal rate returns\\
\midrule
100 &  & 2000 &  & \\
150 & 50 & 2600 & 600 & 12\\
200 & 50 & 3100 & 500 & 10\\
250 & 50 & 3500 & 400 & 8\\
300 & 50 & 3800 & 300 & 6\\
\addlinespace
350 & 50 & 4000 & 200 & 4\\
\bottomrule
\end{tabular}
\end{table}
\end{frame}

\hypertarget{relationship-between-total-average-and-marginal-products}{%
\section{Relationship between total, average and marginal
products}\label{relationship-between-total-average-and-marginal-products}}

\begin{frame}{Physical returns or productivity relationships}
\protect\hypertarget{physical-returns-or-productivity-relationships}{}
\begin{itemize}
\tightlist
\item
  Total, average and marginal products are related.
\item
  Both the average and marginal product curves or relationships can be
  derived once total product curve has been obtained.
\end{itemize}
\end{frame}

\begin{frame}{}
\protect\hypertarget{section-7}{}
\begin{table}

\caption{\label{tab:tc-ac-mc-relationship}Relationship between total, average and marginal products}
\centering
\fontsize{6}{8}\selectfont
\begin{tabular}[t]{rrrrl}
\toprule
Units of fertilizer input & Total product (TP; Y) & Average product (AP; Y/X) & Marginal product (MP; Y/X) & Remarks\\
\midrule
\rowcolor{gray!6}  0 & 0 &  &  & \\

1 & 2 & 2.0 & 2 & \\

\rowcolor{gray!6}  2 & 5 & 2.5 & 3 & \\

3 & 9 & 3.0 & 4 & \multirow{-4}{*}{\raggedright\arraybackslash Increasing at increasing rate}\\
\cmidrule{1-5}
\rowcolor{gray!6}  4 & 14 & 3.5 & 5 & \\

5 & 19 & 3.8 & 5 & \multirow{-2}{*}{\raggedright\arraybackslash Increasing at constant rate}\\
\cmidrule{1-5}
\rowcolor{gray!6}  6 & 23 & 3.8 & 4 & \\

7 & 26 & 3.7 & 3 & \\

\rowcolor{gray!6}  8 & 28 & 3.5 & 2 & \\

9 & 29 & 3.2 & 1 & \\

\rowcolor{gray!6}  10 & 29 & 2.9 & 0 & \multirow{-5}{*}{\raggedright\arraybackslash Increasing at decreasing rate}\\
\cmidrule{1-5}
11 & 28 & 2.5 & -1 & \\

\rowcolor{gray!6}  12 & 26 & 2.2 & -2 & \multirow{-2}{*}{\raggedright\arraybackslash Decreasing at increasing rate}\\
\bottomrule
\end{tabular}
\end{table}
\end{frame}

\begin{frame}{}
\protect\hypertarget{section-8}{}
\begin{figure}
\includegraphics[width=0.7\linewidth]{04-production_function_files/figure-beamer/tc-ac-mc-relationship-plot-1} \caption{Relationship between TP, AP and MP.}\label{fig:tc-ac-mc-relationship-plot}
\end{figure}
\end{frame}

\begin{frame}{Total product (TPC) and Marginal product (MPC)}
\protect\hypertarget{total-product-tpc-and-marginal-product-mpc}{}
\begin{enumerate}
\tightlist
\item
  Since the marginal product (MP) is a measure of rate of change
  therefore:
\item
  When the Total product (TP) is increasing, the MP will be positive.
\item
  When the TP remains constant, the MP will be zero, and
\item
  When the TP decreases, the MP will be negative.
\item
  So long as MP moves upwards or increases, the TP increases at an
  increasing rate.
\item
  When the MP remains constant, the TP increases at a constant rate.
\item
  When the MP starts declining or slopes downward, the TP will be
  increasing at a decreasing rate.
\item
  At the point when MP becomes zero or when MP intersects X-axis, the
  total product will be at maximum.
\end{enumerate}
\end{frame}

\begin{frame}{Marginal product (MPC) and average product (APC)}
\protect\hypertarget{marginal-product-mpc-and-average-product-apc}{}
\begin{enumerate}
\tightlist
\item
  When the MP keeps increasing or is moving upward right from the
  beginning the Average product (AP) curve also keeps moving upward. So
  long as MP curve remains above the AP curve, the AP curve keeps
  increasing. This means when the AP is increasing, the MP must be
  greater than the average product.
\item
  As soon as the MP curve goes below the AP curve, the AP curve starts
  decreasing; i.e.~when AP is decreasing, the MP is always less than the
  AP.
\item
  When AP is equal to MP, at this point AP will be at the maximum. From
  here onward, MP will change from greater to being less than AP, the MP
  curve must therefore intersect AP curve from above at its highest
  point.
\end{enumerate}
\end{frame}

\begin{frame}{Summary of relationship between MP and AP}
\protect\hypertarget{summary-of-relationship-between-mp-and-ap}{}
\begin{enumerate}
\tightlist
\item
  When MP \textgreater{} AP, AP is increasing
\item
  When MP \textless{} AP, AP is decreasing
\item
  When MP = AP, AP is at a maximum.
\end{enumerate}
\end{frame}

\hypertarget{elasticity-of-production}{%
\section{Elasticity of production}\label{elasticity-of-production}}

\begin{frame}{Elasticity of production}
\begin{itemize}
\tightlist
\item
  The elasticity of production refers to the percentage change in output
  in response to the percentage change in input. It is be denoted by the
  symbol \(E_p\) and can be computed as:
\end{itemize}

\[
\begin{aligned}
E_p &= \frac{\frac{\Delta Y}{Y}}{\frac{\Delta X}{X}} \\
&= \frac{X}{Y}\times {\frac{\Delta Y}{\Delta X}}
\end{aligned}
\]
\end{frame}

\begin{frame}{}
\protect\hypertarget{section-9}{}
\begin{itemize}
\tightlist
\item
  Let us consider an example, given in Table
  \ref{tab:elasticity-production}.
\end{itemize}

\begin{table}

\caption{\label{tab:elasticity-production}Relationship between fertilizer input and yield of wheat.}
\centering
\fontsize{8}{10}\selectfont
\begin{tabular}[t]{rr}
\toprule
Fertilizer doses (X) & Total yield attributable to fertilizer (Y)\\
\midrule
\rowcolor{gray!6}  0 & 0\\
1 & 103\\
\rowcolor{gray!6}  2 & 174\\
3 & 223\\
\rowcolor{gray!6}  4 & 257\\
\addlinespace
5 & 281\\
\rowcolor{gray!6}  6 & 298\\
7 & 308\\
\bottomrule
\end{tabular}
\end{table}
\end{frame}

\begin{frame}{}
\protect\hypertarget{section-10}{}
As input increase from 1 to 2 units, total output increase from 103 to
174 units. Output thus increases by 71.9 percent in response to input
increase of 100 percent. The elasticity of production is therefore:

\[
\begin{aligned}
E_p &= {\frac{71.9}{100}} \\
&= 0.719
\end{aligned}
\]

Similarly, between the second and third unit of input, the elasticity
works out to be 0.56.
\end{frame}

\begin{frame}{}
\protect\hypertarget{section-11}{}
\begin{itemize}
\tightlist
\item
  Essential points to remember in elasticity analysis are:

  \begin{enumerate}
  \tightlist
  \item
    A production function with an elasticity of \(E_p = 1.0\) indicates
    constant returns throughout. This means one percent increase in
    input is always accompanied by one percent increase in output.
  \item
    The elasticity is more than 1.0 up to the point of maximum average
    product where it becomes 1.0.
  \item
    The elasticity is less than 1.0 between the points of maximum
    average product and the maximum total product.
  \item
    When it becomes less than zero, total product declines.
  \item
    When elasticity of production is 1.0, marginal and average products
    are equal. This condition holds true at only one point on the
    classical production function as shown in Figure
    \ref{fig:tc-ac-mc-relationship-plot}.
  \item
    A production function for which the elasticity is less than 1.0
    throughout all ranges of input used will indicate diminishing
    returns.
  \end{enumerate}
\end{itemize}
\end{frame}

\hypertarget{three-regions-of-production-function}{%
\section{Three regions of production
function}\label{three-regions-of-production-function}}

\begin{frame}{}
\protect\hypertarget{section-12}{}
\begin{itemize}
\tightlist
\item
  The classic production function (Figure
  \ref{fig:tc-ac-mc-relationship-plot} and Table
  \ref{tab:tc-ac-mc-relationship}) can be divided into three
  ``regions'', ``zones'', ``parts'' or ``stages'', each important from
  the standpoint of decision-making on efficient resource use.
\item
  These are (again shown in Figure
  \ref{fig:production-function-stages}):
\end{itemize}

\begin{figure}
\includegraphics[width=0.7\linewidth]{04-production_function_files/figure-beamer/production-function-stages-1} \caption{Three stages of a production function.}\label{fig:production-function-stages}
\end{figure}
\end{frame}

\begin{frame}{Region 1 (irrational zone)}
\protect\hypertarget{region-1-irrational-zone}{}
\begin{itemize}
\tightlist
\item
  This region hold from the point of origin up to the point the MPP
  remains greater than APP.
\item
  The APP increases throughout this region indicating that the
  efficiency of all the units of variable input keeps increasing.
\item
  Zone terminates as soon as APP equals MPP
\item
  Notes:

  \begin{itemize}
  \tightlist
  \item
    once the farmer decides to produce, he must produce up to the level
    of input use where the APP is highest.
  \item
    the efficiency of the variable input keeps increasing throughout the
    Region 1.
  \item
    it is not reasonable to stop using an input when its efficiency on
    all units used is increasing.
  \item
    reaching to the point of highest average product is always
    profitable
  \end{itemize}
\end{itemize}
\end{frame}

\begin{frame}{Region 3 (irrational zone)}
\protect\hypertarget{region-3-irrational-zone}{}
\begin{itemize}
\tightlist
\item
  This region obtains where MPP crosses zero point and becomes negative.
\item
  Negative MPP occurs when so much excessive quanity of the variable
  input is used that total output begins to decrease.
\item
  Notes:

  \begin{itemize}
  \tightlist
  \item
    in the third region of production, the TP is decreasing.
  \item
    additional quantities of input reduce the total output in this
    region, hence it is not profitable zone even if the additional
    quantities of resources are available free of cost.
  \item
    e.g., if a farmer operates in this region -- a farmer growing local
    variety and applying fertilizer without restraint might suffer loss
    of yield due to lodging and inefficient nutrient utilization -- he
    will incurr double loss:
  \end{itemize}

  \begin{enumerate}
  \tightlist
  \item
    reduced production
  \item
    unnecessary additional cost of inputs
  \end{enumerate}
\end{itemize}
\end{frame}

\begin{frame}{Region 2 (rational zone)}
\protect\hypertarget{region-2-rational-zone}{}
\begin{itemize}
\tightlist
\item
  This region obtains when MPP is decreasing and is less than APP. In
  this region at the starting point, MPP is equal to APP and it extends
  to the point where MPP becomes zero.
\item
  The APP is also decreasing.
\item
  Within the boundaries of this region is the area of economic
  relevance. Optimum point of input-use must be somewhere in this
  rational zone.
\item
  Optimum point can, however, be located only when input and output
  prices are known.
\item
  This region of production embodies diminishing returns phase -- both
  AP and MP are decreasing.
\end{itemize}
\end{frame}

\begin{frame}{Why operate in zone II of production function ?}
\protect\hypertarget{why-operate-in-zone-ii-of-production-function}{}
\begin{itemize}
\item
  When a farmer is undertaking production on his farm, his prime
  objective is to maximize his returns. The TP curve in the production
  function shows only the total production while the MP curve represents
  the rate of returns from production. MP as a measure of farm operation
  efficiency at different level of production is useful to decide how
  much to produce with availabe quantities of input.
\item
  It is of interest to farmer that each additional unit has variable
  relations with quantity of output. One surely will not want to stop
  production when addition of input causes more increment in product
  that the earlier unit of input did. This is the case of Ist zone of
  production function. In this zone efficiency of additional input is
  increasing and the fixed factors of production are not being used upto
  their full potential. To maximize returns from production, it is
  required that input be increased.
\end{itemize}
\end{frame}

\begin{frame}{}
\protect\hypertarget{section-13}{}
\begin{itemize}
\item
  As soon as the production from additional one unit of input stops
  adding to the total product, input use beyond this level is wasteful.
  This leads to farmer incurring double loss (first from increased cost
  of input use second from reduced returns from the product itself).
\item
  However, as long as addition to total product is increasing at
  increasing rate or increasing at constant rate, or even increasing at
  decreasing rate, so that cost of additional unit of input use can be
  justified with returns from product obtained by the same additional
  unit of input, the production is carried out. The exact optimum level
  of input use, however, is determined by the price of input and output,
  both.
\end{itemize}
\end{frame}

\begin{frame}{}
\protect\hypertarget{section-14}{}
\tikzset{
  basic/.style  = {draw, text width=2cm, drop shadow, font=\sffamily, rectangle},
  root/.style   = {basic, rounded corners=2pt, thin, align=center,
                   fill=green!30},
  level 2/.style = {basic, rounded corners=6pt, thin,align=center, fill=green!60,
                   text width=8em},
  level 3/.style = {basic, thin, align=left, fill=pink!60, text width=6.5em}
}

\begin{tikzpicture}[
  level 1/.style={sibling distance=40mm},
  edge from parent/.style={->,draw},
  >=latex]

% root of the the initial tree, level 1
\node[root] {Technology}
% The first level, as children of the initial tree
  child {node[level 2] (c1) {General impacts}}
  child {node[level 2] (c2) {Types}}
  child {node[level 2] (c3) {Broader impacts}};

% The second level, relatively positioned nodes
\begin{scope}[every node/.style={level 3}]
\node [below of = c1, xshift=15pt] (c11) {Yield increasing};
\node [below of = c11] (c12) {Factor saving};
\node [below of = c12] (c13) {Both factor saving and yield increasing};

\node [below of = c2, xshift=15pt] (c21) {Inferior: Downward shift of production function};
\node [below of = c21] (c22) {Superior: Upward shift of production function};

\node [below of = c3, xshift=15pt] (c31) {Increased worker safety};
\node [below of = c31] (c32) {Less run-off of chemicals into rivers and ground water};
\node [below of = c32] (c33) {Reduced impact on natural ecosystem};
\node [below of = c33] (c34) {Lower food prices};
\node [below of = c34] (c35) {Decreased use of water, fertilizer, and pesticides};
\node [below of = c35] (c36) {Higher productivity};
\node [below of = c36] (c37) {Safer growing environment and safer food};

\end{scope}

% lines from each level 1 node to every one of its "children"
\foreach \value in {1,2,3}
  \draw[->] (c1.195) |- (c1\value.west);

\foreach \value in {1,...,2}
  \draw[->] (c2.195) |- (c2\value.west);

\foreach \value in {1,...,7}
  \draw[->] (c3.195) |- (c3\value.west);
\end{tikzpicture}
\end{frame}




\end{document}
