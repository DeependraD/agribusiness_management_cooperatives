\PassOptionsToPackage{unicode=true}{hyperref} % options for packages loaded elsewhere
\PassOptionsToPackage{hyphens}{url}
\documentclass[12pt,ignorenonframetext,aspectratio=169]{beamer}
\IfFileExists{pgfpages.sty}{\usepackage{pgfpages}}{}
\setbeamertemplate{caption}[numbered]
\setbeamertemplate{caption label separator}{: }
\setbeamercolor{caption name}{fg=normal text.fg}
\beamertemplatenavigationsymbolsempty
\usepackage{lmodern}
\usepackage{amssymb}
\usepackage{amsmath}
\usepackage{ifxetex,ifluatex}
\usepackage{fixltx2e} % provides \textsubscript
\ifnum 0\ifxetex 1\fi\ifluatex 1\fi=0 % if pdftex
  \usepackage[T1]{fontenc}
  \usepackage[utf8]{inputenc}
\else % if luatex or xelatex
  \ifxetex
    \usepackage{mathspec}
  \else
    \usepackage{fontspec}
\fi
\defaultfontfeatures{Ligatures=TeX,Scale=MatchLowercase}






%
\fi

  \usetheme[]{iqss}






% use upquote if available, for straight quotes in verbatim environments
\IfFileExists{upquote.sty}{\usepackage{upquote}}{}
% use microtype if available
\IfFileExists{microtype.sty}{%
  \usepackage{microtype}
  \UseMicrotypeSet[protrusion]{basicmath} % disable protrusion for tt fonts
}{}


\newif\ifbibliography


\hypersetup{
      pdftitle={Production relationships},
        pdfauthor={Deependra Dhakal},
          pdfborder={0 0 0},
    breaklinks=true}
%\urlstyle{same}  % Use monospace font for urls







% Prevent slide breaks in the middle of a paragraph:
\widowpenalties 1 10000
\raggedbottom

  \AtBeginPart{
    \let\insertpartnumber\relax
    \let\partname\relax
    \frame{\partpage}
  }
  \AtBeginSection{
    \ifbibliography
    \else
      \let\insertsectionnumber\relax
      \let\sectionname\relax
      \frame{\sectionpage}
    \fi
  }
  \AtBeginSubsection{
    \let\insertsubsectionnumber\relax
    \let\subsectionname\relax
    \frame{\subsectionpage}
  }



\setlength{\parindent}{0pt}
\setlength{\parskip}{6pt plus 2pt minus 1pt}
\setlength{\emergencystretch}{3em}  % prevent overfull lines
\providecommand{\tightlist}{%
  \setlength{\itemsep}{0pt}\setlength{\parskip}{0pt}}

  \setcounter{secnumdepth}{0}


  \usepackage{booktabs}
  \usepackage{longtable}
  \usepackage{emptypage}
  \usepackage{array}
  \usepackage{multirow}
  \usepackage{wrapfig}
  \usepackage{float}
  \usepackage{colortbl}
  \usepackage{pdflscape}
  \usepackage{tabu}
  \usepackage{threeparttable}
  \usepackage{threeparttablex}
  \usepackage[normalem]{ulem}
  \usepackage{rotating}
  \usepackage{makecell}
  \usepackage{xcolor}
  \usepackage{tikz} % required for image opacity change
  \usepackage[absolute,overlay]{textpos} % for text formatting
  \usepackage[utf8]{inputenc}
  \usetikzlibrary{mindmap}

  % this font option is amenable for beamer
  \setbeamerfont{caption}{size=\tiny}


%% IQSS overrides
\iqsssectiontitle{Outline}

\AtBeginSection[]{
  \title{\insertsectionhead}
  {
    \definecolor{white}{rgb}{0.776,0.357,0.157}
    \definecolor{iqss@orange}{rgb}{1,1,1}
    \ifnum \insertmainframenumber > \insertframenumber
    \frame{
      \frametitle{\iqsssectiontitleheader}
      \tableofcontents[currentsection]
    }
    \else
    \frame{
      \frametitle{Backup Slides}
      \tableofcontents[sectionstyle=shaded/shaded,subsectionstyle=shaded/shaded/shaded]
    }
    \fi
  }
}

\AtBeginSubsection[]{}

%%


  \title[]{Production relationships}



  \author[
        Deependra Dhakal
    ]{Deependra Dhakal}

  \institute[
    ]{
    GAASC, Baitadi \and Tribhuwan University
    }

\date[
      \today
  ]{
      \today
        }

\begin{document}

% Hide progress bar and footline on titlepage
  \begin{frame}[plain]
  \titlepage
  \end{frame}



\hypertarget{cost-concepts}{%
\section{Cost concepts}\label{cost-concepts}}

\begin{frame}{Background}
\protect\hypertarget{background}{}
\begin{itemize}
\tightlist
\item
  A farmer can increase his income by one of two ways:

  \begin{enumerate}
  \tightlist
  \item
    Increasing the production
  \item
    Reducing the cost of production
  \end{enumerate}
\item
  Normally in competitive market, prices are not in the control of
  individual farmer, hist additional production must, therefore, sell at
  same or even lower price.
\item
  Additionally, the cost of producing extra units of produce might
  involve higher costs.
\item
  The second alternative is to reduce the cost of production.
\item
  Often the cost of production is a policy issue when producer complain
  about not covering cost of production from the price they receive for
  their produce.
\end{itemize}
\end{frame}

\begin{frame}{Cash cost and non-cash cost}
\protect\hypertarget{cash-cost-and-non-cash-cost}{}
\begin{itemize}
\tightlist
\item
  In economics cost means the total efforts involved in the production
  of a commodity while expense of production only signify only money
  costs.
\item
  Cash costs are incurred when resources are purchased and used
  immediately in the production process.
\item
  Non-cash cost consist of depreciation and payments to resources owned
  by the farmers e.g., depreciation on tractor, equipment, buildings,
  payment made to the farmer himself or family labor, management and
  owned capital.
\end{itemize}
\end{frame}

\begin{frame}{Total cost}
\protect\hypertarget{total-cost}{}
\begin{itemize}
\tightlist
\item
  Fixed costs plus variable costs equal total costs. Total costs are
  required for computing net revenue. Net revenue is equal to total
  revenue less total cost.
\item
  Whether a particular cost item will be considered as fixed or variable
  one depends upon whether the input concerned is fixed or variable for
  the problem under consideration.
\item
  During long run planning period all inputs are variable. Thus in long
  run there are no fixed costs.
\item
  Total cost = Fixed cost + Variable cost
\end{itemize}
\end{frame}

\begin{frame}{Marginal cost}
\protect\hypertarget{marginal-cost}{}
\begin{itemize}
\tightlist
\item
  The additional cost of doing a little bit more (or 1 unit more if a
  unit can be measured) of an activity.
\item
  How do you make a rational decision about when the alarm should go
  off? What you have to do is to weigh up the costs and benefits of
  additional sleep. Each extra minute in bed gives you more sleep (the
  marginal benefit), but gives you more of a rush when you get up (the
  marginal cost).
\item
  The decision is therefore based on the costs and benefits of extra
  sleep, not on the total costs and benefits of a whole night's sleep.
\end{itemize}
\end{frame}

\begin{frame}{Characteristics of marginal cost}
\protect\hypertarget{characteristics-of-marginal-cost}{}
\end{frame}

\begin{frame}{Opportunity cost}
\protect\hypertarget{opportunity-cost}{}
\begin{itemize}
\tightlist
\item
  The farm resources have alternative uses.
\item
  The price that will be required to prevent the transferance of factors
  to other uses is called ``opportunity cost'' or ``alternative cost''.
\item
  The price which should be put on any input is therefore the return
  which must be given up in the next best alternative use.
\item
  Thus every resource used in production has one true cost: opportunity
  cost.
\item
  Suppose a farmer has 40 kgs of fertilizer; it adds Rs 250 to the total
  revenue from wheat and Rs 200 to the revenue from barley. If he
  fertilizes barley, his opportunity cost is Rs 250, which he has
  foregone on wheat. If he fertilizes wheat, his opportunity cost is Rs
  200, foregone on barley.
\end{itemize}
\end{frame}

\begin{frame}{}
\protect\hypertarget{section}{}
\begin{itemize}
\tightlist
\item
  Thus once purchased, market price of input becomes irrelevant to the
  problem of its allocation among alternative uses.
\item
  In case of durable resources such as machinery or land which are used
  in production, the opportunity cost is defined to be the amount that
  total capital investment could earn if invested in its best
  alternative use.
\item
  For simplest case, the opportunity cost will be interest that a
  deposit of money in a bank would fetch.
\end{itemize}
\end{frame}

\begin{frame}{Return concepts}
\protect\hypertarget{return-concepts}{}
\begin{itemize}
\tightlist
\item
  Gross return = Total production \(\times\) price
\item
  Returns to fixed farm resources(or returns over variable costs) =
  Gross returns - Variable cost
\item
  Net return = Gross return - Total cost
\end{itemize}
\end{frame}

\begin{frame}{Rational decision}
\protect\hypertarget{rational-decision}{}
\begin{itemize}
\tightlist
\item
  Doing more of an activity if its marginal benefit exceeds its marginal
  cost and doing less if its marginal cost exceeds its marginal benefit.
\item
  Rational decisions are made with rational choices; that involve
  weighing up the benefit of any activity against its opportunity cost.
\end{itemize}
\end{frame}

\begin{frame}{Cost function}
\protect\hypertarget{cost-function}{}
\begin{itemize}
\tightlist
\item
  The total cost curve or cost function represents the functional
  relationship between output and total cost.
\item
  It shows the change in cost structure when we produce different
  quantities of a commodity.
\item
  The exact nature of the Total Cost function depends on the nature of
  the corresponding production function, provided that the price which
  the producer pays for inputs does not change as the quantity of inputs
  purchased changes.
\end{itemize}
\end{frame}

\begin{frame}{Relationship between production function and total cost
function}
\protect\hypertarget{relationship-between-production-function-and-total-cost-function}{}
\end{frame}

\hypertarget{relationship-between-various-costs}{%
\section{Relationship between various
costs}\label{relationship-between-various-costs}}

\begin{frame}{TC, TFC and TVC}
\protect\hypertarget{tc-tfc-and-tvc}{}
Fig. 5.3 and explaination.
\end{frame}

\begin{frame}{ATC, AFC and AVC}
\protect\hypertarget{atc-afc-and-avc}{}
Fig. 5.4 and explaination.
\end{frame}

\begin{frame}{Relationship between average and marginal cost}
\protect\hypertarget{relationship-between-average-and-marginal-cost}{}
Fig. 5.5 and explaination.
\end{frame}

\hypertarget{bibliography}{%
\section{Bibliography}\label{bibliography}}

\begin{frame}{For more information}
\protect\hypertarget{for-more-information}{}
\end{frame}




\end{document}
