\PassOptionsToPackage{unicode=true}{hyperref} % options for packages loaded elsewhere
\PassOptionsToPackage{hyphens}{url}
\documentclass[12pt,ignorenonframetext,aspectratio=169]{beamer}
\IfFileExists{pgfpages.sty}{\usepackage{pgfpages}}{}
\setbeamertemplate{caption}[numbered]
\setbeamertemplate{caption label separator}{: }
\setbeamercolor{caption name}{fg=normal text.fg}
\beamertemplatenavigationsymbolsempty
\usepackage{lmodern}
\usepackage{amssymb}
\usepackage{amsmath}
\usepackage{ifxetex,ifluatex}
\usepackage{fixltx2e} % provides \textsubscript
\ifnum 0\ifxetex 1\fi\ifluatex 1\fi=0 % if pdftex
  \usepackage[T1]{fontenc}
  \usepackage[utf8]{inputenc}
\else % if luatex or xelatex
  \ifxetex
    \usepackage{mathspec}
  \else
    \usepackage{fontspec}
\fi
\defaultfontfeatures{Ligatures=TeX,Scale=MatchLowercase}






%
\fi

  \usetheme[]{iqss}






% use upquote if available, for straight quotes in verbatim environments
\IfFileExists{upquote.sty}{\usepackage{upquote}}{}
% use microtype if available
\IfFileExists{microtype.sty}{%
  \usepackage{microtype}
  \UseMicrotypeSet[protrusion]{basicmath} % disable protrusion for tt fonts
}{}


\newif\ifbibliography


\hypersetup{
      pdftitle={Product-product relationship},
        pdfauthor={Deependra Dhakal},
          pdfborder={0 0 0},
    breaklinks=true}
%\urlstyle{same}  % Use monospace font for urls







% Prevent slide breaks in the middle of a paragraph:
\widowpenalties 1 10000
\raggedbottom

  \AtBeginPart{
    \let\insertpartnumber\relax
    \let\partname\relax
    \frame{\partpage}
  }
  \AtBeginSection{
    \ifbibliography
    \else
      \let\insertsectionnumber\relax
      \let\sectionname\relax
      \frame{\sectionpage}
    \fi
  }
  \AtBeginSubsection{
    \let\insertsubsectionnumber\relax
    \let\subsectionname\relax
    \frame{\subsectionpage}
  }



\setlength{\parindent}{0pt}
\setlength{\parskip}{6pt plus 2pt minus 1pt}
\setlength{\emergencystretch}{3em}  % prevent overfull lines
\providecommand{\tightlist}{%
  \setlength{\itemsep}{0pt}\setlength{\parskip}{0pt}}

  \setcounter{secnumdepth}{0}


  \usepackage{booktabs}
  \usepackage{longtable}
  \usepackage{emptypage}
  \usepackage{array}
  \usepackage{multirow}
  \usepackage{wrapfig}
  \usepackage{float}
  \usepackage{colortbl}
  \usepackage{pdflscape}
  \usepackage{tabu}
  \usepackage{threeparttable}
  \usepackage{threeparttablex}
  \usepackage[normalem]{ulem}
  \usepackage{rotating}
  \usepackage{makecell}
  \usepackage{xcolor}
  \usepackage{tikz} % required for image opacity change
  \usepackage[absolute,overlay]{textpos} % for text formatting
  \usepackage[utf8]{inputenc}
  \usetikzlibrary{mindmap,arrows,shapes,positioning,shadows,trees}
  \usepackage[skip=2pt]{caption}

  % this font option is amenable for beamer
  \setbeamerfont{caption}{size=\tiny}


%% IQSS overrides
\iqsssectiontitle{Outline}

\AtBeginSection[]{
  \title{\insertsectionhead}
  {
    \definecolor{white}{rgb}{0.776,0.357,0.157}
    \definecolor{iqss@orange}{rgb}{1,1,1}
    \ifnum \insertmainframenumber > \insertframenumber
    \frame{
      \frametitle{\iqsssectiontitleheader}
      \tableofcontents[currentsection]
    }
    \else
    \frame{
      \frametitle{Backup Slides}
      \tableofcontents[sectionstyle=shaded/shaded,subsectionstyle=shaded/shaded/shaded]
    }
    \fi
  }
}

\AtBeginSubsection[]{}

%%


  \title[]{Product-product relationship}



  \author[
        Deependra Dhakal
    ]{Deependra Dhakal}

  \institute[
    ]{
    GAASC, Baitadi \and Tribhuwan University
    }

\date[
      \today
  ]{
      \today
        }

\begin{document}

% Hide progress bar and footline on titlepage
  \begin{frame}[plain]
  \titlepage
  \end{frame}



\hypertarget{product-product-relationship}{%
\section{Product-product
relationship}\label{product-product-relationship}}

\begin{frame}{Background}
\protect\hypertarget{background}{}
\begin{itemize}
\tightlist
\item
  Product-product relationship deals with resource allocation among
  competing enterprises.
\item
  Under product-product relationship, inputs are kept constant while
  product (outputs) are varied.
\item
  This relationship guides the producer in deciding ``what to produce
  ?''
\item
  This relationship is explained by principle of product substitution
  and law of equimarginal returns.
\end{itemize}
\end{frame}

\begin{frame}{Production-possibility frontier}
\protect\hypertarget{production-possibility-frontier}{}
\begin{itemize}
\tightlist
\item
  A production-possibility frontier shows the maximum number of
  alternative combinations of goods and services that a society can
  produce at a given time when there is full utilization of economic
  resources and technology.
\item
  Alternative combinations of Rice and Wheat output from a piece of land
  (Both crops are alloted randomly to the given parcel with no
  difference in use of other inputs) is shown in Table
  \ref{tab:ppschedule}.
\item
  In choosing what to produce, decision makers have a choice of
  producing, for example, alternative C-- 5 tons of Rice and 14 tons of
  Wheat-- or any other alternative presented.
\end{itemize}
\end{frame}

\begin{frame}{Production-possibility schedule}
\protect\hypertarget{production-possibility-schedule}{}
\begin{table}

\caption{\label{tab:ppschedule}Production possibility schedule}
\fontsize{8}{10}\selectfont
\begin{tabular}[t]{>{\centering\arraybackslash}p{10em}>{\centering\arraybackslash}p{10em}>{\centering\arraybackslash}p{10em}}
\toprule
Alternative outputs & Rice (tons) & Wheat (tons)\\
\midrule
\cellcolor[HTML]{FECE91}{\textcolor{white}{\textbf{A}}} & \bgroup\fontsize{8}{10}\selectfont \textcolor[HTML]{440154}{\textbf{0}}\egroup{} & \bgroup\fontsize{16}{18}\selectfont \textcolor[HTML]{7AD151}{\textbf{20}}\egroup{}\\
\cellcolor[HTML]{F2645C}{\textcolor{white}{\textbf{B}}} & \bgroup\fontsize{10}{12}\selectfont \textcolor[HTML]{453882}{\textbf{2}}\egroup{} & \bgroup\fontsize{15}{17}\selectfont \textcolor[HTML]{4CC26C}{\textbf{18}}\egroup{}\\
\cellcolor[HTML]{A1307E}{\textcolor{white}{\textbf{C}}} & \bgroup\fontsize{12}{14}\selectfont \textcolor[HTML]{2A788E}{\textbf{5}}\egroup{} & \bgroup\fontsize{14}{16}\selectfont \textcolor[HTML]{1F9F88}{\textbf{14}}\egroup{}\\
\cellcolor[HTML]{461078}{\textcolor{white}{\textbf{D}}} & \bgroup\fontsize{15}{17}\selectfont \textcolor[HTML]{4CC26C}{\textbf{9}}\egroup{} & \bgroup\fontsize{10}{12}\selectfont \textcolor[HTML]{3C508B}{\textbf{6}}\egroup{}\\
\cellcolor[HTML]{000004}{\textcolor{white}{\textbf{E}}} & \bgroup\fontsize{16}{18}\selectfont \textcolor[HTML]{7AD151}{\textbf{10}}\egroup{} & \bgroup\fontsize{8}{10}\selectfont \textcolor[HTML]{440154}{\textbf{0}}\egroup{}\\
\bottomrule
\end{tabular}
\end{table}

\begin{itemize}
\tightlist
\item
  The curve, labeled PP, is called the production-possibility frontier.
  Point C plots the combination of 5 tons of rice and 14 tons of wheat,
  assuming full employment of the economy's resources and full use of
  its technology, as do all of the alternatives presented in Table
  \ref{tab:ppschedule}.
\end{itemize}
\end{frame}

\begin{frame}{Production-possibility frontier}
\protect\hypertarget{production-possibility-frontier-1}{}
\begin{figure}
\includegraphics[width=0.6\linewidth]{06-product_product_relationship_files/figure-beamer/ppfrontier-1} \caption{Production possibility frontier}\label{fig:ppfrontier}
\end{figure}
\end{frame}

\hypertarget{enterprise-combination}{%
\section{Enterprise combination}\label{enterprise-combination}}

\begin{frame}{Relationships between products (enterprise relations)}
\protect\hypertarget{relationships-between-products-enterprise-relations}{}
\begin{enumerate}
\tightlist
\item
  Joint products
\item
  Complementary products
\item
  Competitive products
\item
  Supplementary products
\end{enumerate}
\end{frame}

\begin{frame}{Joint products}
\protect\hypertarget{joint-products}{}
\begin{itemize}
\tightlist
\item
  Products that are produced through single production process and the
  production of one without another is not possible.
\item
  These products are obtained in fixed proportions. If a given quantity
  of one product is produced, the quantity of other products is fixed by
  nature.
\item
  Joint products are produced through a single production function and
  for the purpose of analysis they may be treated as single product.
\item
  All farm commodities are mostly joint products. E.g. wheat and straw,
  groundnut and hulms, cotton seed and lint, cattle and manure, butter
  and milk etc.
\end{itemize}
\end{frame}

\begin{frame}{}
\protect\hypertarget{section}{}
\begin{figure}

{\centering \includegraphics[width=0.45\linewidth]{06-product_product_relationship_files/figure-beamer/joint-goods-1} 

}

\caption{Joint relationship rice straw and rice grain}\label{fig:joint-goods}
\end{figure}
\end{frame}

\begin{frame}{Complementary goods}
\protect\hypertarget{complementary-goods}{}
\begin{columns}
  \column{0.5\textwidth}

\begin{figure}

{\centering \includegraphics[width=0.6\textwidth]{06-product_product_relationship_files/figure-beamer/complementary-goods-1} 

}

\caption{Complementary relationship between wheat and legume system of cropping}\label{fig:complementary-goods}
\end{figure}

  \column{0.5\textwidth}
  \footnotesize
  \begin{itemize}
  \item When change in level of production of one occurs, another also changes in the same direction. i.e. when resource held constant the increase in the level of output of one product also increases in the level of another output. 
  \item In other words shift of resources from one product to a second product will increase rather than decrease the output of first. But this relation holds only upto certain level of production. 
  \item Leguminous crops increases the fertility status of soil, which is beneficial for production of wheat on a piece of land.
  \end{itemize}

\end{columns}
\end{frame}

\begin{frame}{Competitive goods}
\protect\hypertarget{competitive-goods}{}
\begin{columns}
    \column{0.33\textwidth}
    \centering

\begin{figure}

{\centering \includegraphics[width=0.95\textwidth]{06-product_product_relationship_files/figure-beamer/competitive-goods-constant-mrs-1} 

}

\caption{Competitive relationship between Rice and Wheat (constant marginal rate of product substitution)}\label{fig:competitive-goods-constant-mrs}
\end{figure}

      \column{0.33\textwidth}
      \footnotesize

\begin{figure}

{\centering \includegraphics[width=0.75\textwidth]{06-product_product_relationship_files/figure-beamer/competitive-goods-increasing-mrs-fig-1} 

}

\caption{Competitive relationship between Rice and Wheat (increasing marginal rate of product substitution)}\label{fig:competitive-goods-increasing-mrs-fig}
\end{figure}

\begin{table}[H]
\centering\begingroup\fontsize{5}{7}\selectfont

\resizebox{\linewidth}{!}{
\begin{tabular}{>{\raggedleft\arraybackslash}p{2.4em}>{\raggedleft\arraybackslash}p{2.4em}>{\raggedleft\arraybackslash}p{2.4em}>{\raggedleft\arraybackslash}p{2.4em}>{\raggedleft\arraybackslash}p{3.2em}}
\toprule
rice & wheat & $\Delta \textrm{rice}$ & $\Delta \textrm{wheat}$ & $MRS \newline (\frac{\Delta wheat}{\Delta rice})$\\
\midrule
2.0 & 1.92 &  &  & \\
2.1 & 1.87 & 0.1 & -0.05 & 0.46\\
2.2 & 1.78 & 0.1 & -0.09 & 0.94\\
2.3 & 1.59 & 0.1 & -0.19 & 1.94\\
2.4 & 1.38 & 0.1 & -0.21 & 2.10\\
\addlinespace
2.5 & 1.00 & 0.1 & -0.38 & 3.76\\
2.6 & 0.44 & 0.1 & -0.56 & 5.60\\
\bottomrule
\end{tabular}}
\endgroup{}
\end{table}


      \column{0.33\textwidth}
      \footnotesize

\begin{figure}

{\centering \includegraphics[width=0.75\textwidth]{06-product_product_relationship_files/figure-beamer/competitive-goods-decreasing-mrs-fig-1} 

}

\caption{Competitive relationship between Rice and Wheat (decreasing marginal rate of product substitution)}\label{fig:competitive-goods-decreasing-mrs-fig}
\end{figure}

\begin{table}[H]
\centering\begingroup\fontsize{5}{7}\selectfont

\resizebox{\linewidth}{!}{
\begin{tabular}{>{\raggedleft\arraybackslash}p{2.4em}>{\raggedleft\arraybackslash}p{2.4em}>{\raggedleft\arraybackslash}p{2.4em}>{\raggedleft\arraybackslash}p{2.4em}>{\raggedleft\arraybackslash}p{3.2em}}
\toprule
rice & wheat & $\Delta \textrm{rice}$ & $\Delta \textrm{wheat}$ & $MRS \newline (\frac{\Delta wheat}{\Delta rice})$\\
\midrule
2.0 & 3.2 &  &  & \\
2.1 & 2.8 & 0.1 & -0.36 & 3.60\\
2.2 & 2.6 & 0.1 & -0.22 & 2.24\\
2.3 & 2.4 & 0.1 & -0.20 & 2.02\\
2.4 & 2.2 & 0.1 & -0.17 & 1.74\\
\addlinespace
2.5 & 2.1 & 0.1 & -0.09 & 0.90\\
2.6 & 2.1 & 0.1 & -0.05 & 0.46\\
\bottomrule
\end{tabular}}
\endgroup{}
\end{table}


 \end{columns}
\end{frame}

\begin{frame}{}
\protect\hypertarget{section-1}{}
\footnotesize

\begin{itemize}
\tightlist
\item
  When increase in the production of one product, with resources held
  constant results in the decrease in the output of the other product.
\item
  The marginal rate of product substitution, which indicates the
  quantity of one product that must be given up, when the output of
  other product is increased by one unit, is negative.
\item
  Marginal Rate of Product Substitution (MRPS) =
  \(\large \frac{\Delta Y_2}{\Delta Y_1}\), where \(Y_1\) and \(Y_2\)
  are the two products of a competitive relationship.
\end{itemize}
\end{frame}

\begin{frame}{Supplementary good}
\protect\hypertarget{supplementary-good}{}
\begin{columns}
  \column{0.45\textwidth}

\begin{figure}

{\centering \includegraphics[width=0.95\textwidth]{06-product_product_relationship_files/figure-beamer/supplementary-goods-1} 

}

\caption{Supplementary relationship between Goat and wheat production (After reaching certain level of production relation becomes competitive)}\label{fig:supplementary-goods}
\end{figure}

  \column{0.55\textwidth}
  \footnotesize
  
  \begin{itemize}
  \item When increase or decrease in the output of one product does not affect the production of the other product.
  \item With the same resources, the output of one product can be increased with neither a gain nor a sacrifice in the other product.
  \item Supplementary products use the idle resources. 
  \item On small farms keeping a few goats undertaking wheat cultivation may be enterprises because permanent labour is used in crop without reducing the productivity of goat farming.
  \end{itemize}

\end{columns}
\end{frame}

\hypertarget{production-optimization-concepts}{%
\section{Production optimization
concepts}\label{production-optimization-concepts}}

\begin{frame}{Iso revenue line}
\protect\hypertarget{iso-revenue-line}{}
\begin{itemize}
\tightlist
\item
  It represents all possible combination of two products which would
  yield an equal (same) revenue or income.
\end{itemize}

\textbf{Characteristics}

\begin{enumerate}
\tightlist
\item
  Iso revenue line is a straight line because product prices do not
  change with quantity sold
\item
  As the Total Revenue increases the iso revenue line moves away from
  origin
\item
  The slope indicates ratio of product prices. As long as product prices
  remain constant iso revenue lines are parallel.
\end{enumerate}
\end{frame}

\begin{frame}{}
\protect\hypertarget{section-2}{}
Suppose a farmer intends to earn Rs 10000 of revenue from sales of two
competative ouputs. His isorevenue line is shown in Figure
\ref{fig:isorevenue-line-rice-wheat}.

\begin{figure}
\includegraphics[width=0.45\linewidth]{06-product_product_relationship_files/figure-beamer/isorevenue-line-rice-wheat-1} \caption{Isorevenue line for different quantity combination of Rice and Wheat grain sales}\label{fig:isorevenue-line-rice-wheat}
\end{figure}
\end{frame}

\begin{frame}{Determination of optimum combination of products}
\protect\hypertarget{determination-of-optimum-combination-of-products}{}
\begin{enumerate}
\tightlist
\item
  Tabular method
\item
  Arithmetic method
\item
  Graphical method
\end{enumerate}
\end{frame}




\end{document}
