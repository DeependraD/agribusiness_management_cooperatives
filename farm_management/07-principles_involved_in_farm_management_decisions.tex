\PassOptionsToPackage{unicode=true}{hyperref} % options for packages loaded elsewhere
\PassOptionsToPackage{hyphens}{url}
\documentclass[12pt,ignorenonframetext,aspectratio=169]{beamer}
\IfFileExists{pgfpages.sty}{\usepackage{pgfpages}}{}
\setbeamertemplate{caption}[numbered]
\setbeamertemplate{caption label separator}{: }
\setbeamercolor{caption name}{fg=normal text.fg}
\beamertemplatenavigationsymbolsempty
\usepackage{lmodern}
\usepackage{amssymb}
\usepackage{amsmath}
\usepackage{ifxetex,ifluatex}
\usepackage{fixltx2e} % provides \textsubscript
\ifnum 0\ifxetex 1\fi\ifluatex 1\fi=0 % if pdftex
  \usepackage[T1]{fontenc}
  \usepackage[utf8]{inputenc}
\else % if luatex or xelatex
  \ifxetex
    \usepackage{mathspec}
  \else
    \usepackage{fontspec}
\fi
\defaultfontfeatures{Ligatures=TeX,Scale=MatchLowercase}






%
\fi

  \usetheme[]{iqss}






% use upquote if available, for straight quotes in verbatim environments
\IfFileExists{upquote.sty}{\usepackage{upquote}}{}
% use microtype if available
\IfFileExists{microtype.sty}{%
  \usepackage{microtype}
  \UseMicrotypeSet[protrusion]{basicmath} % disable protrusion for tt fonts
}{}


\newif\ifbibliography


\hypersetup{
      pdftitle={Principles involved in farm management decisions},
        pdfauthor={Deependra Dhakal},
          pdfborder={0 0 0},
    breaklinks=true}
%\urlstyle{same}  % Use monospace font for urls







% Prevent slide breaks in the middle of a paragraph:
\widowpenalties 1 10000
\raggedbottom

  \AtBeginPart{
    \let\insertpartnumber\relax
    \let\partname\relax
    \frame{\partpage}
  }
  \AtBeginSection{
    \ifbibliography
    \else
      \let\insertsectionnumber\relax
      \let\sectionname\relax
      \frame{\sectionpage}
    \fi
  }
  \AtBeginSubsection{
    \let\insertsubsectionnumber\relax
    \let\subsectionname\relax
    \frame{\subsectionpage}
  }



\setlength{\parindent}{0pt}
\setlength{\parskip}{6pt plus 2pt minus 1pt}
\setlength{\emergencystretch}{3em}  % prevent overfull lines
\providecommand{\tightlist}{%
  \setlength{\itemsep}{0pt}\setlength{\parskip}{0pt}}

  \setcounter{secnumdepth}{0}


  \usepackage{booktabs}
  \usepackage{longtable}
  \usepackage{emptypage}
  \usepackage{array}
  \usepackage{multirow}
  \usepackage{wrapfig}
  \usepackage{float}
  \usepackage{colortbl}
  \usepackage{pdflscape}
  \usepackage{tabu}
  \usepackage{threeparttable}
  \usepackage{threeparttablex}
  \usepackage[normalem]{ulem}
  \usepackage{rotating}
  \usepackage{makecell}
  \usepackage{xcolor}
  \usepackage{tikz} % required for image opacity change
  \usepackage[absolute,overlay]{textpos} % for text formatting
  \usepackage[utf8]{inputenc}
  \usetikzlibrary{mindmap,arrows,shapes,positioning,shadows,trees}
  \usepackage[skip=2pt]{caption}

  % this font option is amenable for beamer
  \setbeamerfont{caption}{size=\tiny}


%% IQSS overrides
\iqsssectiontitle{Outline}

\AtBeginSection[]{
  \title{\insertsectionhead}
  {
    \definecolor{white}{rgb}{0.776,0.357,0.157}
    \definecolor{iqss@orange}{rgb}{1,1,1}
    \ifnum \insertmainframenumber > \insertframenumber
    \frame{
      \frametitle{\iqsssectiontitleheader}
      \tableofcontents[currentsection]
    }
    \else
    \frame{
      \frametitle{Backup Slides}
      \tableofcontents[sectionstyle=shaded/shaded,subsectionstyle=shaded/shaded/shaded]
    }
    \fi
  }
}

\AtBeginSubsection[]{}

%%


  \title[]{Principles involved in farm management decisions}



  \author[
        Deependra Dhakal
    ]{Deependra Dhakal}

  \institute[
    ]{
    GAASC, Baitadi \and Tribhuwan University
    }

\date[
      \today
  ]{
      \today
        }

\begin{document}

% Hide progress bar and footline on titlepage
  \begin{frame}[plain]
  \titlepage
  \end{frame}



\hypertarget{principle-of-diminishing-return-principle-of-variable-proportion}{%
\section{Principle of diminishing return (Principle of variable
proportion)}\label{principle-of-diminishing-return-principle-of-variable-proportion}}

\begin{frame}{Background}
\protect\hypertarget{background}{}
\footnotesize

\begin{itemize}
\tightlist
\item
  It explains the basic relationship between factor and product.
\item
  This law derives its name from the fact that as successive units of
  variable resource are used in the combination with a collection of
  fixed resources, the resulting addition to the total product will
  become successively smaller.
\item
  How much to produce? (Optimum production level) It guides in the
  determination of optimum output level to produce.
\item
  How much input to use?

  \begin{itemize}
  \tightlist
  \item
    Production function determines how much inputs are used to produce
    the product.
  \item
    Given goal of maximizing profit the farmer must select from all
    possible input levels, the one which will result in the greatest
    profit.
  \end{itemize}
\end{itemize}
\end{frame}

\begin{frame}{Input determination}
\protect\hypertarget{input-determination}{}
\begin{enumerate}
\tightlist
\item
  Marginal value product (MVP): It is the additional income received
  from using an additional unit in input. It is calculated by using
  following equation:
\end{enumerate}

\[
MVP = \frac{\Delta TR}{\Delta X} = \frac{\Delta Y}{\Delta X}\times P_y
\] \footnotesize

Where,

\(P_y\) = Price of output \(\Delta X\) = Change in input \(\Delta Y\) =
Change in output
\end{frame}

\begin{frame}{}
\protect\hypertarget{section}{}
\begin{enumerate}
\setcounter{enumi}{1}
\tightlist
\item
  Marginal input cost (MIC) or Marginal factor cost (MFC): It is defined
  as the additional cost associated with the use of an additional unit
  of input.
\end{enumerate}

\[
MFC = \frac{\Delta TC}{\Delta X} = P_x
\] \footnotesize

Where,

\(P_x\) = Price of input \(\Delta X\) = Change in input \(\Delta TC\) =
Change in total cost

\begin{itemize}
\tightlist
\item
  The MFC remains constant regardless of how much of the variable input
  is used.
\end{itemize}
\end{frame}

\begin{frame}{Decision rules}
\protect\hypertarget{decision-rules}{}
\begin{itemize}
\tightlist
\item
  If MVP \textgreater{} MIC, additional profit can be made using more
  input
\item
  If MVP \textless{} MIC, more profit can be made by using less input
\item
  Profit is not maximized where production is maximized because
  production is maximized where MVP is \$0. Accordingly, MIC would have
  to be \$0 (that is, free) to maximize profit where production is
  maximized. If variable input has a cost, maximum profit is at some
  level of production less than maximum production.
\item
  Profit maximization or optimum input level is at the point where MVP =
  MIC, or,
\end{itemize}

\[
 \frac{\Delta Y}{\Delta X} \times P_y = P_x
\]
\end{frame}

\begin{frame}{Output determination}
\protect\hypertarget{output-determination}{}
\begin{enumerate}
\tightlist
\item
  Marginal revenue (MR): It is defined as the additional income from
  selling additional unit of output. It is calculated from the following
  equation:
\end{enumerate}

\[
MR = \frac{\Delta TR}{\Delta Y} = \frac{\Delta Y}{\Delta Y}\times P_y = P_y
\]

\begin{enumerate}
\setcounter{enumi}{1}
\tightlist
\item
  Marginal cost (MC): It is defined as the additional cost incurred from
  producing an additional unit of output. It is computed from the
  following equation:
\end{enumerate}

\[
MC = \frac{\Delta X}{\Delta Y} \times P_x
\]
\end{frame}

\begin{frame}{Decision rules}
\protect\hypertarget{decision-rules-1}{}
\begin{itemize}
\tightlist
\item
  If MR \textgreater{} MC, additional profit can be made by producing
  more output
\item
  If MR \textless{} MC, more profit can be made by producing less output
\item
  The profit maximizing output level is at the point where, MR = MC
\end{itemize}
\end{frame}

\begin{frame}{Applications of principle of diminishing returns}
\protect\hypertarget{applications-of-principle-of-diminishing-returns}{}
\footnotesize

\begin{enumerate}
\tightlist
\item
  Role of nature: The agricultural production greatly depends on the
  nature.
\item
  Land is a fixed factor: The supply of land is fixed, that is, we
  cannot increase the land. We have to work within that limited land.
\item
  Less scope of division of labor: Unlike in industry, we cannot make
  specialization in agricultural activities. A family has to plough,
  dig, transplant, harvest, and store, sell altogether. No one is
  specialized for a particular activity, though there are some cultural
  or gender divisions like transplanting by women, ploughing by men etc.
\item
  Less use of machines: Due to less use of machines, there is
  diminishing return.
\item
  Seasonal nature of work: The agricultural labor is engaged in
  particular season. In other season, they may be idle. So the total
  productivity with respect to the inputs go on decreasing.
\item
  Less supervision: The agriculture business really gets less
  supervision. In the field no frequent supervision is possible like in
  closed industry.
\end{enumerate}
\end{frame}

\hypertarget{cost-principle}{%
\section{Cost principle}\label{cost-principle}}

\begin{frame}{Background}
\protect\hypertarget{background-1}{}
\begin{itemize}
\tightlist
\item
  Cost of production of any commodity is the value of the factors of
  production used in producing that unit.
\item
  Producers must take into account cost of production while making
  production decisions.
\end{itemize}
\end{frame}

\begin{frame}{Cost accounting period and cost curves}
\protect\hypertarget{cost-accounting-period-and-cost-curves}{}
\begin{itemize}
\tightlist
\item
  Generally there are two types of cost accounting periods as so are the
  curves -- short run and long run cost.
\item
  The terms short and long are not absolute time based, rather it is
  based on use/employment of resources.
\item
  In long run, none of the inputs are fixed, and all are variable.
\item
  In short run, some inputs are fixed (at least one) and others may be
  variable.
\end{itemize}
\end{frame}

\begin{frame}{Short run cost curve}
\protect\hypertarget{short-run-cost-curve}{}
\begin{itemize}
\tightlist
\item
  The analysis of costs of a firm is based on the theory of production.
  The behavior of the cost shows the behavior of the production.
\item
  The cost is the product of quantity of inputs and the respective
  price. When we transpose the production function (on cost basis) from
  `X'-axis to `Y'-axis, then it becomes the cost curve.
\end{itemize}
\end{frame}

\begin{frame}{Fixed, variable and total costs in the short-run}
\protect\hypertarget{fixed-variable-and-total-costs-in-the-short-run}{}
\begin{table}

\caption{\label{tab:cost-short-run}Fixed, variable and total costs in short run}
\centering
\fontsize{6}{8}\selectfont
\begin{tabular}[t]{>{\raggedleft\arraybackslash}p{7em}>{\raggedleft\arraybackslash}p{7em}>{\raggedleft\arraybackslash}p{7em}>{\raggedleft\arraybackslash}p{7em}>{\raggedleft\arraybackslash}p{7em}>{\raggedleft\arraybackslash}p{10em}}
\toprule
Number of workers (L) & Output (Wheat yield in Quintals; Q) & Daily wages per worker (W) & Total variable costs (TVC) & Total fixed costs (TFC) & Short run total cost (TVC + TFC)\\
\midrule
\rowcolor{gray!6}  0 & 0.0 & 400 & 0 & 1500 & 1500\\
1 & 1.0 & 400 & 400 & 1500 & 1900\\
\rowcolor{gray!6}  2 & 2.2 & 400 & 800 & 1500 & 2300\\
3 & 3.6 & 400 & 1200 & 1500 & 2700\\
\rowcolor{gray!6}  4 & 5.2 & 400 & 1600 & 1500 & 3100\\
5 & 7.0 & 400 & 2000 & 1500 & 3500\\
\rowcolor{gray!6}  6 & 8.6 & 400 & 2400 & 1500 & 3900\\
7 & 10.0 & 400 & 2800 & 1500 & 4300\\
\rowcolor{gray!6}  8 & 11.2 & 400 & 3200 & 1500 & 4700\\
9 & 12.2 & 400 & 3600 & 1500 & 5100\\
\rowcolor{gray!6}  10 & 13.0 & 400 & 4000 & 1500 & 5500\\
11 & 13.7 & 400 & 4400 & 1500 & 5900\\
\rowcolor{gray!6}  12 & 14.3 & 400 & 4800 & 1500 & 6300\\
13 & 14.8 & 400 & 5200 & 1500 & 6700\\
\rowcolor{gray!6}  14 & 15.2 & 400 & 5600 & 1500 & 7100\\
15 & 15.5 & 400 & 6000 & 1500 & 7500\\
\bottomrule
\end{tabular}
\end{table}
\end{frame}

\begin{frame}{}
\protect\hypertarget{section-1}{}
\begin{figure}
\includegraphics[width=0.5\linewidth]{07-principles_involved_in_farm_management_decisions_files/figure-beamer/cost-short-run-plot-1} \caption{Short run total cost curve}\label{fig:cost-short-run-plot}
\end{figure}
\end{frame}

\begin{frame}{}
\protect\hypertarget{section-2}{}
\begin{itemize}
\tightlist
\item
  The cost curves are of short-run period in the Figure
  \ref{fig:cost-short-run-plot}; some inputs are fixed and other are
  variable.
\item
  It means whether we produce or not, there incurs a cost that is called
  the fixed cost.
\item
  Variable cost are those that vary according to the level of
  production.
\item
  Fixed cost remain the same throughout all levels of production.
\item
  If we try to produce more and more with same level of fixed inputs,
  the variable cost curve becomes too steeper.
\end{itemize}
\end{frame}

\begin{frame}{Long run cost curve}
\protect\hypertarget{long-run-cost-curve}{}
\begin{itemize}
\tightlist
\item
  In long run all inputs are variable.
\item
  The long-run may be a month, a year or decade.
\item
  In long -run, the scale of production is varied by varying all the
  inputs.
\end{itemize}

\begin{figure}
\includegraphics[width=0.4\linewidth]{07-principles_involved_in_farm_management_decisions_files/figure-beamer/cost-long-run-plot-1} \caption{Long run total cost curve}\label{fig:cost-long-run-plot}
\end{figure}
\end{frame}

\begin{frame}{Application of cost principle}
\protect\hypertarget{application-of-cost-principle}{}
\begin{itemize}
\tightlist
\item
  To determine shut down point
\item
  To determine break even point
\item
  To know economies of scale of production
\end{itemize}
\end{frame}

\hypertarget{principle-of-factor-substitutionprinciple-of-marginal-rate-of-technical-substiutionprinciple-of-least-cost-combination}{%
\section{Principle of factor substitution/Principle of marginal rate of
technical substiution/Principle of least cost
combination}\label{principle-of-factor-substitutionprinciple-of-marginal-rate-of-technical-substiutionprinciple-of-least-cost-combination}}

\begin{frame}{}
\protect\hypertarget{section-3}{}
\begin{itemize}
\tightlist
\item
  This law or principle in farming is applied in a situation in which
  the farmer has a number of options are open for a number of
  alternatives available among different resource inputs for a
  particular operation.
\item
  He will choose the best alternative, which adds more in output while
  at the same time minimizes the cost.
\end{itemize}
\end{frame}

\begin{frame}{Procedures for working out the least-cost combination}
\protect\hypertarget{procedures-for-working-out-the-least-cost-combination}{}
\begin{enumerate}
\tightlist
\item
  Step I: Compute the substitution ratio i.e.~MRTS, For example MRTS of
  X1 for X2 will be
\end{enumerate}

\[\frac{\textrm{Change in X2 (quantity of replaced input)}}{ \textrm{Change in X1 (quantity of added input)}} = \frac{\Delta X_2}{\Delta X_1}\]

\begin{enumerate}
\setcounter{enumi}{1}
\item
  Step II: Compute the price ratio, i.e.~ratio of price of the added
  input (PX1) and price of the replaced input (PX2), mathematically,
  \(\frac{P_{X_1}}{P_{X_2}}\).
\item
  Step III: Find the point where the substitution ratio and the price
  ratio is equal.
\end{enumerate}
\end{frame}

\begin{frame}{Decision rule for profit maximization}
\protect\hypertarget{decision-rule-for-profit-maximization}{}
\begin{enumerate}
\tightlist
\item
  If the substitution ratio is greater than the price ratio, one can
  reduce the costs by using more of the ``added'' resource. That is, if
  DX2/DX1 \textgreater{} PX1/PX2, use more X1 till equality.
\item
  If the substitution ratio is less than the price ratio, costs can be
  reduced by using more of ``replaced'' resource. That is, if DX2/DX1
  \textless{} PX1/PX2, use more X2 till equality.
\item
  If the substitution ratio is equal to the price ratio, then it is the
  point of least cost.
\end{enumerate}
\end{frame}

\hypertarget{principle-of-product-substitutionprinciple-of-enterprise-combination}{%
\section{Principle of product substitution/Principle of enterprise
combination}\label{principle-of-product-substitutionprinciple-of-enterprise-combination}}

\begin{frame}{}
\protect\hypertarget{section-4}{}
\begin{itemize}
\tightlist
\item
  It solves the problem of ``what to produce''.
\item
  Guides determination of optimum combination of enterprises (products).
  It explains product-product relationship.
\item
  It is economical to substitute one product for another product, if the
  decrease in returns from the product being replaced is less than the
  increase in returns from the product being added.
\item
  The principle of product substitution says that we should go in
  increasing the output of a product so long as decrease in the returns
  from the product being replaced is less than the increase in the
  returns from the product being added.
\end{itemize}
\end{frame}

\begin{frame}{Decision rule}
\protect\hypertarget{decision-rule}{}
\begin{itemize}
\tightlist
\item
  If marginal rate of product substitution (MRPS) is greater than price
  ratio (PR) (\(MRPS (Y_1 \textrm{ for } Y_2) > PR (Y_1 Y_2)\)):

  \begin{itemize}
  \tightlist
  \item
    profits can be increased by producing more of replaced product.
  \item
    either increase production of Y1 or decrease Y2 or both.
  \end{itemize}
\item
  If MRPS \textless{} PR
  (\(MRPS (Y_1 \textrm{ for } Y_2) > PR (Y_1 Y_2)\)):

  \begin{itemize}
  \tightlist
  \item
    profits can be increased by producing more of added product.
  \item
    either decrease production of Y1 or increase Y2 or both.
  \end{itemize}
\item
  Optimum combination of products is when MRPS = PR.
\end{itemize}
\end{frame}

\hypertarget{principle-of-opportunity-cost}{%
\section{Principle of opportunity
cost}\label{principle-of-opportunity-cost}}

\begin{frame}{}
\protect\hypertarget{section-5}{}
\begin{itemize}
\tightlist
\item
  Opportunity cost recognizes the fact that every input has an
  alternative use. Once an input is committed to a particular use, it is
  no longer available for any other alternative use and the income from
  the alternative must be foregone.
\item
  Opportunity cost is defined as the returns that are sacrificed from
  the next best alternative.
\item
  Opportunity cost is also known as real cost or alternate cost.
\item
  Its concept is closely related to the equi-marginal principle.
\end{itemize}
\end{frame}

\begin{frame}{Examples}
\protect\hypertarget{examples}{}
\begin{itemize}
\tightlist
\item
  The opportunity cost of using a machine to produce one product is the
  earnings that would be possible from other products.
\item
  The opportunity cost of using a machine that is useless for any other
  purpose is nil since its use requires no sacrifice of other
  opportunities.
\item
  The opportunity cost of the time one puts into his own business is the
  salary he could earn in other occupations (with a correction for the
  relative psychic income in the two occupations).
\end{itemize}
\end{frame}

\hypertarget{principle-of-comparative-advantage}{%
\section{Principle of comparative
advantage}\label{principle-of-comparative-advantage}}

\begin{frame}{}
\protect\hypertarget{section-6}{}
\begin{itemize}
\tightlist
\item
  It explains regional specialization in the production of commodities.
\item
  Certain crops can be grown in only limited areas because of specific
  soil and climatic requirement. However, even those crop and livestock
  enterprises which can be raised over broad geographical area often
  have production concentrated in one region.
\item
  Regional specialization in the production of the agricultural
  commodities and other products can be explained by the principle of
  comparative advantage.
\item
  The yields, production costs, profits may be different in different
  areas.
\item
  It is relative yields, costs, and profits which are important for the
  application of this principle.
\end{itemize}
\end{frame}

\begin{frame}{Statement of principle}
\protect\hypertarget{statement-of-principle}{}
Individuals or regions will tend to specialize in the production of
those commodities for which resources give them a relative or
comparative advantage.

\begin{tabular}{lrrrr}
\toprule
\multicolumn{1}{c}{ } & \multicolumn{2}{c}{Region A} & \multicolumn{2}{c}{Region B} \\
\cmidrule(l{3pt}r{3pt}){2-3} \cmidrule(l{3pt}r{3pt}){4-5}
Crop account & Wheat & Groundnut & Wheat & Groundnut\\
\midrule
Total cost & 425.0 & 200.0 & 210.0 & 200.0\\
Total return & 500.0 & 225.0 & 225.0 & 200.0\\
Net return & 75.0 & 25.0 & 15.0 & 20.0\\
Rate of return & 1.2 & 1.1 & 1.1 & 1.1\\
\bottomrule
\end{tabular}
\end{frame}

\begin{frame}{}
\protect\hypertarget{section-7}{}
\begin{itemize}
\tightlist
\item
  Region A has greater absolute advantage in growing both wheat and
  groundnut than Region B because the net incomes per acre are Rs 75 and
  Rs 25 respectively which are higher than the net incomes from wheat
  and groundnut in Region B. Farmers of Region A can make more profits
  by growing both the crops.
\item
  But they want to make the greatest profits and this can be done by
  having the largest possible acerage under Wheat alone as it is the
  question of relative advantage.
\item
  Similarly farmers of Region of B have relative advantage in growing
  groundnut.
\end{itemize}
\end{frame}

\hypertarget{principle-of-equimarginal-returns}{%
\section{Principle of
equimarginal-returns}\label{principle-of-equimarginal-returns}}

\begin{frame}{}
\protect\hypertarget{section-8}{}
\begin{itemize}
\tightlist
\item
  In input-output relationship, MC=MR is the economic principle used to
  determine the most profitable level of variable input.
\item
  But it is under the assumption of unlimited availability of variable
  input, which is unrealistic.
\item
  If a scarce resource is to be distributed among two or more uses, the
  highest total return is obtained when the marginal return per unit of
  resource is equal in all alternative uses. -- \textbf{Principle of
  equi-marginal returns}.
\end{itemize}
\end{frame}

\begin{frame}{One input -- several products}
\protect\hypertarget{one-input-several-products}{}
\footnotesize

\begin{itemize}
\tightlist
\item
  Suppose there is a limited amount of a variable input to be allocated
  among several enterprises.
\item
  For this, the production function and product prices must be known for
  each enterprise. Next, the MVP schedule must be computed for each
  enterprise.
\item
  Finally, using the opportunity cost principle, units of input are
  allocated to each enterprise in such a way that profit earned by the
  input is maximum.
\item
  Profit from a limited amount of variable resource is maximized when
  the resource is allocated among the enterprises in such as way that
  marginal earnings of the input are equal in all enterprises. i.e.,
\end{itemize}

\[
VMP_{xy_{1}} = VMP_{xy_{2}} = \ldots = VMP_{xy_{n}}
\] Where, \(VMP_{xy_{1}}\) is the value of marginal product of \(X\)
used on product \(Y_1\); \(VMP_{xy_{2}}\) is the value of marginal
product of \(X\) used on product \(Y_2\) and so on.
\end{frame}

\begin{frame}{}
\protect\hypertarget{section-9}{}
\begin{table}[H]
\centering
\resizebox{\linewidth}{!}{
\begin{tabular}{rrrrrr>{\raggedleft\arraybackslash}p{4em}>{\raggedleft\arraybackslash}p{4em}>{\raggedleft\arraybackslash}p{4em}}
\toprule
x maize & y1 maize & x rice & y2 rice & x sorghum & y3 sorghum & vmp xy1 maize & vmp xy2 rice & vmp xy2 sorghum\\
\midrule
0 & 0 & 0 & 0 & 0 & 0 &  &  & \\
1 & 10 & 1 & 18 & 1 & 7 & 20 & 18 & 14\\
2 & 18 & 2 & 31 & 2 & 13 & 16 & 13 & 12\\
3 & 24 & 3 & 42 & 3 & 18 & 12 & 11 & 10\\
4 & 29 & 4 & 51 & 4 & 22 & 10 & 9 & 8\\
\addlinespace
5 & 33 & 5 & 58 & 5 & 25 & 8 & 7 & 6\\
6 & 36 & 6 & 64 & 6 & 27 & 6 & 6 & 4\\
\bottomrule
\end{tabular}}
\end{table}
\end{frame}

\begin{frame}{}
\protect\hypertarget{section-10}{}
\footnotesize

Suppose that the farmer has five units of X. According to the
opportunity cost principle, he will allocate each successive units of
input to the use where its marginal return, VMP, is the largest; i.e.,
first unit of X used in Enterprise I (Maize) earns Rs 20; second on
first unit of Enterprise II (Rice) earns Rs 18; third on second unit of
Enterprise I earns Rs 16; fourth on first unit of Enterprise III
(Sorghum) earns Rs 14; and 5th on second unit of enterprise II earns Rs
13.

Two units of inputs go on enterprise I, two on II and one on III. Used
in this manner, the five units of inputs will earn Rs 81. No other
allocation of the five units among the three enterprise will earn as
much.
\end{frame}

\begin{frame}{}
\protect\hypertarget{section-11}{}
\begin{itemize}
\item
  What is the maximum amount of input needed for enterprises I, II and
  III ?
\item
  To determine this, manager must determine the most profitable amount
  of input for each enterprise. When input cost is Rs 6.5 per unit, the
  optimum amounts are 5 for I, 5 for II and 4 for III. Cost is Rs 91 ((5
  + 5 + 4 = 14)*6.5 = 91). Thus, the manager would never use more than a
  total of 14 units of inputs, no matter how many units of inputs he
  could afford to buy.
\end{itemize}
\end{frame}

\hypertarget{principle-of-time-comparison}{%
\section{Principle of time
comparison}\label{principle-of-time-comparison}}

\begin{frame}{}
\protect\hypertarget{section-12}{}
\begin{itemize}
\tightlist
\item
  The time value of money is the idea that there is greater benefit to
  receiving a sum of money now rather than an identical sum later. It is
  founded on time preference.
\item
  The time value of money is the reason why interest is paid or earned:
  interest, whether it is on a bank deposit or debt, compensates the
  depositor or lender for the time value of money.
\end{itemize}
\end{frame}

\begin{frame}{}
\protect\hypertarget{section-13}{}
\begin{itemize}
\tightlist
\item
  Time value of money problems involve the net value of cash flows at
  different points in time.
\item
  In a typical case, the variables might be: a balance (the real or
  nominal value of a debt or a financial asset in terms of monetary
  units), a periodic rate of interest, the number of periods, and a
  series of cash flows.
\item
  For example, Rs 100 invested for one year, earning 5\% interest, will
  be worth Rs 105 after one year; therefore, Rs 100 paid now and Rs 105
  paid exactly one year later both have the same value to a recipient
  who expects 5\% interest assuming that inflation would be zero
  percent.
\end{itemize}
\end{frame}

\begin{frame}{Future value of a present sum}
\protect\hypertarget{future-value-of-a-present-sum}{}
The future value (FV) formula is:

\begin{equation}
FV = PV(1 + i)^n
\label{eqn:future-value}
\end{equation}
\end{frame}

\begin{frame}{Present value of a future sum}
\protect\hypertarget{present-value-of-a-future-sum}{}
\footnotesize

The present value formula is the core formula for the time value of
money; each of the other formulae is derived from this formula. For
example, the annuity formula is the sum of a series of present value
calculations.

The present value (PV) formula has four variables, each of which can be
solved for by numerical methods:

\begin{equation}
PV = \frac{FV}{(1 + i)^n}
\label{eqn:current-value}
\end{equation}
\end{frame}

\begin{frame}{}
\protect\hypertarget{section-14}{}
The cumulative present value of future cash flows can be calculated by
summing the contributions of \(FV_t\), the value of cash flow at time
\(t\);

\[
PV = \sum_{t = 1}^n \frac{FV_t}{(1 + i)^t}
\]

\begin{itemize}
\tightlist
\item
  This principle allows for the valuation of a likely stream of income
  in the future, in such a way that annual incomes are discounted and
  then added together, thus providing a lump-sum ``present value'' of
  the entire income stream; all of the standard calculations for time
  value of money derive from the most basic algebraic expression for the
  present value of a future sum, ``discounted'' to the present by an
  amount equal to the time value of money. For example, the future value
  sum \({\displaystyle FV}\) to be received in one year is discounted at
  the rate of interest \({\displaystyle r}\) to give the present value
  sum \({\displaystyle PV}\).
\end{itemize}
\end{frame}

\begin{frame}[fragile]{}
\protect\hypertarget{section-15}{}
Suppose a farmer want's to know value of his current fund deposit Rs
10050. The annual market discounting rate (bank interest rate, if he
deposited it in bank) is 8\%. What sum should he expect if he were to
draw the funds after 3 years.

Using formula for compounding of current value to obtain future value
(Equation \ref{eqn:future-value}), we obtain future value of current Rs
10050 = \ensuremath{1.37\times 10^{4}}.

Similarly, in a second example, let us suppose a farmer expects Rs 10000
from sales of his crops and poultry 2 years later, now find the current
value of his future expected earning. (Suppose half yearly discounting
rate is 4\%).

It is calculated from the discounting future value to current value,
using equation \ref{eqn:current-value}: 8548.04. Here we should use
\texttt{n\ =\ 4}, because time time frequency is 2 times for the given
annual compounding rate.
\end{frame}




\end{document}
