\PassOptionsToPackage{unicode=true}{hyperref} % options for packages loaded elsewhere
\PassOptionsToPackage{hyphens}{url}
\documentclass[12pt,ignorenonframetext,aspectratio=169]{beamer}
\IfFileExists{pgfpages.sty}{\usepackage{pgfpages}}{}
\setbeamertemplate{caption}[numbered]
\setbeamertemplate{caption label separator}{: }
\setbeamercolor{caption name}{fg=normal text.fg}
\beamertemplatenavigationsymbolsempty
\usepackage{lmodern}
\usepackage{amssymb}
\usepackage{amsmath}
\usepackage{ifxetex,ifluatex}
\usepackage{fixltx2e} % provides \textsubscript
\ifnum 0\ifxetex 1\fi\ifluatex 1\fi=0 % if pdftex
  \usepackage[T1]{fontenc}
  \usepackage[utf8]{inputenc}
\else % if luatex or xelatex
  \ifxetex
    \usepackage{mathspec}
  \else
    \usepackage{fontspec}
\fi
\defaultfontfeatures{Ligatures=TeX,Scale=MatchLowercase}






%
\fi

  \usetheme[]{iqss}






% use upquote if available, for straight quotes in verbatim environments
\IfFileExists{upquote.sty}{\usepackage{upquote}}{}
% use microtype if available
\IfFileExists{microtype.sty}{%
  \usepackage{microtype}
  \UseMicrotypeSet[protrusion]{basicmath} % disable protrusion for tt fonts
}{}


\newif\ifbibliography


\hypersetup{
      pdftitle={Farm planning},
        pdfauthor={Deependra Dhakal},
          pdfborder={0 0 0},
    breaklinks=true}
%\urlstyle{same}  % Use monospace font for urls







% Prevent slide breaks in the middle of a paragraph:
\widowpenalties 1 10000
\raggedbottom

  \AtBeginPart{
    \let\insertpartnumber\relax
    \let\partname\relax
    \frame{\partpage}
  }
  \AtBeginSection{
    \ifbibliography
    \else
      \let\insertsectionnumber\relax
      \let\sectionname\relax
      \frame{\sectionpage}
    \fi
  }
  \AtBeginSubsection{
    \let\insertsubsectionnumber\relax
    \let\subsectionname\relax
    \frame{\subsectionpage}
  }



\setlength{\parindent}{0pt}
\setlength{\parskip}{6pt plus 2pt minus 1pt}
\setlength{\emergencystretch}{3em}  % prevent overfull lines
\providecommand{\tightlist}{%
  \setlength{\itemsep}{0pt}\setlength{\parskip}{0pt}}

  \setcounter{secnumdepth}{0}


  \usepackage{booktabs}
  \usepackage{longtable}
  \usepackage{emptypage}
  \usepackage{array}
  \usepackage{multirow}
  \usepackage{wrapfig}
  \usepackage{float}
  \usepackage{colortbl}
  \usepackage{pdflscape}
  \usepackage{tabu}
  \usepackage{threeparttable}
  \usepackage{threeparttablex}
  \usepackage[normalem]{ulem}
  \usepackage{rotating}
  \usepackage{makecell}
  \usepackage{xcolor}
  \usepackage{tikz} % required for image opacity change
  \usepackage[absolute,overlay]{textpos} % for text formatting
  \usepackage[utf8]{inputenc}
  \usetikzlibrary{mindmap,arrows,shapes,positioning,shadows,trees}
  \usepackage[skip=2pt]{caption}

  % this font option is amenable for beamer
  \setbeamerfont{caption}{size=\tiny}


%% IQSS overrides
\iqsssectiontitle{Outline}

\AtBeginSection[]{
  \title{\insertsectionhead}
  {
    \definecolor{white}{rgb}{0.776,0.357,0.157}
    \definecolor{iqss@orange}{rgb}{1,1,1}
    \ifnum \insertmainframenumber > \insertframenumber
    \frame{
      \frametitle{\iqsssectiontitleheader}
      \tableofcontents[currentsection]
    }
    \else
    \frame{
      \frametitle{Backup Slides}
      \tableofcontents[sectionstyle=shaded/shaded,subsectionstyle=shaded/shaded/shaded]
    }
    \fi
  }
}

\AtBeginSubsection[]{}

%%


  \title[]{Farm planning}



  \author[
        Deependra Dhakal
    ]{Deependra Dhakal}

  \institute[
    ]{
    GAASC, Baitadi \and Tribhuwan University
    }

\date[
      \today
  ]{
      \today
        }

\begin{document}

% Hide progress bar and footline on titlepage
  \begin{frame}[plain]
  \titlepage
  \end{frame}



\hypertarget{background}{%
\section{Background}\label{background}}

\begin{frame}{Farm plan: Meaning}
\protect\hypertarget{farm-plan-meaning}{}
\begin{itemize}
\tightlist
\item
  A successful farm business is not result of chance factors' -- good
  weather and good prices -- help but a profitable and growing business
  is the product of good planning.
\item
  A farm plan is a program of total farm activity of a farmer drawn up
  in advance.
\item
  Planning is a process of generating ideas, while reviewing past
  performance and using the knowledge gained to make future
  decision/choices among a variety of them for translating it into
  operation/production.
\item
  Farm planning is a scientific planning that is systematic, written and
  based on the best information available and aimed at achieving the
  maximum satisfaction for the farmer and his family out of their
  resources.
\end{itemize}
\end{frame}

\begin{frame}{Importance and use}
\protect\hypertarget{importance-and-use}{}
\begin{itemize}
\tightlist
\item
  After farm planning, budgeting is undertaken. Budgeting is a method of
  analyzing plans for the use of agricultural resources at the command
  of the decision maker.
\item
  Enable farm entrepreneurs to achieve the objectives in relation to his
  farm and family in a more organized manner;
\item
  Enable careful examination of the existing resources, efficient
  allocation and reduces wastes;
\item
  Help input and output sell, calculate estimated farm incomes,
  arrangement of required credit ahead of time;
\item
  Orderly planning helps preventing stress and strains in the farming
  business;
\item
  Planning eases cash income in a regular way;
\item
  In sum, it is money saving device, cheaper to commit mistakes on the
  paper than in the business.
\end{itemize}
\end{frame}

\hypertarget{good-farm-plan}{%
\section{Good farm plan}\label{good-farm-plan}}

\begin{frame}{Principle characteristics}
\protect\hypertarget{principle-characteristics}{}
\begin{itemize}
\tightlist
\item
  Must be written form by outlining minor details;
\item
  First plan and work up on plan;
\item
  Forward looking: prepare by considering environment, family/hire
  labor, soil fertility management etc. so avoid rigidity in farm
  planning;
\item
  Must be resource use efficient;
\item
  Balance combination of enterprises by due consideration in marketing
  arrangement, soil
\item
  fertility, credit, stabilize farm earning, resource mobilization;
\item
  Must reduce risk, stresses and strains;
\item
  Optimize use of farmer's capacity, knowledge, training \& experience;
\item
  Must record inflow and outflow of the fund;
\item
  Plan must provide flexibility.
\end{itemize}
\end{frame}

\hypertarget{techniques}{%
\section{Techniques}\label{techniques}}

\begin{frame}{Steps in preparing farm plan}
\protect\hypertarget{steps-in-preparing-farm-plan}{}
\begin{enumerate}
\tightlist
\item
  Evaluating present farm situation (with respect to Resource inventory,
  crops grown, extent of resource use, level of production, cost and
  returns)
\item
  List out the risks to farm production (incidence of pest and diseases,
  possibility of drought, cyclones, flood etc. are to be borne in mind
  while planning.)
\item
  Identifying the weakness of the existing plan (Like suitable variety,
  type of fertilizer, and plant protection chemicals, their marketing,
  cultural practices etc. minor operational changes may help in
  increasing returns)
\item
  Specification of technical coefficients of production (to identify
  suitable technology, there is need to gather information from various
  sources regarding the improved farming method and practices and the
  various inputs which can be applied under local condition.)
\end{enumerate}
\end{frame}

\begin{frame}{}
\protect\hypertarget{section}{}
\begin{enumerate}
\setcounter{enumi}{4}
\tightlist
\item
  Specification of appropriate prices (for the specified production
  coefficient average prices are to be determined to estimate the
  expected returns.)
\item
  Preparation of enterprise budget (enterprise budget can be prepared
  with the help of extension leaflets, research station reports,
  publications etc. these budgets will give input output relationship of
  each enterprise.)
\item
  Preparation of alternate plan (an alternate plan can be made by
  improving present cropping scheme and by keeping in view the long
  range farm benefits.)
\item
  Analyze the alternate plan to check profitability (extra returns per
  rupee investment in alternate plan is estimated by partial budgeting.)
\end{enumerate}
\end{frame}




\end{document}
