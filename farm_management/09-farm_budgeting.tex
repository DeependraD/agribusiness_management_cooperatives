\PassOptionsToPackage{unicode=true}{hyperref} % options for packages loaded elsewhere
\PassOptionsToPackage{hyphens}{url}
\documentclass[12pt,ignorenonframetext,aspectratio=169]{beamer}
\IfFileExists{pgfpages.sty}{\usepackage{pgfpages}}{}
\setbeamertemplate{caption}[numbered]
\setbeamertemplate{caption label separator}{: }
\setbeamercolor{caption name}{fg=normal text.fg}
\beamertemplatenavigationsymbolsempty
\usepackage{lmodern}
\usepackage{amssymb}
\usepackage{amsmath}
\usepackage{ifxetex,ifluatex}
\usepackage{fixltx2e} % provides \textsubscript
\ifnum 0\ifxetex 1\fi\ifluatex 1\fi=0 % if pdftex
  \usepackage[T1]{fontenc}
  \usepackage[utf8]{inputenc}
\else % if luatex or xelatex
  \ifxetex
    \usepackage{mathspec}
  \else
    \usepackage{fontspec}
\fi
\defaultfontfeatures{Ligatures=TeX,Scale=MatchLowercase}






%
\fi

  \usetheme[]{iqss}






% use upquote if available, for straight quotes in verbatim environments
\IfFileExists{upquote.sty}{\usepackage{upquote}}{}
% use microtype if available
\IfFileExists{microtype.sty}{%
  \usepackage{microtype}
  \UseMicrotypeSet[protrusion]{basicmath} % disable protrusion for tt fonts
}{}


\newif\ifbibliography


\hypersetup{
      pdftitle={Farm budgeting},
        pdfauthor={Deependra Dhakal},
          pdfborder={0 0 0},
    breaklinks=true}
%\urlstyle{same}  % Use monospace font for urls







% Prevent slide breaks in the middle of a paragraph:
\widowpenalties 1 10000
\raggedbottom

  \AtBeginPart{
    \let\insertpartnumber\relax
    \let\partname\relax
    \frame{\partpage}
  }
  \AtBeginSection{
    \ifbibliography
    \else
      \let\insertsectionnumber\relax
      \let\sectionname\relax
      \frame{\sectionpage}
    \fi
  }
  \AtBeginSubsection{
    \let\insertsubsectionnumber\relax
    \let\subsectionname\relax
    \frame{\subsectionpage}
  }



\setlength{\parindent}{0pt}
\setlength{\parskip}{6pt plus 2pt minus 1pt}
\setlength{\emergencystretch}{3em}  % prevent overfull lines
\providecommand{\tightlist}{%
  \setlength{\itemsep}{0pt}\setlength{\parskip}{0pt}}

  \setcounter{secnumdepth}{0}


  \usepackage{booktabs}
  \usepackage{longtable}
  \usepackage{emptypage}
  \usepackage{array}
  \usepackage{multirow}
  \usepackage{wrapfig}
  \usepackage{float}
  \usepackage{colortbl}
  \usepackage{pdflscape}
  \usepackage{tabu}
  \usepackage{threeparttable}
  \usepackage{threeparttablex}
  \usepackage[normalem]{ulem}
  \usepackage{rotating}
  \usepackage{makecell}
  \usepackage{xcolor}
  \usepackage{tikz} % required for image opacity change
  \usepackage[absolute,overlay]{textpos} % for text formatting
  \usepackage[utf8]{inputenc}
  \usetikzlibrary{mindmap,arrows,shapes,positioning,shadows,trees}
  \usepackage[skip=2pt]{caption}

  % this font option is amenable for beamer
  \setbeamerfont{caption}{size=\tiny}


%% IQSS overrides
\iqsssectiontitle{Outline}

\AtBeginSection[]{
  \title{\insertsectionhead}
  {
    \definecolor{white}{rgb}{0.776,0.357,0.157}
    \definecolor{iqss@orange}{rgb}{1,1,1}
    \ifnum \insertmainframenumber > \insertframenumber
    \frame{
      \frametitle{\iqsssectiontitleheader}
      \tableofcontents[currentsection]
    }
    \else
    \frame{
      \frametitle{Backup Slides}
      \tableofcontents[sectionstyle=shaded/shaded,subsectionstyle=shaded/shaded/shaded]
    }
    \fi
  }
}

\AtBeginSubsection[]{}

%%


  \title[]{Farm budgeting}



  \author[
        Deependra Dhakal
    ]{Deependra Dhakal}

  \institute[
    ]{
    GAASC, Baitadi \and Tribhuwan University
    }

\date[
      \today
  ]{
      \today
        }

\begin{document}

% Hide progress bar and footline on titlepage
  \begin{frame}[plain]
  \titlepage
  \end{frame}



\hypertarget{introduction}{%
\section{Introduction}\label{introduction}}

\begin{frame}{Meaning and definition}
\protect\hypertarget{meaning-and-definition}{}
\begin{itemize}
\tightlist
\item
  A farm plan should show the crops to be grown, practices to be
  followed in their production, combination of enterprises, use of
  labor, investment to be made on equipment, building etc.
\item
  The expression of such a farm plan in monetary terms by estimation of
  receipts, expenses and net income, is called the budgeting.
\item
  In other words, farm budgeting is a process of estimating costs,
  returns and net profit of a farm or a particular enterprise.
\item
  Farm budgeting can be used simply to select the most profitable plan
  from number of alternatives.
\end{itemize}
\end{frame}

\begin{frame}{}
\protect\hypertarget{section}{}
\begin{itemize}
\tightlist
\item
  The physical aspects of farm planning when expressed in monetary terms
  called budgeting.
\item
  The expression of farm plan in monetary terms by estimation of
  receipts, expenses and net income is called budgeting.
\item
  Farm budgeting is a process of estimating costs, returns and net
  profit of a farm or a particular enterprise.
\item
  Budget is a statement of estimated income and expenditure.
\end{itemize}
\end{frame}

\hypertarget{enterprise-budgeting}{%
\section{Enterprise budgeting}\label{enterprise-budgeting}}

\begin{frame}{}
\protect\hypertarget{section-1}{}
\begin{itemize}
\tightlist
\item
  An enterprise is defined as a single crop or livestock commodity being
  produced on the farm.
\item
  Enterprises are the basic building blocks for a farm plan.
\item
  Enterprise budgeting calculates input required, cost involved and
  expected returns of the particular enterprises.
\item
  It helps planning particular enterprises based on planning estimated
  inputs, expense plan and obtain expected benefit.
\item
  Enterprise budgeting can be developed for each actual and potential
  enterprise in a farm plan such as paddy enterprise, wheat enterprise
  or a cow enterprise.
\item
  What you produce determines the profitability of the business.
\item
  By analyzing revenues and expenses associated with individual
  enterprises you can determine which enterprises might be expanded and
  those that should be cut back or eliminated.
\end{itemize}
\end{frame}

\begin{frame}{}
\protect\hypertarget{section-2}{}
\begin{itemize}
\tightlist
\item
  A manager may also want to compare profitability of one production
  technique with another technique (e.g.~minimum till and conventional
  tillage practices).
\item
  A budget can be developed for each existing or potential enterprise in
  a farm plan.
\item
  Several budgets could be developed for a single budget to represent
  alternative combinations of inputs and outputs.
\item
  Each budget should be developed on the basis of a small common unit
  such as one hectare of paddy, wheat, or maize etc. or one head of
  livestock.
\item
  This permits comparison of the profit for alternative and competing
  enterprises.
\end{itemize}
\end{frame}

\hypertarget{partial-budgeting}{%
\section{Partial budgeting}\label{partial-budgeting}}

\begin{frame}{Meaning}
\protect\hypertarget{meaning}{}
\begin{itemize}
\tightlist
\item
  Refers to estimating costs and returns and net income of a particular
  enterprise.
\item
  It refers to estimating the returns for a part of the business
  i.e.~one or few activities; for example, to estimate additional cost
  and returns from growing one hectare of rice in place of vegetable
  crop or to estimate additional cost and return by adopting foliar
  application of chemical fertilizers instead of soil application.
\item
  It consists of four important elements: added costs, added returns,
  reduced returns and reduced costs.
\item
  Partial budgeting technique is generally used to evaluate the
  profitability of input substitution, enterprise substitution and scale
  of operation.
\end{itemize}
\end{frame}

\begin{frame}{Elements}
\protect\hypertarget{elements}{}
\begin{enumerate}
\tightlist
\item
  Added costs: Additional costs are incurred, if the proposed
  modification is the introduction of a new enterprise or increase in
  the size of the existing enterprise.
\item
  Added returns: Additional returns could be received when the proposed
  modification is the addition of a new enterprise, or increase in the
  size of the existing enterprise or adoption of technology that results
  in higher productivity.
\item
  Reduced returns: Decrease in the returns is observed when the proposed
  modification involves the elimination of an existing enterprise or
  reduction in the size of the existing enterprise.
\item
  Reduced costs: Decrease in the costs is found when the proposed
  modification involves the elimination of existing enterprise or
  reduction in the size of the enterprise or adoption of a technology
  that uses fewer amounts of resources.
\end{enumerate}
\end{frame}

\begin{frame}{Book keeping with partial budget}
\protect\hypertarget{book-keeping-with-partial-budget}{}
\begin{table}[H]
\centering\begingroup\fontsize{6}{8}\selectfont

\begin{tabular}{>{\raggedright\arraybackslash}p{12em}>{\raggedright\arraybackslash}p{3em}>{\raggedright\arraybackslash}p{12em}>{\raggedright\arraybackslash}p{3em}}
\toprule
Debit & Amount & Credit & Amount\\
\midrule
\rowcolor{gray!6}  (a) increase in costs per acre &  & (a) decrease in costs per acre & \\
1. Depreciation for 2 hours; $\frac{2200 \times 2}{2500} = \frac{44}{25}$ & 1.76 & Manual labor; 60-2 = 58 man hours (\@ Rs. 0.75 per man hour) & 44\\
\rowcolor{gray!6}  2. Interest \@ 10\% & 0.18 &  & \\
3. Tractor (32 HP) cost for 2 hours \@ Rs 6.56/hr & 13.12 &  & \\
\rowcolor{gray!6}  (b) decrease in returns per acre &  & (b) increase in returns per acre & \\
\addlinespace
Decrease in yield \@ 2.5\%; i.e. 0.15 qtl \@ Rs. 90/qtl & 13.50 &  & \\
\rowcolor{gray!6}  A. Total increased cost and reduced returns Rs. & 28.56 & B. Total reduced costs and increased returns Rs. & 44\\
\bottomrule
\end{tabular}
\endgroup{}
\end{table}
\end{frame}

\hypertarget{complete-budgeting}{%
\section{Complete budgeting}\label{complete-budgeting}}

\begin{frame}{}
\protect\hypertarget{section-3}{}
\begin{itemize}
\tightlist
\item
  It is also called as total budgeting.
\item
  Refers to preparing budget for the farm as a whole.
\item
  Complete budgeting considers all the crops, livestock, methods of
  production and aspects of marketing in consolidated form and estimates
  costs and returns for the farm as a whole.
\item
  Therefore complete budgeting can he specifically defined as ``An
  estimation of the probable income and expenditure is made for the farm
  as a single unit of course, a complete budget is required when a farm
  plan is prepared for new farm or when drastic changes are suggested in
  the plan of the existing pattern on an established farm''.
\item
  Complete budgeting can be prepared for short run (annual budget) and
  for long run.
\end{itemize}
\end{frame}

\hypertarget{comparison-between-budgeting-types}{%
\section{Comparison between budgeting
types}\label{comparison-between-budgeting-types}}

\begin{frame}{}
\protect\hypertarget{section-4}{}
The differences between partial and complete budgeting is presented in
Table \ref{tab:farm-budgeting}.

\begin{table}

\caption{\label{tab:farm-budgeting}Difference between complete and partial budgeting}
\centering
\fontsize{6}{8}\selectfont
\begin{tabular}[t]{>{\raggedleft\arraybackslash}p{4em}>{\raggedleft\arraybackslash}p{16em}>{\raggedleft\arraybackslash}p{4em}>{\raggedleft\arraybackslash}p{16em}}
\toprule
\textbf{} & \textbf{Complete budgeting} & \textbf{} & \textbf{Partial budgeting}\\
\midrule
\rowcolor{gray!6}  1 & The whole farm is considered as one unit & 1 & It is adopted when a minor aspect of farm organization is touched.\\
2 & All the aspects like crops, livestock, machinery and other assets are considered & 2 & It is practiced with in the existing resources structure of the farm.\\
\rowcolor{gray!6}  3 & Both fixed and variable costs are calculated for working out costs and returns. & 3 & Only variable costs are considered.\\
4 & Net income is estimated by deleting fixed costs and costs of variable inputs from the value of the product & 4 & Net income is estimated by deleting only cost of variable inputs from the value of the product.\\
\rowcolor{gray!6}  5 & It requires more efforts and time for preparation. & 5 & It requires relatively less efforts and time for preparation.\\
\bottomrule
\end{tabular}
\end{table}
\end{frame}

\hypertarget{steps-in-farm-budgeting}{%
\section{Steps in farm budgeting}\label{steps-in-farm-budgeting}}

\begin{frame}{}
\protect\hypertarget{section-5}{}
\begin{itemize}
\item
  Step 1: is to estimate total production (output or yield) and expected
  output price. The estimated yields and prices should be what you
  expect under normal conditions. Be as realistic as possible.
\item
  Step 2: is to estimate variable costs. Variable costs are just what
  they sound like they vary with the amount of product you produce.
  These are the out-of- pocket costs that must be incurred if the
  enterprise is produced or grown. Some examples of variable costs are:
  hired labor; repairs; feed; supplies; vet. medicine; fuel; seed; etc.
\end{itemize}
\end{frame}

\begin{frame}{}
\protect\hypertarget{section-6}{}
\begin{itemize}
\tightlist
\item
  Step 3: is to assess fixed costs. Fixed costs will occur and will stay
  about the same no matter how much you produce, or, in most cases
  whether or not you produce at all. Some examples of fixed costs are:
  depreciation; taxes; insurance; etc. Land charges are generally based
  on one of three acceptable methods:

  \begin{enumerate}
  \tightlist
  \item
    interest opportunity based on current value of land;
  \item
    owner rental income; or
  \item
    typical cash rent charge (market rent).
  \end{enumerate}
\item
  Step 4: Calculating net receipts. Net receipts represent that income
  which is left for the farmer and family to live on, pay debt, invest,
  or save.
\end{itemize}
\end{frame}

\hypertarget{enterprise-budgeting-an-example}{%
\section{Enterprise budgeting: An
example}\label{enterprise-budgeting-an-example}}

\begin{frame}{Budgeting of a Mango farm}
\protect\hypertarget{budgeting-of-a-mango-farm}{}
\begin{columns}

\begin{column}{0.4\textwidth}
\tiny
% \scalebox{0.3}{\begin{minipage}{1.20\textwidth}
\begin{itemize}
\item Area: 1 ha
\item Place: Bharatpur, Chitwan
\item Duration: 1 year (2060-2061 BS)
\end{itemize}
% \end{minipage}}
\end{column}

\begin{column}{0.8\textwidth}

\scalebox{0.60}{\begin{minipage}{1.2\textwidth}

\vspace{0.25cm}

\begin{table}[H]
\centering\begingroup\fontsize{6}{8}\selectfont

\begin{tabular}{llrrr}
\toprule
Stream & Particulars & Quantity & Rate & Amount\\
\midrule
 & Human labor & 50 & 400 & 20000\\

 & Tractor & 3 & 1600 & 4800\\

 & Pump set water pond & 15 & 100 & 1500\\

 & Layout & 4 & 1000 & 4000\\

 & Sapling & 110 & 60 & 6600\\

 & Manure & 2200 & 2 & 4400\\

 & Fertilizer (Urea) & 25 & 25 & 625\\

 & Fertilizer (DAP) & 15 & 50 & 750\\

 & Fertilizer (Potash) & 10 & 36 & 360\\

 & Plant protection chemicals (Bordeaux mixture and micronutrient) &  &  & 250\\

 & Plant protection chemicals (Training and pruning) &  &  & 300\\

 & Equipments & 4 & 800 & 3200\\

 & Land lease value &  &  & 25000\\

 & Management cost & 12 & 1500 & 18000\\

 & Others &  &  & 2500\\

 & Interest on variable cost &  &  & 11548\\

\multirow{-17}{*}{\raggedright\arraybackslash Variable} & Total &  &  & 103833\\
\cmidrule{1-5}
 & Land cost &  &  & 300\\

 & Water cost &  &  & 500\\

 & Depreciation of farm equipments &  &  & 500\\

 & Repair and maintenance of farm equipments &  &  & 500\\

\multirow{-5}{*}{\raggedright\arraybackslash Fixed} & Total &  &  & 1800\\
\cmidrule{1-5}
 & Production & 2000 & 52 & 105000\\

\multirow{-2}{*}{\raggedright\arraybackslash Production} & Total &  &  & 105000\\
\cmidrule{1-5}
Net profit &  &  &  & 1266\\
\bottomrule
\end{tabular}
\endgroup{}
\end{table}
\end{minipage}}

\end{column}

\end{columns}
\end{frame}




\end{document}
