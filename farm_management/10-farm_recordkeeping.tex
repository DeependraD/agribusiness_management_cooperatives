\PassOptionsToPackage{unicode=true}{hyperref} % options for packages loaded elsewhere
\PassOptionsToPackage{hyphens}{url}
\documentclass[12pt,ignorenonframetext,aspectratio=169]{beamer}
\IfFileExists{pgfpages.sty}{\usepackage{pgfpages}}{}
\setbeamertemplate{caption}[numbered]
\setbeamertemplate{caption label separator}{: }
\setbeamercolor{caption name}{fg=normal text.fg}
\beamertemplatenavigationsymbolsempty
\usepackage{lmodern}
\usepackage{amssymb}
\usepackage{amsmath}
\usepackage{ifxetex,ifluatex}
\usepackage{fixltx2e} % provides \textsubscript
\ifnum 0\ifxetex 1\fi\ifluatex 1\fi=0 % if pdftex
  \usepackage[T1]{fontenc}
  \usepackage[utf8]{inputenc}
\else % if luatex or xelatex
  \ifxetex
    \usepackage{mathspec}
  \else
    \usepackage{fontspec}
\fi
\defaultfontfeatures{Ligatures=TeX,Scale=MatchLowercase}






%
\fi

  \usetheme[]{iqss}






% use upquote if available, for straight quotes in verbatim environments
\IfFileExists{upquote.sty}{\usepackage{upquote}}{}
% use microtype if available
\IfFileExists{microtype.sty}{%
  \usepackage{microtype}
  \UseMicrotypeSet[protrusion]{basicmath} % disable protrusion for tt fonts
}{}


\newif\ifbibliography


\hypersetup{
      pdftitle={Farm recordkeeping},
        pdfauthor={Deependra Dhakal},
          pdfborder={0 0 0},
    breaklinks=true}
%\urlstyle{same}  % Use monospace font for urls







% Prevent slide breaks in the middle of a paragraph:
\widowpenalties 1 10000
\raggedbottom

  \AtBeginPart{
    \let\insertpartnumber\relax
    \let\partname\relax
    \frame{\partpage}
  }
  \AtBeginSection{
    \ifbibliography
    \else
      \let\insertsectionnumber\relax
      \let\sectionname\relax
      \frame{\sectionpage}
    \fi
  }
  \AtBeginSubsection{
    \let\insertsubsectionnumber\relax
    \let\subsectionname\relax
    \frame{\subsectionpage}
  }



\setlength{\parindent}{0pt}
\setlength{\parskip}{6pt plus 2pt minus 1pt}
\setlength{\emergencystretch}{3em}  % prevent overfull lines
\providecommand{\tightlist}{%
  \setlength{\itemsep}{0pt}\setlength{\parskip}{0pt}}

  \setcounter{secnumdepth}{0}


  \usepackage{booktabs}
  \usepackage{longtable}
  \usepackage{emptypage}
  \usepackage{array}
  \usepackage{multirow}
  \usepackage{wrapfig}
  \usepackage{float}
  \usepackage{colortbl}
  \usepackage{pdflscape}
  \usepackage{tabu}
  \usepackage{threeparttable}
  \usepackage{threeparttablex}
  \usepackage[normalem]{ulem}
  \usepackage{rotating}
  \usepackage{makecell}
  \usepackage{xcolor}
  \usepackage{tikz} % required for image opacity change
  \usepackage[absolute,overlay]{textpos} % for text formatting
  \usepackage[utf8]{inputenc}
  \usetikzlibrary{mindmap,arrows,shapes,positioning,shadows,trees}
  \usepackage[skip=2pt]{caption}

  % this font option is amenable for beamer
  \setbeamerfont{caption}{size=\tiny}


%% IQSS overrides
\iqsssectiontitle{Outline}

\AtBeginSection[]{
  \title{\insertsectionhead}
  {
    \definecolor{white}{rgb}{0.776,0.357,0.157}
    \definecolor{iqss@orange}{rgb}{1,1,1}
    \ifnum \insertmainframenumber > \insertframenumber
    \frame{
      \frametitle{\iqsssectiontitleheader}
      \tableofcontents[currentsection]
    }
    \else
    \frame{
      \frametitle{Backup Slides}
      \tableofcontents[sectionstyle=shaded/shaded,subsectionstyle=shaded/shaded/shaded]
    }
    \fi
  }
}

\AtBeginSubsection[]{}

%%


  \title[]{Farm recordkeeping}



  \author[
        Deependra Dhakal
    ]{Deependra Dhakal}

  \institute[
    ]{
    GAASC, Baitadi \and Tribhuwan University
    }

\date[
      \today
  ]{
      \today
        }

\begin{document}

% Hide progress bar and footline on titlepage
  \begin{frame}[plain]
  \titlepage
  \end{frame}



\hypertarget{meaning-and-definition}{%
\section{Meaning and definition}\label{meaning-and-definition}}

\begin{frame}{}
\protect\hypertarget{section}{}
\begin{itemize}
\tightlist
\item
  Farm accountancy is defined as the art as well as the science of
  recording in books business transactions in regular and systematic
  manner so that their nature, extent and financial effects can be
  readily ascertained at any time of the year.
\item
  Farm accounting is an application of the accounting principles to the
  business of farming.
\item
  Farm book keeping is known as a system of records written to furnish a
  history of the business transactions, with special reference to its
  financial side.
\end{itemize}
\end{frame}

\begin{frame}{}
\protect\hypertarget{section-1}{}
\begin{itemize}
\tightlist
\item
  A record keeping system should go beyond the basic listing of income
  and expenses.
\item
  The major financial accounting statements aim to provide a picture of
  the financial position and performance of a business.
\item
  A business's accounting system will normally produce three particular
  statements on a regular, recurring basis.
\end{itemize}
\end{frame}

\hypertarget{importance}{%
\section{Importance}\label{importance}}

\begin{frame}{}
\protect\hypertarget{section-2}{}
\begin{enumerate}
\tightlist
\item
  Measure profit and assess financial condition.
\item
  Provide data for business analysis.
\item
  Assist in preparing reports for partners, lenders, landlords, input
  providers, and government agencies.
\item
  Measure the profitability of individual enterprises.
\item
  Assist in the analysis of new investments.
\item
  Prepare income tax returns.
\end{enumerate}
\end{frame}

\hypertarget{farm-records-examples}{%
\section{Farm records -- Examples}\label{farm-records-examples}}

\begin{frame}{Dairy farm}
\protect\hypertarget{dairy-farm}{}
\begin{itemize}
\tightlist
\item
  Livestock register : This register records the number of the animals
  at the farm along with their identification number, date of birth,
  sire number, dam number, calf and its sex, date of calving, date of
  purchase, date of sale/auction/death.
\item
  Calving register : This register maintains the records of calving that
  take place in the farm. It maintains dam and sire number of the calf,
  calf number, sex and its date of birth and any other remarks like type
  of calving (normal/abnormal).
\item
  Daily milk yield register : This register records the daily milk yield
  performance of the cows.
\item
  Calf register : maintains the records of calf at the farm, calf
  number, sex of the calf, sire number, dam number, birth weight etc.
\item
  Growth record of young stock : this record maintains the weight of the
  young stocks at different intervals.
\item
  Daily feeding register : This register records the amount concentrate,
  dry fodder, green fodder and other feeds given to the animals daily.
\item
  Herd health register : This register maintains the record of the
  diseased animals along with history, symptoms, diagnosed disease,
  treatment given and name of the veterinarian who treated.
\item
  Cattle breeding register : This register maintains the details of
  breeding practices in the farm such as cow number, date of calving,
  date of heat and services along with the bull number, date of
  successful service, pregnancy diagnosis records, expected date of
  calving, actual date of calving, calf number etc.
\item
  Animal History sheet : This maintains animal number, breed, date of
  birth, sire and dam number, lactation yield records, date of drying,
  date of disposal/death, cause of disposal etc.
\end{itemize}
\end{frame}

\begin{frame}{Crop production farm}
\protect\hypertarget{crop-production-farm}{}
\begin{itemize}
\tightlist
\item
  Field locations book
\item
  Crop rotation and input history by field
\item
  Individual field activity log
\item
  Storage records
\item
  Sales record
\item
  Estimated and actual individual crop income and expense
\item
  Income worksheet
\item
  Expense worksheet
\end{itemize}
\end{frame}

\hypertarget{farm-inventory}{%
\section{Farm inventory}\label{farm-inventory}}

\begin{frame}{Making an inventory}
\protect\hypertarget{making-an-inventory}{}
\begin{itemize}
\tightlist
\item
  An inventory is generally done at least once a year and involves the
  counting of everything on the farm, including hectares of land, tons
  of grain in stock, animals and so on. The inventory will also include
  the money value in Rand of the assets.
\item
  Walk around the farm and make a general inspection of farm assets.
  Make a list or spreadsheet of items according to main categories such
  as buildings, land, machinery, equipment, farm supplies and stock.
\end{itemize}
\end{frame}

\begin{frame}{Details on a farm inventory}
\protect\hypertarget{details-on-a-farm-inventory}{}
\begin{itemize}
\tightlist
\item
  Size of the farm, the right of ownership, the size of land that is
  used for agriculture and a valuation of the production unit.
\item
  Description and valuation of fixed improvements such as a new borehole
  or roof.
\item
  Vehicles, machinery and equipment, as well as numbers and types or
  models.
\item
  Numbers and types of livestock.
\item
  Stocks, farm supplies and production inputs such as seed, fertiliser.
\item
  Finished and semi-finished products at the beginning and end of the
  financial year.
\item
  In separate columns of the inventory indicate the following:

  \begin{itemize}
  \tightlist
  \item
    The year when each item of property was purchased.
  \item
    The purchase price of each item.
  \item
    The expected lifetime of machinery/equipment.
  \item
    Then calculate the annual depreciation for each item using the
    straight-line method or the reducing-value method.
  \end{itemize}
\end{itemize}
\end{frame}

\begin{frame}{Expected lifetime of equipment}
\protect\hypertarget{expected-lifetime-of-equipment}{}
\begin{itemize}
\item
  Below are the expected lifetimes for some machinery and equipment used
  on a farm. In brackets is the salvage value as a percentage of the
  purchase price - the value of the item at the end of its life.
\item
  Vehicles - 5 years (10\%)
\item
  Tractors - 10 years (10\%)
\item
  Implements - 10 years (10\%)
\item
  Tools - 10 years (10\%)
\item
  Pumps and electrical motors - 15 years (0\%)
\end{itemize}
\end{frame}




\end{document}
