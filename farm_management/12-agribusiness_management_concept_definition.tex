\PassOptionsToPackage{unicode=true}{hyperref} % options for packages loaded elsewhere
\PassOptionsToPackage{hyphens}{url}
\documentclass[12pt,ignorenonframetext,aspectratio=169]{beamer}
\IfFileExists{pgfpages.sty}{\usepackage{pgfpages}}{}
\setbeamertemplate{caption}[numbered]
\setbeamertemplate{caption label separator}{: }
\setbeamercolor{caption name}{fg=normal text.fg}
\beamertemplatenavigationsymbolsempty
\usepackage{lmodern}
\usepackage{amssymb}
\usepackage{amsmath}
\usepackage{ifxetex,ifluatex}
\usepackage{fixltx2e} % provides \textsubscript
\ifnum 0\ifxetex 1\fi\ifluatex 1\fi=0 % if pdftex
  \usepackage[T1]{fontenc}
  \usepackage[utf8]{inputenc}
\else % if luatex or xelatex
  \ifxetex
    \usepackage{mathspec}
  \else
    \usepackage{fontspec}
\fi
\defaultfontfeatures{Ligatures=TeX,Scale=MatchLowercase}






%
\fi

  \usetheme[]{iqss}






% use upquote if available, for straight quotes in verbatim environments
\IfFileExists{upquote.sty}{\usepackage{upquote}}{}
% use microtype if available
\IfFileExists{microtype.sty}{%
  \usepackage{microtype}
  \UseMicrotypeSet[protrusion]{basicmath} % disable protrusion for tt fonts
}{}


\newif\ifbibliography


\hypersetup{
      pdftitle={Agribusiness management: Concept definition and scope},
        pdfauthor={Deependra Dhakal},
          pdfborder={0 0 0},
    breaklinks=true}
%\urlstyle{same}  % Use monospace font for urls







% Prevent slide breaks in the middle of a paragraph:
\widowpenalties 1 10000
\raggedbottom

  \AtBeginPart{
    \let\insertpartnumber\relax
    \let\partname\relax
    \frame{\partpage}
  }
  \AtBeginSection{
    \ifbibliography
    \else
      \let\insertsectionnumber\relax
      \let\sectionname\relax
      \frame{\sectionpage}
    \fi
  }
  \AtBeginSubsection{
    \let\insertsubsectionnumber\relax
    \let\subsectionname\relax
    \frame{\subsectionpage}
  }



\setlength{\parindent}{0pt}
\setlength{\parskip}{6pt plus 2pt minus 1pt}
\setlength{\emergencystretch}{3em}  % prevent overfull lines
\providecommand{\tightlist}{%
  \setlength{\itemsep}{0pt}\setlength{\parskip}{0pt}}

  \setcounter{secnumdepth}{0}


  \usepackage{booktabs}
  \usepackage{longtable}
  \usepackage{emptypage}
  \usepackage{array}
  \usepackage{multirow}
  \usepackage{wrapfig}
  \usepackage{float}
  \usepackage{colortbl}
  \usepackage{pdflscape}
  \usepackage{tabu}
  \usepackage{threeparttable}
  \usepackage{threeparttablex}
  \usepackage[normalem]{ulem}
  \usepackage{rotating}
  \usepackage{makecell}
  \usepackage{xcolor}
  \usepackage{tikz} % required for image opacity change
  \usepackage[absolute,overlay]{textpos} % for text formatting
  \usepackage[utf8]{inputenc}
  \usetikzlibrary{mindmap,arrows,shapes,positioning,shadows,trees}
  \usepackage[skip=2pt]{caption}

  % this font option is amenable for beamer
  \setbeamerfont{caption}{size=\tiny}


%% IQSS overrides
\iqsssectiontitle{Outline}

\AtBeginSection[]{
  \title{\insertsectionhead}
  {
    \definecolor{white}{rgb}{0.776,0.357,0.157}
    \definecolor{iqss@orange}{rgb}{1,1,1}
    \ifnum \insertmainframenumber > \insertframenumber
    \frame{
      \frametitle{\iqsssectiontitleheader}
      \tableofcontents[currentsection]
    }
    \else
    \frame{
      \frametitle{Backup Slides}
      \tableofcontents[sectionstyle=shaded/shaded,subsectionstyle=shaded/shaded/shaded]
    }
    \fi
  }
}

\AtBeginSubsection[]{}

%%


  \title[]{Agribusiness management: Concept definition and scope}



  \author[
        Deependra Dhakal
    ]{Deependra Dhakal}

  \institute[
    ]{
    GAASC, Baitadi \and Tribhuwan University
    }

\date[
      \today
  ]{
      \today
        }

\begin{document}

% Hide progress bar and footline on titlepage
  \begin{frame}[plain]
  \titlepage
  \end{frame}



\begin{frame}{Agriculture}
\protect\hypertarget{agriculture}{}
\footnotesize

\begin{itemize}
\tightlist
\item
  Agriculture is the art and science of crop and livestock production.
  In its broadest sense, agriculture comprises the entire range of
  technologies associated with the production of useful products from
  plants and animals, including soil cultivation, crop and livestock
  management, and the activities of processing and marketing.
  Agriculture is the science, skill or practice of cultivating crop
  plants for human food and animal feed and livestock products.
\item
  Agriculture is the process of producing food, feed, fiber and other
  desired products by cultivation of certain plants and the raising of
  domesticated animals.
\item
  Agriculture consists of growing plants and rearing animals for food,
  manure, decoration and sale. It includes farming of cereal crops,
  grain legumes, oil seed crops root and tuber crops, sugar crops, fiber
  crops, fruits, vegetables, spices, agro-forestry, fishery,
  floriculture, sericulture, apiculture, livestock, poultry, medicinal
  herbs.
\end{itemize}
\end{frame}

\begin{frame}{Business}
\protect\hypertarget{business}{}
\footnotesize

\begin{itemize}
\tightlist
\item
  Business involves activities connected with the production of wealth.
  It is an organized and systematized human activity involving and
  purchase of goods and service with the object of selling them at a
  profit. Business concerns with buying and selling goods, manufacturing
  goods or providing services in order to earn profit.
\item
  Commercial activity in which goods and services are exchanged for one
  another or money, on the basis of their perceived worth.
\item
  Activity of manufacturing, buying, selling, or supplying goods and
  services for money, commerce and trade.
\item
  A business is typically formed to earn profit that will increase the
  wealth of its owners and grow the business itself.
\item
  It is the process of buying and selling of commodities with or without
  value addition and generating profit from it.
\end{itemize}
\end{frame}

\begin{frame}{Management}
\protect\hypertarget{management}{}
\footnotesize

\begin{itemize}
\tightlist
\item
  Management can be defined as the process of achieving organizational
  goals through planning, organizing, leading, and controlling the
  human, physical, financial, and information resources of the
  organization in an effective and efficient manner.
\item
  Management has been described as a social process involving
  responsibility for economical and effective planning and regulation of
  operation of an enterprise in the fulfillment of given purposes.
\item
  Management is the judicious use of knowledge and skill of
  person/persons utilizing the various resources in appropriate time,
  place and methods for the better results.
\item
  Henry Fayol who is considered as the father of principles of
  management: To manage is to forecast, to plan, to organize, to command
  coordinate and to control.
\end{itemize}
\end{frame}

\begin{frame}{}
\protect\hypertarget{section}{}
\begin{itemize}
\tightlist
\item
  George Terry: Management is a distinct process consisting of planning,
  organizing actuating and controlling performance to determine and
  accomplish the objectives by the use of people and resources.
\item
  Management helps coordinate both the human and material resources
  towards objective accomplishment through collective effect.
\item
  Elements of management:

  \begin{enumerate}
  \tightlist
  \item
    Towards objective
  \item
    Through people
  \item
    Via technique
  \item
    In an organization
  \end{enumerate}
\end{itemize}
\end{frame}

\begin{frame}{Agribusiness management}
\protect\hypertarget{agribusiness-management}{}
\begin{itemize}
\tightlist
\item
  The term ``Agribusiness'' was first introduced by John Davis \& Ray
  Goldberg of Harvard University in their book `A Conception of
  Agribusiness' in 1957.
\item
  Process wise, they defined agribusiness as: ``The sum total of
  operation involved in the manufacture and distribution of farm
  supplies, production activities on the farms and the storage,
  processing and distribution of farm commodities and items made from
  them''. Thus, it represents three part system made up of:

  \begin{enumerate}
  \tightlist
  \item
    the agriculture input sector,
  \item
    the production sector, and
  \item
    the processing-manufacturing sector.
  \end{enumerate}
\item
  Agribusiness includes total input-farm-product sectors that supply
  from inputs, are involved in production, and finally, handle the
  processing, distributing, wholesaling and retailing of the product to
  the final consumer. -- Downey and Trocke, 1987.
\end{itemize}
\end{frame}

\begin{frame}{Features of agribusiness}
\protect\hypertarget{features-of-agribusiness}{}
\begin{itemize}
\tightlist
\item
  Since agribusiness is applied economics, problems of agribusiness have
  their roots as well as solution in economics. Problems in decision
  making are:

  \begin{enumerate}
  \tightlist
  \item
    Operational or internal problem; how/what/how much to produce, how
    to promote sales, how to face price competition, etc.
  \item
    Environment or external problem
  \end{enumerate}
\item
  Agribusiness has a vertical structure, compose of input suppliers,
  farmers, processors, transport operators, financers, wholesalers,
  retailers, and consumers. These components participate in the movement
  of the commodity from the producer down to final consumer.
\item
  The agribusiness concept is market-oriented (demand-driven approach).
  This means that the component must function in a way that will lead to
  satisfaction of consumers' need.
\end{itemize}
\end{frame}

\begin{frame}{Scope}
\protect\hypertarget{scope}{}
\begin{itemize}
\tightlist
\item
  The ``whole agricultural sector'' (including fishery and forestry)
\item
  The portion of the ``industrial sector'' which is composed of
  manufacturers or suppliers of inputs (i.e.~for the farm, processing
  plants, and marketing firms) and processors of products, therefore it
  is directly related to industry, commerce and trade\footnote<.->{Industry
    is concerned with the production of commodities and materials while
    commerce and trade are concerned with their distribution}.
\item
  Agriculture is the back bone of Nepalese economy. Hence,
  agriculture-based activities are the basis of developing economy of
  Nepal. Agro-business-led growth has good potential to contribute in
  sustained economic development of country.
\item
  Recent trends in globalization and integration of international
  consumer market offer further opportunities for development of
  agro-business and food industry across the world which would also
  benefit developing country like Nepal.
\end{itemize}
\end{frame}

\begin{frame}{}
\protect\hypertarget{section-1}{}
\begin{itemize}
\tightlist
\item
  Contribution of this sector to GDP for the fiscal year 2009/10 is
  estimated to be about 33 percent and about 66 percent of total
  population is engaged in agriculture. Since it is a major source of
  income and employment for the majority of people in rural area,
  agriculture sector received top priority since the early periodic
  plans and policies of the nation.
\item
  The importance of agro-business can hardly be exaggerated especially
  in coping with globally increasing problem of food crisis. It is the
  only means for the overall food security of the country.
\item
  Nepal is endowed with varied ago-climate, which facilitates production
  of temperate, sub-tropical and tropical agricultural commodities.
\item
  Nepal is rich in natural resources and bio-diversities. Nepal with
  0.03 \% of the world's area having more than 2 \% bio-diversities.
\end{itemize}
\end{frame}

\begin{frame}{}
\protect\hypertarget{section-2}{}
\begin{itemize}
\tightlist
\item
  Nepal has comparative and competitive advantages of growing and
  exports of niche products like orthodox tea, coffee, cardamom, Zinger,
  honey, cheese, green vegetables, vegetable seed, labor intensive and
  manually prepared agricultural products.
\item
  There is growing demand for agricultural inputs like feed and fodder,
  inorganic fertilizers, bio-fertilizers.
\item
  Export of agricultural products can be the good source of economic
  growth. Emerging global and regional free trade regimes such as World
  Trade Organization(WTO), South Asian Free Trade Area (SAFTA), Bay of
  Bengal Initiatives for Multi-sectoral Technical and Economic
  Cooperation(BIMSTEC), Asia Pacific Trading Area (APTA), Nepal has vast
  potential to improve it present position in the World trade of
  agricultural commodities both raw and processed form.
\item
  At present processing is done at primary level only and the rising
  standard of living expands opportunities for secondary and tertiary
  processing of agricultural commodities.
\end{itemize}
\end{frame}

\begin{frame}{}
\protect\hypertarget{section-3}{}
\begin{itemize}
\tightlist
\item
  The livestock wealth gives enormous scope for production of meat, milk
  and milk products, poultry products etc.
\item
  Organic farming has highest potential in Nepal as the pesticide and
  inorganic fertilizer application are less in Nepal compared to
  industrial nations of the world. The farmers can be encouraged and
  educated to switch over for organic farming.
\item
  The enhanced agricultural production throws open opportunities for
  employment in marketing, transport, cold storage and warehousing
  facilities, credit, insurance and logistic support services.
\item
  Emerging new national consensus in social and political economy has
  broadened the scope of agro-business.
\end{itemize}
\end{frame}

\begin{frame}{Agribusiness elements}
\protect\hypertarget{agribusiness-elements}{}
\begin{figure}
\includegraphics[width=0.7\linewidth]{12-agribusiness_management_concept_definition_files/figure-beamer/agribusiness-elements-1} \caption{Agribusiness elements in flow chart}\label{fig:agribusiness-elements}
\end{figure}
\end{frame}

\begin{frame}{Importance of agribusiness}
\protect\hypertarget{importance-of-agribusiness}{}
\begin{itemize}
\tightlist
\item
  Utilization of niche based potentiality according to comparative
  advantage of agricultural goods and services.
\item
  Pro-poor growth strategy in rural farming areas
\item
  Value added and upgrading of agricultural commodity
\item
  Market-oriented (demand-driven business approach)
\end{itemize}
\end{frame}

\begin{frame}{Distinctive features of agribusiness}
\protect\hypertarget{distinctive-features-of-agribusiness}{}
\begin{enumerate}
\tightlist
\item
  Tremendous variety in kind of business the agricultural sector has
  given rise to.
\item
  Agricultural products have relatively inelastic demand than the
  industrial products.
\item
  Infinite variety in size of business
\item
  Many agribusiness workers exhibit a traditional philosophy of life,
  which tend to make agribusiness more conservative than other business.
\item
  Agricultural products are seasonal in nature/low storage life.
\item
  Agribusiness deals with vagaries of nature
\item
  Agribusiness firm tends to be family oriented/community oriented.
\item
  Several government polices have important impression in agricultural
  production and business -- price policy, health policy, fiscal policy,
  monetary policy, etc.
\end{enumerate}
\end{frame}

\begin{frame}{Problems of agribusiness}
\protect\hypertarget{problems-of-agribusiness}{}
\begin{itemize}
\tightlist
\item
  Subsistence farming with low productivity
\item
  Low investment in agriculture related activities.
\item
  Lack of linkage between agriculture and industry.
\item
  Lack of location specific and farmer oriented technology.
\item
  Lack of integrated approach.
\item
  Heavy dependent on rain fed condition.
\item
  Small, fragmented and absentee land holding with high cost of
  production.
\item
  Inadequate market infrastructure facilities.
\item
  Scattered production hence high collection cost.
\item
  Low infrastructure development for agro-business.
\end{itemize}
\end{frame}

\begin{frame}{}
\protect\hypertarget{section-4}{}
\begin{itemize}
\tightlist
\item
  Lack of skilled and qualified human resources.
\item
  Lack of business culture, monitoring and evaluation and reward
  punishment system.
\item
  Unconsolidated legal framework- overlapping and cross cutting.
\item
  Unskilled, aged backward community is engaged in farming.
\item
  Labor migration.
\item
  Insufficient government service delivery system.
\item
  Insufficient and timely unavailability of agricultural inputs.
\item
  Low volume of commercial production for market.
\item
  Weak capacity of planning and policy analysis.
\item
  Low linkage between research and extension.
\end{itemize}
\end{frame}




\end{document}
