\PassOptionsToPackage{unicode=true}{hyperref} % options for packages loaded elsewhere
\PassOptionsToPackage{hyphens}{url}
\documentclass[12pt,ignorenonframetext,aspectratio=169]{beamer}
\IfFileExists{pgfpages.sty}{\usepackage{pgfpages}}{}
\setbeamertemplate{caption}[numbered]
\setbeamertemplate{caption label separator}{: }
\setbeamercolor{caption name}{fg=normal text.fg}
\beamertemplatenavigationsymbolsempty
\usepackage{lmodern}
\usepackage{amssymb}
\usepackage{amsmath}
\usepackage{ifxetex,ifluatex}
\usepackage{fixltx2e} % provides \textsubscript
\ifnum 0\ifxetex 1\fi\ifluatex 1\fi=0 % if pdftex
  \usepackage[T1]{fontenc}
  \usepackage[utf8]{inputenc}
\else % if luatex or xelatex
  \ifxetex
    \usepackage{mathspec}
  \else
    \usepackage{fontspec}
\fi
\defaultfontfeatures{Ligatures=TeX,Scale=MatchLowercase}






%
\fi

  \usetheme[]{iqss}






% use upquote if available, for straight quotes in verbatim environments
\IfFileExists{upquote.sty}{\usepackage{upquote}}{}
% use microtype if available
\IfFileExists{microtype.sty}{%
  \usepackage{microtype}
  \UseMicrotypeSet[protrusion]{basicmath} % disable protrusion for tt fonts
}{}


\newif\ifbibliography


\hypersetup{
      pdftitle={Basic concept of farm firms, plant, industry and their interrelationships in agricultural commodities},
        pdfauthor={Deependra Dhakal},
          pdfborder={0 0 0},
    breaklinks=true}
%\urlstyle{same}  % Use monospace font for urls







% Prevent slide breaks in the middle of a paragraph:
\widowpenalties 1 10000
\raggedbottom

  \AtBeginPart{
    \let\insertpartnumber\relax
    \let\partname\relax
    \frame{\partpage}
  }
  \AtBeginSection{
    \ifbibliography
    \else
      \let\insertsectionnumber\relax
      \let\sectionname\relax
      \frame{\sectionpage}
    \fi
  }
  \AtBeginSubsection{
    \let\insertsubsectionnumber\relax
    \let\subsectionname\relax
    \frame{\subsectionpage}
  }



\setlength{\parindent}{0pt}
\setlength{\parskip}{6pt plus 2pt minus 1pt}
\setlength{\emergencystretch}{3em}  % prevent overfull lines
\providecommand{\tightlist}{%
  \setlength{\itemsep}{0pt}\setlength{\parskip}{0pt}}

  \setcounter{secnumdepth}{0}


  \usepackage{booktabs}
  \usepackage{longtable}
  \usepackage{emptypage}
  \usepackage{array}
  \usepackage{multirow}
  \usepackage{wrapfig}
  \usepackage{float}
  \usepackage{colortbl}
  \usepackage{pdflscape}
  \usepackage{tabu}
  \usepackage{threeparttable}
  \usepackage{threeparttablex}
  \usepackage[normalem]{ulem}
  \usepackage{rotating}
  \usepackage{makecell}
  \usepackage{xcolor}
  \usepackage{tikz} % required for image opacity change
  \usepackage[absolute,overlay]{textpos} % for text formatting
  \usepackage[utf8]{inputenc}
  \usetikzlibrary{mindmap,arrows,shapes,positioning,shadows,trees}
  \usepackage[skip=2pt]{caption}

  % this font option is amenable for beamer
  \setbeamerfont{caption}{size=\tiny}


%% IQSS overrides
\iqsssectiontitle{Outline}

\AtBeginSection[]{
  \title{\insertsectionhead}
  {
    \definecolor{white}{rgb}{0.776,0.357,0.157}
    \definecolor{iqss@orange}{rgb}{1,1,1}
    \ifnum \insertmainframenumber > \insertframenumber
    \frame{
      \frametitle{\iqsssectiontitleheader}
      \tableofcontents[currentsection]
    }
    \else
    \frame{
      \frametitle{Backup Slides}
      \tableofcontents[sectionstyle=shaded/shaded,subsectionstyle=shaded/shaded/shaded]
    }
    \fi
  }
}

\AtBeginSubsection[]{}

%%


  \title[]{Basic concept of farm firms, plant, industry and their
interrelationships in agricultural commodities}



  \author[
        Deependra Dhakal
    ]{Deependra Dhakal}

  \institute[
    ]{
    GAASC, Baitadi \and Tribhuwan University
    }

\date[
      \today
  ]{
      \today
        }

\begin{document}

% Hide progress bar and footline on titlepage
  \begin{frame}[plain]
  \titlepage
  \end{frame}



\begin{frame}{Farm}
\protect\hypertarget{farm}{}
\begin{itemize}
\tightlist
\item
  A piece of land where various enterprise are produced, generally
  agricultural commodities.
\item
  Function of land, labor, and capital for producing particular output.
\item
  Types of farm:

  \begin{itemize}
  \tightlist
  \item
    Family farm: Household labor dependent and diversified production
  \item
    Commercial farm: Commercial approach of production, market-oriented
    and has generally around Rs. 250000 sales annually.
  \end{itemize}
\end{itemize}
\end{frame}

\begin{frame}{Firm}
\protect\hypertarget{firm}{}
\begin{itemize}
\tightlist
\item
  A firm has a component of management and someone makes a decision.
\item
  There is some hierarchy and an organizational head.
\item
  Business enterprise are called firm.
\item
  The function of making fundamental policy decision in a firm is
  generally called `entrepreneurship'.
\item
  Technical unit of production is a firm.
\end{itemize}
\end{frame}

\begin{frame}{}
\protect\hypertarget{section}{}
\includegraphics[width=0.7\linewidth]{13-concept_farm_firm_industry_files/figure-beamer/technology-basics-1}
\end{frame}

\begin{frame}{}
\protect\hypertarget{section-1}{}
\textbf{Objective of firm}

\begin{itemize}
\tightlist
\item
  To increase the output/to minimize the cost
\item
  Product levels is limited by the standard of technolgy used and input
\item
  Inputs support production process
\item
  Resources provide platform of production
\end{itemize}
\end{frame}

\begin{frame}{}
\protect\hypertarget{section-2}{}
\begin{figure}
\includegraphics[width=0.55\linewidth]{13-concept_farm_firm_industry_files/figure-beamer/input-output-relation-1} \caption{Relationship between input and output in production shown in terms of marginal productivity (MP).}\label{fig:input-output-relation}
\end{figure}

\begin{itemize}
\tightlist
\item
  Allocation of total resources/income in different factor of
  production. Factor of production paid according to marginal
  productivity.
\end{itemize}
\end{frame}

\begin{frame}{Plant}
\protect\hypertarget{plant}{}
\begin{itemize}
\tightlist
\item
  It is another name of a firm, interchangeable aspect of it.
\item
  \textbf{Objective of plant}

  \begin{itemize}
  \tightlist
  \item
    Technical efficiency: Physical
  \item
    Economic efficiency: Monetary
  \item
    e.g.~crop variety change, cropping system change, fertilizer
    application either in direct broadcast or coated granule
    formulation, technology.
  \end{itemize}
\item
  Thus, plant is also a set of machinery/technology used in production
  process. For example, milk processing plant, jam/jelly processing
  plant.
\item
  Plants are the input of firm and by using this input the firm produce
  certain ouput and this summation gives industry output.
\end{itemize}
\end{frame}

\begin{frame}{Industry}
\protect\hypertarget{industry}{}
\begin{itemize}
\tightlist
\item
  It comprises several similar firms or plants (Group of similar
  firms/plants).
\item
  For e.g.: agricultural industry, livestock industry, poultry industry,
  agri-feed industry, eco-tourism industry, mining industry and coaling
  industry etc.
\item
  Agricultural industry has millions of farms, 2.6 to 2.7 millions of
  farm in Nepalese agri-industry.
\item
  \textbf{Objective of industry}:

  \begin{itemize}
  \tightlist
  \item
    To achieve/increase physical optima or
  \item
    To achieve/increase economic optima
  \end{itemize}
\end{itemize}
\end{frame}

\begin{frame}{}
\protect\hypertarget{section-3}{}
\begin{itemize}
\tightlist
\item
  Physical optima: Level of input used that gives maximum output. This
  is important in cases where physical inputs are insufficient. For
  e.g., country where there is widespread ``malnutrition'', physical
  optima is hard to reach.
\item
  Economic optima: Those countries which achieve self-sufficiency, no
  malnutrition problem working on economic optima, i.e., level of input
  used that maximize the ``profit'';
\end{itemize}

\[
\textrm{Marginal revenue (MR)} = \textrm{Marginal cost (MC)}
\]

\begin{itemize}
\tightlist
\item
  if MR \textgreater{} MC firm applies more inputs to maximize profit
  while MR \textless{} MC firm/producer should apply less inputs.
\end{itemize}
\end{frame}

\begin{frame}{Inter-relationship between firm, plant and industry with
respect to agricultural production}
\protect\hypertarget{inter-relationship-between-firm-plant-and-industry-with-respect-to-agricultural-production}{}
\textbf{Common features of farm and enterprise}

\begin{enumerate}
\tightlist
\item
  Coordination/networking
\item
  Resource/fund flow
\item
  Management and decision-making
\item
  Value adding practices and upgrading
\end{enumerate}

\includegraphics[width=0.45\linewidth]{13-concept_farm_firm_industry_files/figure-beamer/interrelationship-firm-plant-industry-1}
\end{frame}




\end{document}
