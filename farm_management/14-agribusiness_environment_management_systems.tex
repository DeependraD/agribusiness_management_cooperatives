\PassOptionsToPackage{unicode=true}{hyperref} % options for packages loaded elsewhere
\PassOptionsToPackage{hyphens}{url}
\documentclass[12pt,ignorenonframetext,aspectratio=169]{beamer}
\IfFileExists{pgfpages.sty}{\usepackage{pgfpages}}{}
\setbeamertemplate{caption}[numbered]
\setbeamertemplate{caption label separator}{: }
\setbeamercolor{caption name}{fg=normal text.fg}
\beamertemplatenavigationsymbolsempty
\usepackage{lmodern}
\usepackage{amssymb}
\usepackage{amsmath}
\usepackage{ifxetex,ifluatex}
\usepackage{fixltx2e} % provides \textsubscript
\ifnum 0\ifxetex 1\fi\ifluatex 1\fi=0 % if pdftex
  \usepackage[T1]{fontenc}
  \usepackage[utf8]{inputenc}
\else % if luatex or xelatex
  \ifxetex
    \usepackage{mathspec}
  \else
    \usepackage{fontspec}
\fi
\defaultfontfeatures{Ligatures=TeX,Scale=MatchLowercase}






%
\fi

  \usetheme[]{iqss}






% use upquote if available, for straight quotes in verbatim environments
\IfFileExists{upquote.sty}{\usepackage{upquote}}{}
% use microtype if available
\IfFileExists{microtype.sty}{%
  \usepackage{microtype}
  \UseMicrotypeSet[protrusion]{basicmath} % disable protrusion for tt fonts
}{}


\newif\ifbibliography


\hypersetup{
      pdftitle={Agribusiness environment and management system},
        pdfauthor={Deependra Dhakal},
          pdfborder={0 0 0},
    breaklinks=true}
%\urlstyle{same}  % Use monospace font for urls







% Prevent slide breaks in the middle of a paragraph:
\widowpenalties 1 10000
\raggedbottom

  \AtBeginPart{
    \let\insertpartnumber\relax
    \let\partname\relax
    \frame{\partpage}
  }
  \AtBeginSection{
    \ifbibliography
    \else
      \let\insertsectionnumber\relax
      \let\sectionname\relax
      \frame{\sectionpage}
    \fi
  }
  \AtBeginSubsection{
    \let\insertsubsectionnumber\relax
    \let\subsectionname\relax
    \frame{\subsectionpage}
  }



\setlength{\parindent}{0pt}
\setlength{\parskip}{6pt plus 2pt minus 1pt}
\setlength{\emergencystretch}{3em}  % prevent overfull lines
\providecommand{\tightlist}{%
  \setlength{\itemsep}{0pt}\setlength{\parskip}{0pt}}

  \setcounter{secnumdepth}{0}


  \usepackage{booktabs}
  \usepackage{longtable}
  \usepackage{emptypage}
  \usepackage{array}
  \usepackage{multirow}
  \usepackage{wrapfig}
  \usepackage{float}
  \usepackage{colortbl}
  \usepackage{pdflscape}
  \usepackage{tabu}
  \usepackage{threeparttable}
  \usepackage{threeparttablex}
  \usepackage[normalem]{ulem}
  \usepackage{rotating}
  \usepackage{makecell}
  \usepackage{xcolor}
  \usepackage{tikz} % required for image opacity change
  \usepackage[absolute,overlay]{textpos} % for text formatting
  \usepackage[utf8]{inputenc}
  \usetikzlibrary{mindmap,arrows,shapes,positioning,shadows,trees}
  \usepackage[skip=2pt]{caption}

  % this font option is amenable for beamer
  \setbeamerfont{caption}{size=\tiny}


%% IQSS overrides
\iqsssectiontitle{Outline}

\AtBeginSection[]{
  \title{\insertsectionhead}
  {
    \definecolor{white}{rgb}{0.776,0.357,0.157}
    \definecolor{iqss@orange}{rgb}{1,1,1}
    \ifnum \insertmainframenumber > \insertframenumber
    \frame{
      \frametitle{\iqsssectiontitleheader}
      \tableofcontents[currentsection]
    }
    \else
    \frame{
      \frametitle{Backup Slides}
      \tableofcontents[sectionstyle=shaded/shaded,subsectionstyle=shaded/shaded/shaded]
    }
    \fi
  }
}

\AtBeginSubsection[]{}

%%


  \title[]{Agribusiness environment and management system}



  \author[
        Deependra Dhakal
    ]{Deependra Dhakal}

  \institute[
    ]{
    GAASC, Baitadi \and Tribhuwan University
    }

\date[
      \today
  ]{
      \today
        }

\begin{document}

% Hide progress bar and footline on titlepage
  \begin{frame}[plain]
  \titlepage
  \end{frame}



\hypertarget{meaning-and-definition}{%
\section{Meaning and definition}\label{meaning-and-definition}}

\begin{frame}{Environment}
\protect\hypertarget{environment}{}
\begin{itemize}
\tightlist
\item
  It is the situation in which agribusiness firm is operated.
\item
  It checks the potentiality and feasibility of business for overall
  investment
\item
  One of the important steps in business initiation
\item
  Analysis is conducted throughout the business period
\item
  The environmental factors or forces which affect the success of
  agribusiness are:

  \begin{enumerate}
  \tightlist
  \item
    Economic environment
  \item
    Demographic environment
  \item
    Socio-cultural environment
  \item
    Technological environment
  \item
    Political environment
  \item
    Legal environment
  \end{enumerate}
\item
  Overall, it is resultant of those components like government,
  technology, management, organization and individuals (public).
\end{itemize}
\end{frame}

\begin{frame}{}
\protect\hypertarget{section}{}
\begin{itemize}
\tightlist
\item
  In agribusiness there are several producers and several consumers as
  well. So the effect of single producer and consumer is negligible;
  i.e.~they have no single effect on fixing market price, demand and
  supply.
\item
  Therefore the market equilibrium does not change unless external
  forces are operating.
\end{itemize}
\end{frame}

\begin{frame}{Forces affecing agribusiness}
\protect\hypertarget{forces-affecing-agribusiness}{}
\begin{itemize}
\tightlist
\item
  technology changes,
\item
  production (yields per acre, rate of gain per animal, pounds of milk
  per animal),
\item
  local labor (availability and price),
\item
  land and machinery (availability and price),
\item
  consumer and industry demand for your products (tase and preference),
\item
  markets (availability, niche, traditional, organic),
\item
  operating environmental forces,
\item
  conservation programs (regulations),
\item
  family living and medical expenses (current and future),
\item
  macroeconomic forces (inflation, interest, and exchange rates),
\item
  trade agreements (GATT, SAFTA, BIMSTEC), and
\item
  other items unique to the farm business.
\end{itemize}
\end{frame}

\begin{frame}{Importance of agribusiness environment analysis}
\protect\hypertarget{importance-of-agribusiness-environment-analysis}{}
\begin{itemize}
\tightlist
\item
  The manager needs to be dynamic to effectively deal with the
  challenges of the environment as the business environment is
  non-static. Some of the benefits of environment scanning are:
\end{itemize}

\begin{enumerate}
\tightlist
\item
  It creates an increased general awareness of environment changes on
  the part of management.
\item
  It guides with greater effectiveness in matters relating to
  government.
\item
  It helps in marketing analysis.
\item
  It suggests improvement in diversification and resources allocation
\item
  It helps firms to identify and capitalize upon opportunities rather
  than losing out to competitors.
\item
  It provides base of objective qualitative information about the
  business environment that can subsequently be of value in designing
  the strategies.
\end{enumerate}
\end{frame}

\hypertarget{management-system-and-managerial-decisions}{%
\section{Management system and managerial
decisions}\label{management-system-and-managerial-decisions}}

\begin{frame}{Agribusiness management functions}
\protect\hypertarget{agribusiness-management-functions}{}
\begin{columns}

\column{0.6\textwidth}

\footnotesize
1. Planning

\begin{itemize}
\item What to produce ? How to produce ? How much to produce ?
\item Set objectives, formulate product production, processing or marketing strategies and plan accordingly
\item Prepare work schedule and budget accordingly
\item Plan joint venture
\item First prepare managerial work plan and then work under it.
\end{itemize}

\column{0.4\textwidth}


\includegraphics[width=0.28\linewidth]{14-agribusiness_environment_management_systems_files/figure-beamer/management-functions-1} 


\end{columns}
\end{frame}

\begin{frame}{}
\protect\hypertarget{section-1}{}
\begin{enumerate}
\setcounter{enumi}{1}
\tightlist
\item
  Organizing
\end{enumerate}

\begin{itemize}
\tightlist
\item
  Systematic classification and grouping of human and other resources in
  a manner consistent with the firm's goals.
\item
  Organizing task occurs continuously throughout the life of a firm
\item
  Defining line of authority and responsibility
\item
  Decentralization of authority
\item
  Coach and train all employees
\end{itemize}
\end{frame}

\begin{frame}{}
\protect\hypertarget{section-2}{}
\begin{enumerate}
\setcounter{enumi}{2}
\tightlist
\item
  Staffing
\end{enumerate}

\begin{itemize}
\tightlist
\item
  Process of recruiting, selecting, training and developing human
  resources in organization
\end{itemize}

\begin{enumerate}
\setcounter{enumi}{3}
\tightlist
\item
  Directing
\end{enumerate}

\begin{itemize}
\tightlist
\item
  Process of influencing employees to work for the achievement of the
  organizational goals
\end{itemize}

\begin{enumerate}
\setcounter{enumi}{4}
\tightlist
\item
  Controlling
\end{enumerate}

\begin{itemize}
\tightlist
\item
  The collection of relevant feedback information, analyzing of such
  data and as per need taking of corrective action
\item
  It measures current performance of the firm, cooperatives or company.
\end{itemize}
\end{frame}

\begin{frame}{Best management practices}
\protect\hypertarget{best-management-practices}{}
\begin{itemize}
\tightlist
\item
  The best management practices are research driven, industry based and
  proven in practice.
\item
  Enlisting:

  \begin{enumerate}
  \tightlist
  \item
    Visioning
  \item
    Risk management
  \item
    Human resource management
  \item
    Technically innovative business
  \item
    Market proactive and consumer oriented (marketing management)
  \item
    Addressing sustainable development practices
  \end{enumerate}
\end{itemize}
\end{frame}

\begin{frame}{Decision-making at the level of top management}
\protect\hypertarget{decision-making-at-the-level-of-top-management}{}
\begin{itemize}
\tightlist
\item
  Determine the internal allocation of most of the resources.
\item
  Two criteria are set:

  \begin{itemize}
  \tightlist
  \item
    The first is a budgetary (financial) criterion, i.e.~are there fund
    available for the realization of the proposed agribusiness project?
  \item
    The second is an improvement criterion, i.e.~does the agribusiness
    project being proposed improve the existing situation beyond doubt?
  \end{itemize}
\end{itemize}
\end{frame}

\begin{frame}{Decision-making at lower levels of management
(Administration)}
\protect\hypertarget{decision-making-at-lower-levels-of-management-administration}{}
\begin{itemize}
\tightlist
\item
  Adaptive rational system, i.e.~on the floor, day-to-day activities,
  `blue print' and rule-of-thumb.
\item
  It involves choosing the best course of action (choosing among various
  sets of alternatives) for accomplishing goals out of available
  alternatives.
\end{itemize}
\end{frame}




\end{document}
