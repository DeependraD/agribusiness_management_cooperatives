\PassOptionsToPackage{unicode=true}{hyperref} % options for packages loaded elsewhere
\PassOptionsToPackage{hyphens}{url}
\documentclass[12pt,ignorenonframetext,aspectratio=169]{beamer}
\IfFileExists{pgfpages.sty}{\usepackage{pgfpages}}{}
\setbeamertemplate{caption}[numbered]
\setbeamertemplate{caption label separator}{: }
\setbeamercolor{caption name}{fg=normal text.fg}
\beamertemplatenavigationsymbolsempty
\usepackage{lmodern}
\usepackage{amssymb}
\usepackage{amsmath}
\usepackage{ifxetex,ifluatex}
\usepackage{fixltx2e} % provides \textsubscript
\ifnum 0\ifxetex 1\fi\ifluatex 1\fi=0 % if pdftex
  \usepackage[T1]{fontenc}
  \usepackage[utf8]{inputenc}
\else % if luatex or xelatex
  \ifxetex
    \usepackage{mathspec}
  \else
    \usepackage{fontspec}
\fi
\defaultfontfeatures{Ligatures=TeX,Scale=MatchLowercase}






%
\fi

  \usetheme[]{iqss}






% use upquote if available, for straight quotes in verbatim environments
\IfFileExists{upquote.sty}{\usepackage{upquote}}{}
% use microtype if available
\IfFileExists{microtype.sty}{%
  \usepackage{microtype}
  \UseMicrotypeSet[protrusion]{basicmath} % disable protrusion for tt fonts
}{}


\newif\ifbibliography


\hypersetup{
      pdftitle={Human resource, organization and business management functions},
        pdfauthor={Deependra Dhakal},
          pdfborder={0 0 0},
    breaklinks=true}
%\urlstyle{same}  % Use monospace font for urls







% Prevent slide breaks in the middle of a paragraph:
\widowpenalties 1 10000
\raggedbottom

  \AtBeginPart{
    \let\insertpartnumber\relax
    \let\partname\relax
    \frame{\partpage}
  }
  \AtBeginSection{
    \ifbibliography
    \else
      \let\insertsectionnumber\relax
      \let\sectionname\relax
      \frame{\sectionpage}
    \fi
  }
  \AtBeginSubsection{
    \let\insertsubsectionnumber\relax
    \let\subsectionname\relax
    \frame{\subsectionpage}
  }



\setlength{\parindent}{0pt}
\setlength{\parskip}{6pt plus 2pt minus 1pt}
\setlength{\emergencystretch}{3em}  % prevent overfull lines
\providecommand{\tightlist}{%
  \setlength{\itemsep}{0pt}\setlength{\parskip}{0pt}}

  \setcounter{secnumdepth}{0}


  \usepackage{booktabs}
  \usepackage{longtable}
  \usepackage{emptypage}
  \usepackage{array}
  \usepackage{multirow}
  \usepackage{wrapfig}
  \usepackage{float}
  \usepackage{colortbl}
  \usepackage{pdflscape}
  \usepackage{tabu}
  \usepackage{threeparttable}
  \usepackage{threeparttablex}
  \usepackage[normalem]{ulem}
  \usepackage{rotating}
  \usepackage{makecell}
  \usepackage{xcolor}
  \usepackage{tikz} % required for image opacity change
  \usepackage[absolute,overlay]{textpos} % for text formatting
  \usepackage[utf8]{inputenc}
  \usetikzlibrary{mindmap,arrows,shapes,positioning,shadows,trees}
  \usepackage[skip=2pt]{caption}

  % this font option is amenable for beamer
  \setbeamerfont{caption}{size=\tiny}


%% IQSS overrides
\iqsssectiontitle{Outline}

\AtBeginSection[]{
  \title{\insertsectionhead}
  {
    \definecolor{white}{rgb}{0.776,0.357,0.157}
    \definecolor{iqss@orange}{rgb}{1,1,1}
    \ifnum \insertmainframenumber > \insertframenumber
    \frame{
      \frametitle{\iqsssectiontitleheader}
      \tableofcontents[currentsection]
    }
    \else
    \frame{
      \frametitle{Backup Slides}
      \tableofcontents[sectionstyle=shaded/shaded,subsectionstyle=shaded/shaded/shaded]
    }
    \fi
  }
}

\AtBeginSubsection[]{}

%%


  \title[]{Human resource, organization and business management
functions}



  \author[
        Deependra Dhakal
    ]{Deependra Dhakal}

  \institute[
    ]{
    GAASC, Baitadi \and Tribhuwan University
    }

\date[
      \today
  ]{
      \today
        }

\begin{document}

% Hide progress bar and footline on titlepage
  \begin{frame}[plain]
  \titlepage
  \end{frame}



\hypertarget{human-resource-management-in-organization}{%
\section{Human resource management in
organization}\label{human-resource-management-in-organization}}

\begin{frame}{Meaning}
\protect\hypertarget{meaning}{}
\begin{itemize}
\tightlist
\item
  Human resource are the people working for the organization
\item
  Human Resource Management (HRM) is the term used to describe formal
  systems developed for the management of people within an organization.
\item
  It involves finding out the needs and aspirations of employees and
  creating opportunities within the organization (job enlargement,
  position creation) and outside the organization (support for advance
  learning, training and exposure) for employees to improve themselves.
\item
  The responsibilities of a human resource manager fall into three major
  areas: staffing, employee compensation and benefits, and
  defining/designing work.
\item
  Essentially, the purpose of HRM is to maximize the productivity of an
  organization by optimizing the effectiveness of its employees.
\end{itemize}
\end{frame}

\begin{frame}{Functions}
\protect\hypertarget{functions}{}
\begin{itemize}
\tightlist
\item
  Recruiting employees
\item
  Fixing position in the organogram
\item
  Fixing remuneration and incentives
\item
  Fixing time rate to be worked
\item
  Providing trainings, visits, exposure or advanced education for
  potential employees
\item
  Assess job satisfaction
\item
  Award, punish or fire staffs
\end{itemize}
\end{frame}

\begin{frame}{Human behavior in organization}
\protect\hypertarget{human-behavior-in-organization}{}
\begin{itemize}
\tightlist
\item
  Mobilizing employee means mobilizing human factors in the organization
\item
  Employee work is perishable
\item
  Employee is a mobile factor
\item
  Employee has different efficiencies
\end{itemize}
\end{frame}

\begin{frame}{Organizational behavior}
\protect\hypertarget{organizational-behavior}{}
\begin{itemize}
\tightlist
\item
  Organizational behavior is the system of culture, leadership,
  communication and group dynamics that determines an organization's
  actions.
\item
  According to Keith Davis, ``It is the study and application of
  knowledge about how people act within an organization''
\item
  Employee have a variety of needs, irrespective of one's status, age
  and achievements;
\item
  Human behavior in organizations is as complex as the social system
  itself.
\item
  The organizational system consists of social, technical and economic
  elements which coordinate human and material resources to achieve
  various organizational objectives
\item
  If understood and addressed properly, OB improves people's
  understanding of interpersonal skills and so also the ability to work
  together as a team to achieve organizational goals effectively.
\end{itemize}
\end{frame}

\begin{frame}{Theories of human motivation (Theory of X and Y)}
\protect\hypertarget{theories-of-human-motivation-theory-of-x-and-y}{}
\begin{itemize}
\tightlist
\item
  Douglas McGreator and MIT Sloan School of Management (1960s)
  introduced Theory X and Theory Y
\item
  Theory specially used for human resource management, organizational
  behavior and organizational development.
\end{itemize}

\textbf{Theory X}

\begin{itemize}
\tightlist
\item
  This theory of management assumes that employees are inherently lazy
  and would like to avoid work responsibility as much as possible.
\item
  Because of this reason employees need to be closely supervised and
  comprehensive system of controls developed
\item
  A hierarchical structure is needed with the narrow span of control at
  each level.
\item
  Pessimistic view of employees.
\item
  The major problem of this theory is causing diseconomies of scale in
  large business, increases administrative cost.
\end{itemize}
\end{frame}

\begin{frame}{}
\protect\hypertarget{section}{}
\textbf{Theory Y}

\begin{itemize}
\tightlist
\item
  This theory of management assumes that employees may be ambitious,
  self motivated, anxious to accept greater responsibility and exercise
  self control and self direction
\item
  It is believed that employees enjoy their mental and physical work
  duties
\item
  If management gives chances and freedoms employees create creativity
  and forward thinking in the work place, enhance workforce productivity
\item
  It is positive set of assumptions about the workers
\item
  Thus managers need to be open to a more positive views of workers and
  create many management possibilities for employees
\end{itemize}
\end{frame}

\hypertarget{organization-and-business-management-functions}{%
\section{Organization and business management
functions}\label{organization-and-business-management-functions}}

\begin{frame}{}
\protect\hypertarget{section-1}{}
\begin{itemize}
\tightlist
\item
  A social unit of people that is structured and managed to meet a need
  or to pursue collective goals. (Dictionary)
\item
  A social unit where individuals from diverse backgrounds, different
  educational qualifications and varied interests come together to work
  towards a common goal is called an organization.
\item
  The employees must work in close coordination with each other and try
  their level best to achieve the organization's goals.
\item
  It is essential to manage the employees well for them to feel
  indispensable for the organization.
\item
  Organization management helps to extract the best out of each employee
  so that they accomplish the tasks within the given time frame.
\item
  Organization management binds the employees together and gives them a
  sense of loyalty towards the organization.
\end{itemize}
\end{frame}

\begin{frame}{}
\protect\hypertarget{section-2}{}
\begin{itemize}
\tightlist
\item
  An effective management ensures profitability for the organization.
\item
  Organization management refers to efficient handling of the
  organization as well as its employees.
\item
  Even though agribusiness organizations are smaller and leaner, they
  have own management committee and with assigned role and
  responsibility in organizational structure
\item
  Example of business organization:

  \begin{itemize}
  \tightlist
  \item
    Farmer's, producer's group, cooperatives, private firm, association,
    company, etc.
  \end{itemize}
\item
  Organizational behavior is the study and application of knowledge
  about how people as individuals and as groups act within organization
\end{itemize}
\end{frame}

\hypertarget{managerial-decision-process-in-agribusiness}{%
\section{Managerial decision process in
agribusiness}\label{managerial-decision-process-in-agribusiness}}

\begin{frame}{Role of agribusiness manager}
\protect\hypertarget{role-of-agribusiness-manager}{}
\begin{itemize}
\tightlist
\item
  Creating environment wherein groups of people can work effectively and
  efficiently for achievement of goals
\item
  Planning, organizing, staffing, directing and controlling are basic
  roles
\item
  Support junior manager and other staffs in official activities
\item
  Lead for inventory management and elimination of unnecessary tasks and
  procedures
\item
  Flow of information, logical support in the business system and sub
  system
\item
  Maintain ethical and moral behavior in the organization
\end{itemize}
\end{frame}

\begin{frame}{The process}
\protect\hypertarget{the-process}{}
\begin{figure}
\includegraphics[width=0.95\linewidth]{15-human_resource_organization_business_management_files/figure-beamer/decision-process-1} \caption{Rational deision making process (Stronger, Freedman and Gilbert, 1995)}\label{fig:decision-process}
\end{figure}
\end{frame}




\end{document}
