\PassOptionsToPackage{unicode=true}{hyperref} % options for packages loaded elsewhere
\PassOptionsToPackage{hyphens}{url}
\documentclass[12pt,ignorenonframetext,aspectratio=169]{beamer}
\IfFileExists{pgfpages.sty}{\usepackage{pgfpages}}{}
\setbeamertemplate{caption}[numbered]
\setbeamertemplate{caption label separator}{: }
\setbeamercolor{caption name}{fg=normal text.fg}
\beamertemplatenavigationsymbolsempty
\usepackage{lmodern}
\usepackage{amssymb}
\usepackage{amsmath}
\usepackage{ifxetex,ifluatex}
\usepackage{fixltx2e} % provides \textsubscript
\ifnum 0\ifxetex 1\fi\ifluatex 1\fi=0 % if pdftex
  \usepackage[T1]{fontenc}
  \usepackage[utf8]{inputenc}
\else % if luatex or xelatex
  \ifxetex
    \usepackage{mathspec}
  \else
    \usepackage{fontspec}
\fi
\defaultfontfeatures{Ligatures=TeX,Scale=MatchLowercase}






%
\fi

  \usetheme[]{iqss}






% use upquote if available, for straight quotes in verbatim environments
\IfFileExists{upquote.sty}{\usepackage{upquote}}{}
% use microtype if available
\IfFileExists{microtype.sty}{%
  \usepackage{microtype}
  \UseMicrotypeSet[protrusion]{basicmath} % disable protrusion for tt fonts
}{}


\newif\ifbibliography


\hypersetup{
      pdftitle={Production planning in agribusiness and risk and uncertainity management},
        pdfauthor={Deependra Dhakal},
          pdfborder={0 0 0},
    breaklinks=true}
%\urlstyle{same}  % Use monospace font for urls







% Prevent slide breaks in the middle of a paragraph:
\widowpenalties 1 10000
\raggedbottom

  \AtBeginPart{
    \let\insertpartnumber\relax
    \let\partname\relax
    \frame{\partpage}
  }
  \AtBeginSection{
    \ifbibliography
    \else
      \let\insertsectionnumber\relax
      \let\sectionname\relax
      \frame{\sectionpage}
    \fi
  }
  \AtBeginSubsection{
    \let\insertsubsectionnumber\relax
    \let\subsectionname\relax
    \frame{\subsectionpage}
  }



\setlength{\parindent}{0pt}
\setlength{\parskip}{6pt plus 2pt minus 1pt}
\setlength{\emergencystretch}{3em}  % prevent overfull lines
\providecommand{\tightlist}{%
  \setlength{\itemsep}{0pt}\setlength{\parskip}{0pt}}

  \setcounter{secnumdepth}{0}


  \usepackage{booktabs}
  \usepackage{longtable}
  \usepackage{emptypage}
  \usepackage{array}
  \usepackage{multirow}
  \usepackage{wrapfig}
  \usepackage{float}
  \usepackage{colortbl}
  \usepackage{pdflscape}
  \usepackage{tabu}
  \usepackage{threeparttable}
  \usepackage{threeparttablex}
  \usepackage[normalem]{ulem}
  \usepackage{rotating}
  \usepackage{makecell}
  \usepackage{xcolor}
  \usepackage{tikz} % required for image opacity change
  \usepackage[absolute,overlay]{textpos} % for text formatting
  \usepackage[utf8]{inputenc}
  \usetikzlibrary{mindmap,arrows,shapes,positioning,shadows,trees}
  \usepackage[skip=2pt]{caption}

  % this font option is amenable for beamer
  \setbeamerfont{caption}{size=\tiny}


%% IQSS overrides
\iqsssectiontitle{Outline}

\AtBeginSection[]{
  \title{\insertsectionhead}
  {
    \definecolor{white}{rgb}{0.776,0.357,0.157}
    \definecolor{iqss@orange}{rgb}{1,1,1}
    \ifnum \insertmainframenumber > \insertframenumber
    \frame{
      \frametitle{\iqsssectiontitleheader}
      \tableofcontents[currentsection]
    }
    \else
    \frame{
      \frametitle{Backup Slides}
      \tableofcontents[sectionstyle=shaded/shaded,subsectionstyle=shaded/shaded/shaded]
    }
    \fi
  }
}

\AtBeginSubsection[]{}

%%


  \title[]{Production planning in agribusiness and risk and uncertainity
management}



  \author[
        Deependra Dhakal
    ]{Deependra Dhakal}

  \institute[
    ]{
    GAASC, Baitadi \and Tribhuwan University
    }

\date[
      \today
  ]{
      \today
        }

\begin{document}

% Hide progress bar and footline on titlepage
  \begin{frame}[plain]
  \titlepage
  \end{frame}



\hypertarget{production-planning-in-agribusiness}{%
\section{Production planning in
agribusiness}\label{production-planning-in-agribusiness}}

\begin{frame}{Background\footnote<.->{Production planning is related to
  earlier lecture on Farm Planning}}
\protect\hypertarget{background}{}
\begin{itemize}
\tightlist
\item
  All business activities ranging from small corner shop to large
  commercial firms should plan their production and marketing operations
\item
  Planning is secret of economic success of business
\item
  With recent technological development in agriculture, farming has
  become more complex business and requires careful planning for
  successful operation
\item
  Fail to make a plan is a plan to fail
\end{itemize}
\end{frame}

\begin{frame}{Advantages of planning}
\protect\hypertarget{advantages-of-planning}{}
\begin{enumerate}
\tightlist
\item
  Income improvement
\end{enumerate}

\begin{itemize}
\tightlist
\item
  Farm planning is an integrated, coordinated and advance programmes of
  actions which show opportunity to cultivators to improve in income
\item
  Income maximization can be achieved through by efficient utilization
  of present resources/technologies and introduction of new technology
\item
  Farm planning gives idea about this
\end{itemize}
\end{frame}

\begin{frame}{}
\protect\hypertarget{section}{}
\begin{enumerate}
\setcounter{enumi}{1}
\tightlist
\item
  Educational process
\end{enumerate}

\begin{itemize}
\tightlist
\item
  FP is an educational tool to bring a change in the outlook of the
  cultivators and the extension workers
\item
  Knowledge in advances in agriculture is pre-requisite for better farm
  planning
\item
  So, farmers keep their information up-to-date by farm planning
\item
  This acts as a self-educating tool for the farmers
\item
  Farmers can closely study their own business and see more clearly
  their opportunities and limitations, thus, improving managerial
  ability
\end{itemize}
\end{frame}

\begin{frame}{}
\protect\hypertarget{section-1}{}
\begin{enumerate}
\setcounter{enumi}{2}
\tightlist
\item
  Desirable organizational changes
\end{enumerate}

\begin{itemize}
\tightlist
\item
  FM is an approach which introduces desirable changes in farm
  organization and operations and makes the farm a viable unit
\end{itemize}
\end{frame}

\begin{frame}{Objectives of planning}
\protect\hypertarget{objectives-of-planning}{}
\begin{itemize}
\tightlist
\item
  Ultimate objective of FP is the improvement in the standard of living
  of farmer
\item
  Immediate goal is to maximize the net income of the farmer
\end{itemize}
\end{frame}

\begin{frame}{Characteristics of a good plan}
\protect\hypertarget{characteristics-of-a-good-plan}{}
\begin{enumerate}
\tightlist
\item
  It should provide ways for efficient use of farm resources such as
  labor, power and equipment
\item
  The crop plan should have balanced combinations of enterprises,
\item
  Avoid excessive risks
\item
  Provide flexibility
\item
  Utilize the farmers knowledge, training and experience and take
  account of the farmers likes and dislikes
\item
  Give considerations to efficient marketing facilities
\item
  Provide program of obtaining, using and repaying the credit
\item
  Provide for the use of up-to-date modern agricultural methods and
  practices
\item
  Provide regular income and employment to the farm families
\item
  Do not produce negative environmental impact
\end{enumerate}
\end{frame}

\begin{frame}{Steps in farm planing and budgeting}
\protect\hypertarget{steps-in-farm-planing-and-budgeting}{}
\footnotesize

\begin{enumerate}
\tightlist
\item
  Specification of the technical coefficients of production:
  specification of technical coefficients for producing different
  outputs
\item
  Specification of appropriate prices: price forecasting for outputs
  (last three years average price or last year price plus inflation)
\item
  Preparation of enterprise profitability chart: prepare profitability
  ranking charts for enterprises based on net return
\item
  Preparation of farm map: prepare farm map showing all physical
  features of farm
\item
  Preparation of inventory of limited farm resources: prepare list of
  all resources like land, labor, animals, buildings, machines, cash and
  farm products along with their value
\item
  Examine the existing farm plan: evaluate present plan on the basis of
  costs, returns and resource use patterns
\end{enumerate}

\begin{itemize}
\tightlist
\item
  Work out variable cost for each enterprise
\item
  Work out gross income for various enterprises
\item
  Calculate returns to fixed farm resources (gross income variable cost)
\item
  Work out net profit (gross income-total cost)
\end{itemize}
\end{frame}

\begin{frame}{}
\protect\hypertarget{section-2}{}
\begin{enumerate}
\setcounter{enumi}{6}
\tightlist
\item
  Locate the weakness of present plan: identifying weaknesses of present
  plan may serve as guidelines for improving alternative plan
\item
  List the risks in farm production: identify probable risks and think
  about effective strategies to cope them
\item
  Prepare the alternative plans: alternative plans can be developed
  considering resource restriction and weakness of existing plan and
  possibilities of incorporating modern technology
\item
  Analyzing alternative plans: new plans are analyzed for costs and
  returns and practicability and the optimum one which shows highest
  returns is selected
\item
  Implementing the plan: execution of plan
\end{enumerate}
\end{frame}

\hypertarget{uncertainty-and-risk-management}{%
\section{Uncertainty and risk
management}\label{uncertainty-and-risk-management}}

\begin{frame}{Background}
\protect\hypertarget{background-1}{}
\begin{itemize}
\tightlist
\item
  Perfect knowledge situation in agribusiness management decisions is
  far from reality
\item
  It is necessary to study the effect of technical progress on the
  production relation and incorporate all the complications due to time
  and risk and uncertainty in decision making
\item
  Consideration of such aspects should help to arrive at some
  adjustments with the introduction of time and risk and uncertainty
  aspects
\end{itemize}
\end{frame}

\begin{frame}{Decision making under risk and uncertainty}
\protect\hypertarget{decision-making-under-risk-and-uncertainty}{}
\begin{figure}
\includegraphics[width=0.45\linewidth]{18-production_planning_in_agribusiness_files/figure-beamer/decision-risk-uncertainty-1} \caption{Frank Knight's classification of knowledge situation.}\label{fig:decision-risk-uncertainty}
\end{figure}
\end{frame}

\begin{frame}{}
\protect\hypertarget{section-3}{}
\footnotesize

\textbf{Perfect knowledge}

\begin{itemize}
\tightlist
\item
  Every thing (technology, price, organizational behavior etc) about the
  future of business is know with certainty
\item
  No need of farm management expert
\item
  But does not reflect the real world situation
\end{itemize}

\textbf{Imperfect knowledge}

\begin{itemize}
\tightlist
\item
  May be either risk or uncertainty
\item
  Risk represents less imperfection in knowledge than does uncertainty
\item
  Under risk the occurrence of future events can be predicted fairly
  accurately by specifying the level of probability
\item
  When a risk situation prevails, it can be said, for instance, that the
  chances of a hailstorm at the time of harvesting wheat are 5:95 or
  20:80.
\item
  In practice, however, farmers are unable to draw a clear distinction
  between risk and uncertainty, though the reaction in each situation is
  markedly different
\item
  Thus in most cases risk and uncertainty are taken as similar in
  decision making
\end{itemize}
\end{frame}

\begin{frame}{Types of risk and uncertainty}
\protect\hypertarget{types-of-risk-and-uncertainty}{}
\begin{enumerate}
\tightlist
\item
  Economic uncertainties
\end{enumerate}

\begin{itemize}
\tightlist
\item
  Input and output price uncertainties
\item
  In many developed countries this uncertainty are reduced by price
  announcement before crop season
\item
  This uncertainty is caused by national and international policies
  which are beyond the approach of individual farmer
\end{itemize}

\begin{enumerate}
\setcounter{enumi}{1}
\tightlist
\item
  Biological uncertainties
\end{enumerate}

\begin{itemize}
\tightlist
\item
  Common and important in agriculture
\item
  Rain, drought, flood, hailstorm, frost may cause disease and pest
  incidence
\end{itemize}
\end{frame}

\begin{frame}{}
\protect\hypertarget{section-4}{}
\footnotesize

\begin{enumerate}
\setcounter{enumi}{2}
\tightlist
\item
  Technological uncertainty
\end{enumerate}

\begin{itemize}
\tightlist
\item
  Continuous advancement of knowledge through research
\item
  New technology (method, practice, raw material, market etc) available
  to farmers - increased efficiency of production
\end{itemize}

\begin{enumerate}
\setcounter{enumi}{3}
\tightlist
\item
  Institutional uncertainty
\end{enumerate}

\begin{itemize}
\tightlist
\item
  Government, banks may cause uncertainties for farmers
\item
  Credit squeeze, price supports, subsidies etc may be enforced and
  withdrawn without considering individual farmers
\end{itemize}

\begin{enumerate}
\setcounter{enumi}{4}
\tightlist
\item
  Personal uncertainty
\end{enumerate}

\begin{itemize}
\tightlist
\item
  Unexpected happening in farmers' household or its labor
\item
  Sum of production and technical risk, marketing and price risk, and
  personal risk is called business risk.
\end{itemize}
\end{frame}

\begin{frame}{Risk attitudes}
\protect\hypertarget{risk-attitudes}{}
\begin{enumerate}
\tightlist
\item
  Risk avoider/avert
\item
  Risk neutral
\item
  Risk bearer/taker
\end{enumerate}
\end{frame}

\begin{frame}{Risk management}
\protect\hypertarget{risk-management}{}
\begin{itemize}
\tightlist
\item
  It is a way for an organization to avoid losses and maximize
  opportunities
\item
  Steps in project risk management

  \begin{enumerate}
  \tightlist
  \item
    Establishing context
  \item
    Identifying important risky decision problems
  \item
    Structure problems
  \item
    Analyze options and consequences
  \item
    Evaluate and decide
  \item
    Implement and manage
  \item
    Monitoring and review
  \end{enumerate}
\end{itemize}
\end{frame}

\begin{frame}{Risk management strategies}
\protect\hypertarget{risk-management-strategies}{}
\begin{itemize}
\tightlist
\item
  Some farmers take more risk than others
\item
  However, every farmer take some safeguarding measures to minimize loss
\item
  Some measures to safeguard against risk and uncertainty are:

  \begin{enumerate}
  \tightlist
  \item
    Selection of enterprises with low variability
  \item
    Insurance
  \item
    Forward contract
  \item
    Flexibility
  \item
    Liquidity and asset management
  \item
    Diversification
  \item
    Maintaining resource in reserve
  \end{enumerate}
\end{itemize}
\end{frame}




\end{document}
