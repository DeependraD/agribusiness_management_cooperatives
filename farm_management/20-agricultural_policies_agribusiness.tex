\PassOptionsToPackage{unicode=true}{hyperref} % options for packages loaded elsewhere
\PassOptionsToPackage{hyphens}{url}
\documentclass[12pt,ignorenonframetext,aspectratio=169]{beamer}
\IfFileExists{pgfpages.sty}{\usepackage{pgfpages}}{}
\setbeamertemplate{caption}[numbered]
\setbeamertemplate{caption label separator}{: }
\setbeamercolor{caption name}{fg=normal text.fg}
\beamertemplatenavigationsymbolsempty
\usepackage{lmodern}
\usepackage{amssymb}
\usepackage{amsmath}
\usepackage{ifxetex,ifluatex}
\usepackage{fixltx2e} % provides \textsubscript
\ifnum 0\ifxetex 1\fi\ifluatex 1\fi=0 % if pdftex
  \usepackage[T1]{fontenc}
  \usepackage[utf8]{inputenc}
\else % if luatex or xelatex
  \ifxetex
    \usepackage{mathspec}
  \else
    \usepackage{fontspec}
\fi
\defaultfontfeatures{Ligatures=TeX,Scale=MatchLowercase}






%
\fi

  \usetheme[]{iqss}






% use upquote if available, for straight quotes in verbatim environments
\IfFileExists{upquote.sty}{\usepackage{upquote}}{}
% use microtype if available
\IfFileExists{microtype.sty}{%
  \usepackage{microtype}
  \UseMicrotypeSet[protrusion]{basicmath} % disable protrusion for tt fonts
}{}


\newif\ifbibliography


\hypersetup{
      pdftitle={Agricultural policies},
        pdfauthor={Deependra Dhakal},
          pdfborder={0 0 0},
    breaklinks=true}
%\urlstyle{same}  % Use monospace font for urls







% Prevent slide breaks in the middle of a paragraph:
\widowpenalties 1 10000
\raggedbottom

  \AtBeginPart{
    \let\insertpartnumber\relax
    \let\partname\relax
    \frame{\partpage}
  }
  \AtBeginSection{
    \ifbibliography
    \else
      \let\insertsectionnumber\relax
      \let\sectionname\relax
      \frame{\sectionpage}
    \fi
  }
  \AtBeginSubsection{
    \let\insertsubsectionnumber\relax
    \let\subsectionname\relax
    \frame{\subsectionpage}
  }



\setlength{\parindent}{0pt}
\setlength{\parskip}{6pt plus 2pt minus 1pt}
\setlength{\emergencystretch}{3em}  % prevent overfull lines
\providecommand{\tightlist}{%
  \setlength{\itemsep}{0pt}\setlength{\parskip}{0pt}}

  \setcounter{secnumdepth}{0}


  \usepackage{booktabs}
  \usepackage{longtable}
  \usepackage{emptypage}
  \usepackage{array}
  \usepackage{multirow}
  \usepackage{wrapfig}
  \usepackage{float}
  \usepackage{colortbl}
  \usepackage{pdflscape}
  \usepackage{tabu}
  \usepackage{threeparttable}
  \usepackage{threeparttablex}
  \usepackage[normalem]{ulem}
  \usepackage{rotating}
  \usepackage{makecell}
  \usepackage{xcolor}
  \usepackage{tikz} % required for image opacity change
  \usepackage[absolute,overlay]{textpos} % for text formatting
  \usepackage[utf8]{inputenc}
  \usetikzlibrary{mindmap,arrows,shapes,positioning,shadows,trees}
  \usepackage[skip=2pt]{caption}

  % this font option is amenable for beamer
  \setbeamerfont{caption}{size=\tiny}


%% IQSS overrides
\iqsssectiontitle{Outline}

\AtBeginSection[]{
  \title{\insertsectionhead}
  {
    \definecolor{white}{rgb}{0.776,0.357,0.157}
    \definecolor{iqss@orange}{rgb}{1,1,1}
    \ifnum \insertmainframenumber > \insertframenumber
    \frame{
      \frametitle{\iqsssectiontitleheader}
      \tableofcontents[currentsection]
    }
    \else
    \frame{
      \frametitle{Backup Slides}
      \tableofcontents[sectionstyle=shaded/shaded,subsectionstyle=shaded/shaded/shaded]
    }
    \fi
  }
}

\AtBeginSubsection[]{}

%%


  \title[]{Agricultural policies}



  \author[
        Deependra Dhakal
    ]{Deependra Dhakal}

  \institute[
    ]{
    GAASC, Baitadi \and Tribhuwan University
    }

\date[
      \today
  ]{
      \today
        }

\begin{document}

% Hide progress bar and footline on titlepage
  \begin{frame}[plain]
  \titlepage
  \end{frame}



\hypertarget{agribusiness-related-policies-by-government}{%
\section{Agribusiness related policies by
government}\label{agribusiness-related-policies-by-government}}

\begin{frame}{}
\protect\hypertarget{section}{}
\begin{itemize}
\tightlist
\item
  Policies

  \begin{itemize}
  \tightlist
  \item
    Trade policy 1992
  \item
    National Agricultural Policy 2004
  \item
    Agribusiness Promotion policy 2006 AD/2063 BS
  \item
    National Tea Policy 2057 (2000),
  \item
    National Coffee Policy 2060 (2003)
  \item
    Floral Promotion Policy 2069 (2012)
  \end{itemize}
\item
  Plans

  \begin{itemize}
  \tightlist
  \item
    Agriculture Perspective plan (APP) 1995 to 2015
  \item
    Periodic plans
  \end{itemize}
\item
  Strategies

  \begin{itemize}
  \tightlist
  \item
    Nepal trade integration strategy 2010
  \item
    Nepal trade integration strategy 2016
  \item
    Agricultural Development Strategy 2015 to 2035
  \end{itemize}
\end{itemize}
\end{frame}

\hypertarget{policies}{%
\section{Policies}\label{policies}}

\begin{frame}{The agribusiness promotion policy, 2063}
\protect\hypertarget{the-agribusiness-promotion-policy-2063}{}
\footnotesize

\begin{itemize}
\tightlist
\item
  The Agri-Business Promotion Policy highlights the diversification,
  commercialization and promotion of agriculture sector with private
  sector involvement in commercial farming.
\item
  It emphasizes that the living standard of the farmer would not improve
  unless the agriculture sector is transformed from subsistence level to
  commercial farming.
\item
  The policy aims to reduce poverty by encouraging production of
  market-oriented and competitive agro-products.
\item
  It realizes the need of promoting internal and external markets.
\item
  This policy was prepared in the spirit of National Agriculture Policy
  2061 emphasizing business service centers establishment for quality
  agriculture inputs and services.
\item
  Partnership between the private sector and Government has been
  emphasized for the export of quality goods.
\item
  In the context of Nepal's entry into the WTO, developing market
  network is its priority.
\item
  The policy considers infrastructure development as a cornerstone for
  commercialization and has envisaged promotion of partnership approach
  between Government and the private sector.
\end{itemize}
\end{frame}

\begin{frame}{}
\protect\hypertarget{section-1}{}
\begin{itemize}
\tightlist
\item
  Nepal Government has launched Agri-business Promotion Policy in 2063
  (2006 AD) with giving high priority on diversification, modernization,
  commercialization and promotion of agriculture sector.
\item
  This policy was prepared in the spirit of National Agriculture Policy,
  2061 emphasizing agriculture production as agri-business or an
  enterprise.
\item
  It realizes the important role of private sector to promote commercial
  farming.
\item
  It has emphasized the transformation of subsistence agriculture into
  commercial agriculture which is imperative to improve the living
  standard of the farmers from their current situation.
\end{itemize}
\end{frame}

\begin{frame}{Objectives}
\protect\hypertarget{objectives}{}
\begin{itemize}
\tightlist
\item
  Enhance and/or support market oriented and competitive agricultural
  production;
\item
  Contribute to promote domestic and export markets by developing
  agro-based industries; and
\item
  Reduce poverty through the commercialization of agriculture sector.
\end{itemize}
\end{frame}

\begin{frame}{}
\protect\hypertarget{section-2}{}
\begin{itemize}
\tightlist
\item
  The policy has identified a total of 44 strategic issues as its main
  policies for the commercialization of agriculture and poverty
  reduction. Among them a few are:

  \begin{itemize}
  \tightlist
  \item
    Focus on the formation and promotion of Massive Production/Growth
    Centres based on geographic, technical and economic possibilities
    (and/or based on the comparative advantages),
  \item
    Emphasis on special economic zones (for agriculture production,
    manufacturing and export),
  \item
    Services such as production inputs, technologies and technical
    services, agriculture roads, rural electricity, irrigation, credit,
    insurance, market management, information system, appropriate
    agriculture mechanization and processing etc. will be assured to the
    specified production area in a integrated way through the
    collaboration and coordination of government,
  \end{itemize}
\end{itemize}
\end{frame}

\begin{frame}{}
\protect\hypertarget{section-3}{}
\begin{itemize}
\item
  \begin{itemize}
  \tightlist
  \item
    Establishment and operation of Commercial Service Centres (CSCs) to
    provide commercial services for high quality production inputs and
    for the collection of farm produces, storage, processing,
    transportation and market price based on the area of trade and
    geographic locations. This will be managed through the participation
    of GOs and NGOs, GOs, Cooperative and private sector,
  \item
    Promotion of partnership approach between government and the private
    sector for agriculture development.
  \item
    So, it has emphasized private sectors involvement for the export of
    quality goods and market Network.
  \end{itemize}
\end{itemize}
\end{frame}

\begin{frame}{}
\protect\hypertarget{section-4}{}
\begin{itemize}
\item
  \begin{itemize}
  \tightlist
  \item
    Establishment and operation of Commercial Service Centres (CSCs) to
    provide commercial services for high quality production inputs and
    for the collection of farm produces, storage, processing,
    transportation and market price based on the area of trade and
    geographic locations. This will be managed through the participation
    of GOs and NGOs,
  \item
    Prerequisites of agribusiness such as irrigation, agriculture roads,
    collection centres, various types of refrigerated storage, rural
    electrification, development of appropriate technologies, laboratory
    services etc. will be promoted through the collaboration of
    government, private sector, NGOs, cooperative and civil societies
  \item
    Provision of 75\% duty-free on the import of agri-business
    machineries such as thresher, sprinkler, weeding machine, harvester,
    chilling vat, cooling Vat, milk processer, agriculture tools and
    implements etc.
  \end{itemize}
\end{itemize}
\end{frame}




\end{document}
