\PassOptionsToPackage{unicode=true}{hyperref} % options for packages loaded elsewhere
\PassOptionsToPackage{hyphens}{url}
\documentclass[12pt,ignorenonframetext,aspectratio=169]{beamer}
\IfFileExists{pgfpages.sty}{\usepackage{pgfpages}}{}
\setbeamertemplate{caption}[numbered]
\setbeamertemplate{caption label separator}{: }
\setbeamercolor{caption name}{fg=normal text.fg}
\beamertemplatenavigationsymbolsempty
\usepackage{lmodern}
\usepackage{amssymb}
\usepackage{amsmath}
\usepackage{ifxetex,ifluatex}
\usepackage{fixltx2e} % provides \textsubscript
\ifnum 0\ifxetex 1\fi\ifluatex 1\fi=0 % if pdftex
  \usepackage[T1]{fontenc}
  \usepackage[utf8]{inputenc}
\else % if luatex or xelatex
  \ifxetex
    \usepackage{mathspec}
  \else
    \usepackage{fontspec}
\fi
\defaultfontfeatures{Ligatures=TeX,Scale=MatchLowercase}






%
\fi

  \usetheme[]{iqss}






% use upquote if available, for straight quotes in verbatim environments
\IfFileExists{upquote.sty}{\usepackage{upquote}}{}
% use microtype if available
\IfFileExists{microtype.sty}{%
  \usepackage{microtype}
  \UseMicrotypeSet[protrusion]{basicmath} % disable protrusion for tt fonts
}{}


\newif\ifbibliography


\hypersetup{
      pdftitle={Concept of cooperatives},
        pdfauthor={Deependra Dhakal},
          pdfborder={0 0 0},
    breaklinks=true}
%\urlstyle{same}  % Use monospace font for urls







% Prevent slide breaks in the middle of a paragraph:
\widowpenalties 1 10000
\raggedbottom

  \AtBeginPart{
    \let\insertpartnumber\relax
    \let\partname\relax
    \frame{\partpage}
  }
  \AtBeginSection{
    \ifbibliography
    \else
      \let\insertsectionnumber\relax
      \let\sectionname\relax
      \frame{\sectionpage}
    \fi
  }
  \AtBeginSubsection{
    \let\insertsubsectionnumber\relax
    \let\subsectionname\relax
    \frame{\subsectionpage}
  }



\setlength{\parindent}{0pt}
\setlength{\parskip}{6pt plus 2pt minus 1pt}
\setlength{\emergencystretch}{3em}  % prevent overfull lines
\providecommand{\tightlist}{%
  \setlength{\itemsep}{0pt}\setlength{\parskip}{0pt}}

  \setcounter{secnumdepth}{0}


  \usepackage{booktabs}
  \usepackage{longtable}
  \usepackage{emptypage}
  \usepackage{array}
  \usepackage{multirow}
  \usepackage{wrapfig}
  \usepackage{float}
  \usepackage{colortbl}
  \usepackage{pdflscape}
  \usepackage{tabu}
  \usepackage{threeparttable}
  \usepackage{threeparttablex}
  \usepackage[normalem]{ulem}
  \usepackage{rotating}
  \usepackage{makecell}
  \usepackage{xcolor}
  \usepackage{tikz} % required for image opacity change
  \usepackage[absolute,overlay]{textpos} % for text formatting
  \usepackage[utf8]{inputenc}
  \usetikzlibrary{mindmap,arrows,shapes,positioning,shadows,trees}

  % custom commands for flowchart
  \newcommand*{\h}{\hspace{5pt}}% for indentation
  \newcommand*{\hh}{\h\h}% double indentation

  \usepackage[skip=2pt]{caption}

  % this font option is amenable for beamer
  \setbeamerfont{caption}{size=\tiny}


%% IQSS overrides
\iqsssectiontitle{Outline}

\AtBeginSection[]{
  \title{\insertsectionhead}
  {
    \definecolor{white}{rgb}{0.776,0.357,0.157}
    \definecolor{iqss@orange}{rgb}{1,1,1}
    \ifnum \insertmainframenumber > \insertframenumber
    \frame{
      \frametitle{\iqsssectiontitleheader}
      \tableofcontents[currentsection]
    }
    \else
    \frame{
      \frametitle{Backup Slides}
      \tableofcontents[sectionstyle=shaded/shaded,subsectionstyle=shaded/shaded/shaded]
    }
    \fi
  }
}

\AtBeginSubsection[]{}

%%


  \title[]{Concept of cooperatives}



  \author[
        Deependra Dhakal
    ]{Deependra Dhakal}

  \institute[
    ]{
    GAASC, Baitadi \and Tribhuwan University
    }

\date[
      \today
  ]{
      \today
        }

\begin{document}

% Hide progress bar and footline on titlepage
  \begin{frame}[plain]
  \titlepage
  \end{frame}



\hypertarget{introduction}{%
\section{Introduction}\label{introduction}}

\begin{frame}{Definition}
\protect\hypertarget{definition}{}
\begin{itemize}
\tightlist
\item
  A cooperative is an autonomous association of people who voluntarily
  cooperate for their mutual social, economic, and cultural benefit.
\item
  Includes non-profit community organizations and businesses that are
  owned and managed by the people who use its services (a consumer
  cooperative) or by the people who work there (a worker cooperative) or
  by the people who live there (a housing cooperative).
\item
  In short: ``a jointly owned enterprise engaging in the production or
  distribution of goods or the supplying of services, operated by its
  members for their mutual benefit, typically organized by consumers or
  farmers.''
\item
  Co-operatives frequently have social goals which they aim to
  accomplish by investing a proportion of trading profits back into
  their communities.
\item
  The Rochdale Society of Equitable Pioneers, founded in 1844, is
  usually considered the first successful cooperative enterprise. They
  set up the society to open their own store selling food items they
  could not otherwise afford.
\end{itemize}
\end{frame}

\begin{frame}{Cooperative values}
\protect\hypertarget{cooperative-values}{}
\begin{itemize}
\tightlist
\item
  Self help
\item
  Self accountability
\item
  Democracy
\item
  Equality
\item
  Equity
\item
  Solidarity
\item
  Honesty
\item
  Openness
\item
  Social responsibility
\item
  Caring for others
\end{itemize}
\end{frame}

\begin{frame}{Principles\footnote<.->{Adopted by the international
  cooperative alliance in 1995}}
\protect\hypertarget{principles}{}
\footnotesize

\begin{enumerate}
\tightlist
\item
  Voluntary and open membership
\end{enumerate}

\begin{itemize}
\tightlist
\item
  Cooperatives are open to all people willing to volunteer to use its
  services and willing to accept the responsibilities of membership,
  without gender, social, racial, political or religious discrimination.
\end{itemize}

\begin{enumerate}
\setcounter{enumi}{1}
\tightlist
\item
  Democratic member control
\end{enumerate}

\begin{itemize}
\tightlist
\item
  Those who buy the goods or use the services of the cooperatives also
  actively participate in setting policies and making decisions.
\end{itemize}

\begin{enumerate}
\setcounter{enumi}{2}
\tightlist
\item
  Members' economic participation
\end{enumerate}

\begin{itemize}
\tightlist
\item
  Members contribute equally to and democratically control the capital
  of the cooperative.
\item
  Benefits are distributed proportionally to each member's level of
  participation in the cooperative, for instance, by a dividend on sales
  or purchases, rather than according to capital invested.
\end{itemize}
\end{frame}

\begin{frame}{}
\protect\hypertarget{section}{}
\footnotesize

\begin{enumerate}
\setcounter{enumi}{3}
\tightlist
\item
  Autonomy and independence
\end{enumerate}

\begin{itemize}
\tightlist
\item
  Cooperatives are autonomous, self-help organizations controlled by
  their members.
\item
  If they enter into agreements with other organizations, including
  governments, or raise capital from external sources, they maintain
  their cooperative autonomy.
\end{itemize}

\begin{enumerate}
\setcounter{enumi}{4}
\tightlist
\item
  Education, training and information
\end{enumerate}

\begin{itemize}
\tightlist
\item
  Co-operatives provide education and training for their members,
  elected representatives, managers, and employees so they can
  contribute effectively to the development of their co-operatives.
\end{itemize}
\end{frame}

\begin{frame}{}
\protect\hypertarget{section-1}{}
\begin{enumerate}
\setcounter{enumi}{5}
\tightlist
\item
  Cooperation among cooperatives
\end{enumerate}

\begin{itemize}
\tightlist
\item
  Co-operatives serve their members most effectively and strengthen the
  co-operative movement by working together through local, national,
  regional and international structures.
\end{itemize}

\begin{enumerate}
\setcounter{enumi}{6}
\tightlist
\item
  Concern for community
\end{enumerate}

\begin{itemize}
\tightlist
\item
  Co-operative work for the sustainable development of their communities
  through policies approved by their members.
\end{itemize}
\end{frame}

\hypertarget{organizationstructures}{%
\section{Organization/structures}\label{organizationstructures}}

\begin{frame}{}
\protect\hypertarget{section-2}{}
\begin{tikzpicture}[auto,
    %decision/.style={diamond, draw=black, thick, fill=white,
    %text width=8em, text badly centered,
    %inner sep=1pt, font=\sffamily\small},
    block_center/.style ={rectangle, draw=black, thick, fill=white,
      text width=8em, text centered,
      minimum height=4em},
    block_left/.style ={rectangle, draw=black, thick, fill=white,
      text width=16em, text ragged, minimum height=4em, inner sep=6pt},
    block_noborder/.style ={rectangle, draw=none, thick, fill=none,
      text width=18em, text centered, minimum height=1em},
    block_assign/.style ={rectangle, draw=black, thick, fill=white,
      text width=18em, text ragged, minimum height=3em, inner sep=6pt},
    block_lost/.style ={rectangle, draw=black, thick, fill=white,
      text width=16em, text ragged, minimum height=3em, inner sep=6pt},
      line/.style ={draw, thick, -latex', shorten >=0pt}]
    % outlining the flowchart using the PGF/TikZ matrix funtion
    \matrix [column sep=5mm,row sep=3mm] {
      % enrollment - row 1
      \node [block_center] (referred) {Referred (n=173)};
      & \node [block_left] (excluded1) {Excluded (n=17): \\
        a) Did not wish to participate (n=9) \\
        b) Did not show for interview (n=5) \\
        c) Other reasons (n=3)}; \\
      % enrollment - row 2
      \node [block_center] (assessment) {Assessed for eligibility (n=156)}; 
      & \node [block_left] (excluded2) {Excluded (n=54): \\
        a) Inclusion criteria not met (n=22) \\
        b) Exclusion criteria(s) met (n=13) \\
        c) Not suited for group (n=7) \\
        d) Not suited for CBT (n=2) \\
        e) Sought other treatment (n=3) \\
        f) Other reasons (n=7)}; \\
      % enrollment - row 3
      \node [block_center] (random) {Randomised (n=102)}; 
      & \\
      % follow-up - row 4
      \node [block_noborder] (i) {Intervention group}; 
      & \node [block_noborder] (wlc) {Wait-list control group}; \\
      % follow-up - row 5
      \node [block_assign] (i_T0) {Allocated to intervention (n=51): \\
      \h Received intervention (n=49) \\
      \h Did not receive intervention (n=2, \\
      \hh 1 with primary anxiety disorder, \\
      \hh 1 could not find time to participate)}; 
      & \node [block_assign] (wlc_T0) {Allocated to wait-list (n=51): \\
      \h Stayed on wait-list (n=48) \\
      \h Did not stay on wait-list (n=3, \\
      \hh 2 changed jobs and lost motivation, \\
      \hh 1 was offered treatment elsewhere)}; \\
      % follow-up - row 6
      \node [block_lost] (i_T3) {Post-intervention measurement: \\
      \h Lost to follow-up (n=5, \\
      \hh 2 dropped out of the intervention, \\
      \hh 3 did not complete measurement)}; 
      & \node [block_lost] (wlc_T3) {Post-wait-list measurement: \\
      \h Lost to follow-up (n=6, \\
      \hh 3 dropped out of the wait-list, \\
      \hh 3 did not complete measurement)}; \\
      % follow-up - row 7
      % empty first column for intervention group 
      & \node [block_assign] (wlc_T36) {Allocated to intervention (n=48): \\
      \h Received intervention (n=46) \\
      \h Did not receive intervention (n=2, \\
      \hh 1 reported low motivation, \\
      \hh 1 could not find time to participate)}; \\
      % follow-up - row 8
      \node [block_lost] (i_T6) {3-months follow-up measurement: \\
      \h Lost to follow-up (n=9, \\
      \hh did not complete measurement)}; 
      & \node [block_lost] (wlc_T6) {Post-intervention measurement: \\
      \h Lost to follow-up (n=5, \\
      \hh 2 dropped out of the intervention, \\
      \hh 3 did not complete measurement)}; \\
      % follow-up - row 9
      % empty first column for intervention group 
      & \node [block_lost] (wlc_T9) {3-months follow-up measurement \\
      \h Lost to follow-up (n=2, \\
      \hh did not complete measurement)}; \\
      % analysis - row 10
      \node [block_assign] (i_ana) {Analysed (n=51)}; 
      & \node [block_assign] (wlc_ana) {Analysed (n=51)}; \\
    };% end matrix
    % connecting nodes with paths
    \begin{scope}[every path/.style=line]
      % paths for enrollemnt rows
      \path (referred)   -- (excluded1);
      \path (referred)   -- (assessment);
      \path (assessment) -- (excluded2);
      \path (assessment) -- (random);
      \path (random)     -- (i);
      \path (random)     -| (wlc);
      % paths for i-group follow-up rows
      \path (i)          -- (i_T0);
      \path (i_T0)       -- (i_T3);
      \path (i_T3)       -- (i_T6);
      \path (i_T6)       -- (i_ana);
      % paths for wlc-group follow-up rows
      \path (wlc)        -- (wlc_T0);
      \path (wlc_T0)     -- (wlc_T3);
      \path (wlc_T3)     -- (wlc_T36);
      \path (wlc_T36)    -- (wlc_T6);
      \path (wlc_T6)     -- (wlc_T9);
      \path (wlc_T9)     -- (wlc_ana);
    \end{scope}
  \end{tikzpicture}

\includegraphics[width=0.8\linewidth]{21-concept_of_cooperatives_files/figure-beamer/organogram-cooperatives-1}
\end{frame}

\begin{frame}{Types of farmer cooperative}
\protect\hypertarget{types-of-farmer-cooperative}{}
\footnotesize

\begin{enumerate}
\tightlist
\item
  Production cooperatives
\end{enumerate}

\begin{itemize}
\tightlist
\item
  Farmer cooperatives for agricultural production such as milk, fruits
  and vegetables, poultry, etc. fall into this category.
\end{itemize}

\begin{enumerate}
\setcounter{enumi}{1}
\tightlist
\item
  Supply cooperatives
\end{enumerate}

\begin{itemize}
\tightlist
\item
  Farm supply cooperatives are vital for the dependable supply of farm
  inputs such as farm machinery, equipment, fertilizers, housing
  materials, livestock feed, seed and petroleum products.
\item
  They may also handle items such as lawn equipment, food items, or
  necessary items for gardening.
\item
  Cooperative endeavors such as feed mills, farm machinery, and
  fertilizer plants can be established at the regional or national
  level, whereas the farm supply depot can be established at the local
  level.
\end{itemize}
\end{frame}

\begin{frame}{}
\protect\hypertarget{section-3}{}
\begin{enumerate}
\setcounter{enumi}{2}
\tightlist
\item
  Service cooperatives
\end{enumerate}

\begin{itemize}
\tightlist
\item
  Set up for special services such as credit services,
  telephone/electric service, insurance services, irrigation services,
  grain banks, trucking, artificial insemination, cotton ginning, ginger
  drying, rice drying, etc.
\item
  Service cooperatives may also provide items such as chemicals, diesel,
  oil, gas, feed, seedlings, and seeds to its members.
\item
  Soil testing, crop scouting, and land leveling are other services a
  service cooperative may provide to its members.
\end{itemize}

\begin{enumerate}
\setcounter{enumi}{3}
\tightlist
\item
  Credit
\end{enumerate}

\begin{itemize}
\tightlist
\item
  Members should be able to obtain the required credit from their
  cooperative, and the cooperative should be able to deliver the credit
  and accept deposits from its members in order to build some capital
  for itself.
\end{itemize}
\end{frame}

\begin{frame}{}
\protect\hypertarget{section-4}{}
\begin{enumerate}
\setcounter{enumi}{3}
\tightlist
\item
  Processing
\end{enumerate}

\begin{itemize}
\tightlist
\item
  Farmer's cooperatives for processing agricultural commodities such as
  fruits, vegetables, spices and beverages crops. They are important to
  minimize post harvest losses and increase the income of the farmers.
\end{itemize}

\begin{enumerate}
\setcounter{enumi}{4}
\tightlist
\item
  Guidance (which includes education, training and extension services):
\end{enumerate}

\begin{itemize}
\tightlist
\item
  The members should be able to receive spontaneous guidance and advice
  from their cooperative, and the cooperative should be able to deliver
  expert advice to the members
\end{itemize}

\begin{enumerate}
\setcounter{enumi}{5}
\tightlist
\item
  Marketing cooperatives
\end{enumerate}

\begin{itemize}
\tightlist
\item
  Includes commodities or commodity groups such as cotton, dairy, fruit
  and vegetables, poultry, and livestock marketing cooperatives.
\item
  Primary objectives is marketing the farm produce of its members.
\item
  As marketing is difficult task for individual farmers, establishing a
  marketing cooperative is very important for commercialized
  agricultural development and to maintain the farm profit.
\end{itemize}
\end{frame}

\hypertarget{roles-of-cooperative-in-commercial-farming}{%
\section{Roles of cooperative in commercial
farming}\label{roles-of-cooperative-in-commercial-farming}}

\begin{frame}{}
\protect\hypertarget{section-5}{}
\footnotesize

\begin{itemize}
\tightlist
\item
  Agricultural cooperatives play an important role in food production
  and distribution, and in supporting long-term food security.
\item
  Agricultural cooperatives play an important role in supporting small
  agricultural producers and marginalized groups such as young people
  and women.
\item
  Cooperatives offer small agricultural producers opportunities and a
  wide range of services, including improved access to,

  \begin{itemize}
  \tightlist
  \item
    markets,
  \item
    natural resources,
  \item
    information,
  \item
    communications,
  \item
    technologies,
  \item
    credit,
  \item
    training, and
  \item
    warehouses.
  \end{itemize}
\end{itemize}
\end{frame}

\begin{frame}{}
\protect\hypertarget{section-6}{}
\footnotesize

\begin{itemize}
\tightlist
\item
  Facilitate smallholder producers' participation in decision-making at
  all levels, support them in securing land-use rights, and negotiate
  better terms for engagement in contract farming and lower prices for
  agricultural inputs such as seeds, fertilizer and equipment.
\item
  Through this support, smallholder producers can secure their
  livelihoods and play a greater role in meeting the growing demand for
  food on local, national and international markets, thus contributing
  to poverty alleviation, food security and the eradication of hunger.
\item
  Agricultural cooperatives also promote the participation of women in
  economic production

  \begin{itemize}
  \tightlist
  \item
    women are able to unite in solidarity and provide a network of
    mutual support to overcome cultural restrictions to pursuing
    commercial economic activities.
  \item
    For example, women-only cooperatives in South Asia facilitate
    economic independence and improve the social standing of women
    through their active participation in businesses and management.
  \end{itemize}
\end{itemize}
\end{frame}

\hypertarget{cooperatives-laws-and-by-laws}{%
\section{Cooperatives laws and
by-laws}\label{cooperatives-laws-and-by-laws}}

\begin{frame}{Acts, policies and laws}
\protect\hypertarget{acts-policies-and-laws}{}
\begin{itemize}
\tightlist
\item
  Cooperative act, 2048 BS
\item
  Cooperative regulation, 2049 BS
\item
  Cooperative registration operation, auditing, monitoring and execution
  standards, 2068 BS
\item
  National cooperative policy, 2069 BS
\item
  Cooperative organization unification related guideline, 2070 BS
\item
  Cooperative act, 2074 BS\footnote<.->{Refer to lecture slides in
    Nepali for details}
\end{itemize}
\end{frame}

\begin{frame}{National cooperative federation of Nepal (NCFN)}
\protect\hypertarget{national-cooperative-federation-of-nepal-ncfn}{}
\begin{itemize}
\tightlist
\item
  Established in June 20, 1993 under the Co-operative Act, 1992
\item
  The apex body of the cooperative movement of all types and levels of
  cooperatives organized on the basis of universally accepted
  cooperative values and principles.
\item
  Represents in government, national and international forum on behalf
  of all cooperatives.
\item
  There are more than 24 thousand Primary Cooperatives, 223 District
  Level Cooperative Unions, 14 Central Cooperative Unions and 1 National
  Cooperative Bank under the umbrella of NCFN.
\item
  There are more than 3.2 million cooperative members including around
  42 percent women membership. (Source: Department of Cooperatives, July
  15, 2011)
\item
  NCFN is a member of International Cooperative Alliance (ICA), Geneva.
  It is also affiliated with the Network for the Development of
  Agricultural Cooperatives (NEDAC), Thailand.
\end{itemize}
\end{frame}




\end{document}
