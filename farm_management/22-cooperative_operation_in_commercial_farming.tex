\PassOptionsToPackage{unicode=true}{hyperref} % options for packages loaded elsewhere
\PassOptionsToPackage{hyphens}{url}
\documentclass[12pt,ignorenonframetext,aspectratio=169]{beamer}
\IfFileExists{pgfpages.sty}{\usepackage{pgfpages}}{}
\setbeamertemplate{caption}[numbered]
\setbeamertemplate{caption label separator}{: }
\setbeamercolor{caption name}{fg=normal text.fg}
\beamertemplatenavigationsymbolsempty
\usepackage{lmodern}
\usepackage{amssymb}
\usepackage{amsmath}
\usepackage{ifxetex,ifluatex}
\usepackage{fixltx2e} % provides \textsubscript
\ifnum 0\ifxetex 1\fi\ifluatex 1\fi=0 % if pdftex
  \usepackage[T1]{fontenc}
  \usepackage[utf8]{inputenc}
\else % if luatex or xelatex
  \ifxetex
    \usepackage{mathspec}
  \else
    \usepackage{fontspec}
\fi
\defaultfontfeatures{Ligatures=TeX,Scale=MatchLowercase}






%
\fi

  \usetheme[]{iqss}






% use upquote if available, for straight quotes in verbatim environments
\IfFileExists{upquote.sty}{\usepackage{upquote}}{}
% use microtype if available
\IfFileExists{microtype.sty}{%
  \usepackage{microtype}
  \UseMicrotypeSet[protrusion]{basicmath} % disable protrusion for tt fonts
}{}


\newif\ifbibliography


\hypersetup{
      pdftitle={Cooperative operation in commercial farming},
        pdfauthor={Deependra Dhakal},
          pdfborder={0 0 0},
    breaklinks=true}
%\urlstyle{same}  % Use monospace font for urls







% Prevent slide breaks in the middle of a paragraph:
\widowpenalties 1 10000
\raggedbottom

  \AtBeginPart{
    \let\insertpartnumber\relax
    \let\partname\relax
    \frame{\partpage}
  }
  \AtBeginSection{
    \ifbibliography
    \else
      \let\insertsectionnumber\relax
      \let\sectionname\relax
      \frame{\sectionpage}
    \fi
  }
  \AtBeginSubsection{
    \let\insertsubsectionnumber\relax
    \let\subsectionname\relax
    \frame{\subsectionpage}
  }



\setlength{\parindent}{0pt}
\setlength{\parskip}{6pt plus 2pt minus 1pt}
\setlength{\emergencystretch}{3em}  % prevent overfull lines
\providecommand{\tightlist}{%
  \setlength{\itemsep}{0pt}\setlength{\parskip}{0pt}}

  \setcounter{secnumdepth}{0}


  \usepackage{booktabs}
  \usepackage{longtable}
  \usepackage{emptypage}
  \usepackage{array}
  \usepackage{multirow}
  \usepackage{wrapfig}
  \usepackage{float}
  \usepackage{colortbl}
  \usepackage{pdflscape}
  \usepackage{tabu}
  \usepackage{threeparttable}
  \usepackage{threeparttablex}
  \usepackage[normalem]{ulem}
  \usepackage{rotating}
  \usepackage{makecell}
  \usepackage{xcolor}
  \usepackage{tikz} % required for image opacity change
  \usepackage[absolute,overlay]{textpos} % for text formatting
  \usepackage[utf8]{inputenc}
  \usetikzlibrary{mindmap,arrows,shapes,positioning,shadows,trees}
  \usepackage[skip=2pt]{caption}

  % this font option is amenable for beamer
  \setbeamerfont{caption}{size=\tiny}


%% IQSS overrides
\iqsssectiontitle{Outline}

\AtBeginSection[]{
  \title{\insertsectionhead}
  {
    \definecolor{white}{rgb}{0.776,0.357,0.157}
    \definecolor{iqss@orange}{rgb}{1,1,1}
    \ifnum \insertmainframenumber > \insertframenumber
    \frame{
      \frametitle{\iqsssectiontitleheader}
      \tableofcontents[currentsection]
    }
    \else
    \frame{
      \frametitle{Backup Slides}
      \tableofcontents[sectionstyle=shaded/shaded,subsectionstyle=shaded/shaded/shaded]
    }
    \fi
  }
}

\AtBeginSubsection[]{}

%%


  \title[]{Cooperative operation in commercial farming}



  \author[
        Deependra Dhakal
    ]{Deependra Dhakal}

  \institute[
    ]{
    GAASC, Baitadi \and Tribhuwan University
    }

\date[
      \today
  ]{
      \today
        }

\begin{document}

% Hide progress bar and footline on titlepage
  \begin{frame}[plain]
  \titlepage
  \end{frame}



\begin{frame}{Commercial farming}
\protect\hypertarget{commercial-farming}{}
\begin{itemize}
\tightlist
\item
  Cultivation of crops, rearing of livestock and producing farm
  commodities for purpose of generating profit.
\item
  Commercial farming makes use of cost reducing technologies in
  production and optimize production levels based on consumer demand and
  price of produce.
\item
  These types of agricultural systems are more specialized in choice of
  crop or enterprise, so that operational plan may learn from industry
  sector.
\item
  Term generally refers to high input use condition where inputs such as
  irrigation, fertilizer and land management are heavily controlled to
  aim for higher volume of output.
\item
  Cooperative farming ensures better production of agricultural goods
  through encouragement of use of bigger scale technology, besides also
  providing chance for capital formation among smallholder farmers.
\end{itemize}
\end{frame}

\begin{frame}{Cooperative: Lesson from India}
\protect\hypertarget{cooperative-lesson-from-india}{}
\footnotesize

\begin{itemize}
\tightlist
\item
  Dairy farming based on the Anand Pattern, with a single marketing
  cooperative, is India's largest self-sustaining industry and its
  largest rural employment provider.
\item
  Successful implementation of the Anand model has made India the
  world's largest milk producer.
\item
  Here small, marginal farmers with a couple or so heads of milch cattle
  queue up twice daily to pour milk from their small containers into the
  village union collection points.
\item
  The milk after processing at the district unions is then marketed by
  the state cooperative federation nationally under the Amul brand name,
  India's largest food brand.
\item
  With the Anand pattern three-fourth of the price paid by the mainly
  urban consumers goes into the hands of millions of small dairy
  farmers, who are the owners of the brand and the cooperative.
\item
  The cooperative hires professionals for their expertise and skills and
  uses hi-tech research labs and modern processing plants and transport
  cold-chains, to ensure quality of their produce and value-add to the
  milk.
\end{itemize}
\end{frame}

\begin{frame}{Agriculture cooperatives types}
\protect\hypertarget{agriculture-cooperatives-types}{}
\footnotesize

\begin{enumerate}
\tightlist
\item
  Marketing cooperatives
\end{enumerate}

\begin{itemize}
\tightlist
\item
  Includes commodities or commodity groups such as cotton, dairy, fruit
  and vegetables, poultry, and livestock marketing cooperatives.
\item
  Primary objectives is marketing the farm produce of its members.
\item
  As marketing is difficult task for individual farmers, establishing a
  marketing cooperative is very important for commercialized
  agricultural development and to maintain the farm profit.
\end{itemize}

\begin{enumerate}
\setcounter{enumi}{1}
\tightlist
\item
  Farm supply cooperatives
\end{enumerate}

\begin{itemize}
\tightlist
\item
  Farm supply cooperatives are vital for the dependable supply of farm
  inputs such as farm machinery, equipment, fertilizers, housing
  materials, livestock feed, seed and petroleum products.
\item
  They may also handle items such as lawn equipment, food items, or
  necessary items for gardening.
\end{itemize}
\end{frame}

\begin{frame}{}
\protect\hypertarget{section}{}
\footnotesize

\begin{itemize}
\tightlist
\item
  Cooperative endeavors such as feed mills, farm machinery, and
  fertilizer plants can be established at the regional or national
  level, whereas the farm supply depot can be established at the local
  level.
\end{itemize}

\begin{enumerate}
\setcounter{enumi}{2}
\tightlist
\item
  Service cooperatives
\end{enumerate}

\begin{itemize}
\tightlist
\item
  Set up for special services such as credit services,
  telephone/electric service, insurance services, irrigation services,
  grain banks, trucking, artificial insemination, cotton ginning, ginger
  drying, rice drying, etc.
\item
  Service Cooperatives may also provide items such as chemicals, diesel,
  oil, gas, feed, seedlings, and seeds to its members.
\item
  Soil testing, crop scouting, and land leveling are other services a
  service cooperative may provide to its members.
\end{itemize}
\end{frame}

\begin{frame}{}
\protect\hypertarget{section-1}{}
\footnotesize

\begin{enumerate}
\setcounter{enumi}{3}
\tightlist
\item
  Production Cooperatives
\end{enumerate}

\begin{itemize}
\tightlist
\item
  Farmer cooperatives for agricultural production such as milk, fruits
  and vegetables, poultry, etc. fall into this category.
\end{itemize}

\begin{enumerate}
\setcounter{enumi}{4}
\tightlist
\item
  Processing cooperatives
\end{enumerate}

\begin{itemize}
\tightlist
\item
  Farmer cooperatives for processing agricultural commodities such as
  fruits, vegetables, flowers, etc. are important to minimize losses
  from perishable commodities and increase income from these
  commodities.
\end{itemize}
\end{frame}

\begin{frame}{Advantages of cooperatives in commercial farming}
\protect\hypertarget{advantages-of-cooperatives-in-commercial-farming}{}
\footnotesize

\begin{enumerate}
\item
  \textbf{Increases bargaining strength of the farmers}: Many of the
  defects of the present agricultural marketing system arise because
  often one ignorant and illiterate farmer (as an individual) has to
  face well-organized mass of clever intermediaries. If the farmers join
  hands and for a co-operative, naturally they will be less prone to
  exploitation and malpractices. Instead of marketing their produce
  separately, they will market it together through one agency.
\item
  \textbf{Direct dealing with final buyers}: The co-operatives can
  altogether skip the intermediaries and enter into direct relations
  with the final buyers. This practice will eliminate exploiters and
  ensure fair prices to both the producers and the consumers.
\item
  \textbf{Provision of credit}: The marketing co-operative societies
  provide credit to the farmers to save them from the necessity of
  selling their produce immediately after harvesting. This ensures
  better returns to the farmers.
\end{enumerate}
\end{frame}

\begin{frame}{}
\protect\hypertarget{section-2}{}
\footnotesize

\begin{enumerate}
\setcounter{enumi}{3}
\tightlist
\item
  \textbf{Easier and cheaper transport}: Bulk transport of agricultural
  produce by the societies is often easier and cheaper. Sometimes the
  societies have their own means of transport.
\item
  \textbf{Storage facilities}: The co-operative marketing societies
  generally have storage facilities. Thus the farmers can wait for
  better prices.
\item
  \textbf{Grading and standardization}: This task can be done more
  easily for a co-operative agency than for an individual farmer. For
  this purpose, they can seek assistance from the government or can even
  evolve their own grading arrangements.
\item
  \textbf{Market intelligence}: The co-operatives can arrange to obtain
  data on market prices, demand and supply and other related information
  from the markets on a regular basis and can plan their activities
  accordingly.
\end{enumerate}
\end{frame}

\begin{frame}{}
\protect\hypertarget{section-3}{}
\begin{enumerate}
\setcounter{enumi}{7}
\tightlist
\item
  \textbf{Influencing marketing prices}: Wherever strong marketing
  co-operative are operative, they have bargained for and have achieved,
  better prices for their agricultural produce.
\item
  \textbf{Provision of inputs and consumer goods}: The co-operative
  marketing societies can easily arrange for bulk purchase of
  agricultural inputs, like seeds, manures fertilizers etc. and consumer
  goods at relatively lower price and can then distribute them to the
  members.
\item
  \textbf{Processing of agricultural produce}: The co-operative
  societies can undertake processing activities like crushing seeds,
  ginning and pressing of cotton, etc. In addition to all these
  advantages, the co-operative marketing system can arouse the spirit of
  self-confidence and collective action in the farmers without which the
  programs of agricultural development, howsoever well-conceived and
  implemented, holds no promise to success.
\end{enumerate}
\end{frame}




\end{document}
