\PassOptionsToPackage{unicode=true}{hyperref} % options for packages loaded elsewhere
\PassOptionsToPackage{hyphens}{url}
\documentclass[12pt,ignorenonframetext,aspectratio=169]{beamer}
\IfFileExists{pgfpages.sty}{\usepackage{pgfpages}}{}
\setbeamertemplate{caption}[numbered]
\setbeamertemplate{caption label separator}{: }
\setbeamercolor{caption name}{fg=normal text.fg}
\beamertemplatenavigationsymbolsempty
\usepackage{lmodern}
\usepackage{amssymb}
\usepackage{amsmath}
\usepackage{ifxetex,ifluatex}
\usepackage{fixltx2e} % provides \textsubscript
\ifnum 0\ifxetex 1\fi\ifluatex 1\fi=0 % if pdftex
  \usepackage[T1]{fontenc}
  \usepackage[utf8]{inputenc}
\else % if luatex or xelatex
  \ifxetex
    \usepackage{mathspec}
  \else
    \usepackage{fontspec}
\fi
\defaultfontfeatures{Ligatures=TeX,Scale=MatchLowercase}






%
\fi

  \usetheme[]{iqss}






% use upquote if available, for straight quotes in verbatim environments
\IfFileExists{upquote.sty}{\usepackage{upquote}}{}
% use microtype if available
\IfFileExists{microtype.sty}{%
  \usepackage{microtype}
  \UseMicrotypeSet[protrusion]{basicmath} % disable protrusion for tt fonts
}{}


\newif\ifbibliography


\hypersetup{
      pdftitle={Cooperative operation in commercial farming},
        pdfauthor={Deependra Dhakal},
          pdfborder={0 0 0},
    breaklinks=true}
%\urlstyle{same}  % Use monospace font for urls







% Prevent slide breaks in the middle of a paragraph:
\widowpenalties 1 10000
\raggedbottom

  \AtBeginPart{
    \let\insertpartnumber\relax
    \let\partname\relax
    \frame{\partpage}
  }
  \AtBeginSection{
    \ifbibliography
    \else
      \let\insertsectionnumber\relax
      \let\sectionname\relax
      \frame{\sectionpage}
    \fi
  }
  \AtBeginSubsection{
    \let\insertsubsectionnumber\relax
    \let\subsectionname\relax
    \frame{\subsectionpage}
  }



\setlength{\parindent}{0pt}
\setlength{\parskip}{6pt plus 2pt minus 1pt}
\setlength{\emergencystretch}{3em}  % prevent overfull lines
\providecommand{\tightlist}{%
  \setlength{\itemsep}{0pt}\setlength{\parskip}{0pt}}

  \setcounter{secnumdepth}{0}


  \usepackage{booktabs}
  \usepackage{longtable}
  \usepackage{emptypage}
  \usepackage{array}
  \usepackage{multirow}
  \usepackage{wrapfig}
  \usepackage{float}
  \usepackage{colortbl}
  \usepackage{pdflscape}
  \usepackage{tabu}
  \usepackage{threeparttable}
  \usepackage{threeparttablex}
  \usepackage[normalem]{ulem}
  \usepackage{rotating}
  \usepackage{makecell}
  \usepackage{xcolor}
  \usepackage{tikz} % required for image opacity change
  \usepackage[absolute,overlay]{textpos} % for text formatting
  \usepackage[utf8]{inputenc}
  \usetikzlibrary{mindmap,arrows,shapes,positioning,shadows,trees}
  \usepackage[skip=2pt]{caption}

  % this font option is amenable for beamer
  \setbeamerfont{caption}{size=\tiny}


%% IQSS overrides
\iqsssectiontitle{Outline}

\AtBeginSection[]{
  \title{\insertsectionhead}
  {
    \definecolor{white}{rgb}{0.776,0.357,0.157}
    \definecolor{iqss@orange}{rgb}{1,1,1}
    \ifnum \insertmainframenumber > \insertframenumber
    \frame{
      \frametitle{\iqsssectiontitleheader}
      \tableofcontents[currentsection]
    }
    \else
    \frame{
      \frametitle{Backup Slides}
      \tableofcontents[sectionstyle=shaded/shaded,subsectionstyle=shaded/shaded/shaded]
    }
    \fi
  }
}

\AtBeginSubsection[]{}

%%


  \title[]{Cooperative operation in commercial farming}



  \author[
        Deependra Dhakal
    ]{Deependra Dhakal}

  \institute[
    ]{
    GAASC, Baitadi \and Tribhuwan University
    }

\date[
      \today
  ]{
      \today
        }

\begin{document}

% Hide progress bar and footline on titlepage
  \begin{frame}[plain]
  \titlepage
  \end{frame}



\begin{frame}{Commercial farming}
\protect\hypertarget{commercial-farming}{}
\begin{itemize}
\tightlist
\item
  Cultivation of crops, rearing of livestock and producing farm
  commodities for purpose of generating profit.
\item
  Commercial farming makes use of cost reducing technologies in
  production and optimize production levels based on consumer demand and
  price of produce.
\item
  These types of agricultural systems are more specialized in choice of
  crop or enterprise, so that operational plan may learn from industry
  sector.
\item
  Term generally refers to high input use condition where inputs such as
  irrigation, fertilizer and land management are heavily controlled to
  aim for higher volume of output.
\item
  Cooperative farming ensures better production of agricultural goods
  through encouragement of use of bigger scale technology, besides also
  providing chance for capital formation among smallholder farmers.
\end{itemize}
\end{frame}

\begin{frame}{Cooperative: Lesson from India}
\protect\hypertarget{cooperative-lesson-from-india}{}
\footnotesize

\begin{itemize}
\tightlist
\item
  Dairy farming based on the Anand Pattern, with a single marketing
  cooperative, is India's largest self-sustaining industry and its
  largest rural employment provider.
\item
  Successful implementation of the Anand model has made India the
  world's largest milk producer.
\item
  Here small, marginal farmers with a couple or so heads of milch cattle
  queue up twice daily to pour milk from their small containers into the
  village union collection points.
\item
  The milk after processing at the district unions is then marketed by
  the state cooperative federation nationally under the Amul brand name,
  India's largest food brand.
\item
  With the Anand pattern three-fourth of the price paid by the mainly
  urban consumers goes into the hands of millions of small dairy
  farmers, who are the owners of the brand and the cooperative.
\item
  The cooperative hires professionals for their expertise and skills and
  uses hi-tech research labs and modern processing plants and transport
  cold-chains, to ensure quality of their produce and value-add to the
  milk.
\end{itemize}
\end{frame}

\begin{frame}{Cooperative farming}
\protect\hypertarget{cooperative-farming}{}
\footnotesize

\begin{center}\includegraphics[width=0.8\linewidth]{22-cooperative_operation_in_commercial_farming_files/figure-beamer/mermaid-cooperative-farming-1} \end{center}
\end{frame}

\begin{frame}{}
\protect\hypertarget{section}{}
\begin{enumerate}
\tightlist
\item
  Cooperative joint farming society
\end{enumerate}

\begin{itemize}
\tightlist
\item
  Most comprehensive type of cooperative farming society.
\item
  Small owners pool their land for the purpose of joint cultivation.
\item
  The ownership is individual but the operations are collective.
\item
  The management is democratic and is elected by the members of the
  society.
\item
  It purchases various inputs from the market and arranges for the
  marketing of the produce.
\item
  It also seeks financial assistance from outside agencies to carry on
  these activities.
\end{itemize}
\end{frame}

\begin{frame}{}
\protect\hypertarget{section-1}{}
\begin{enumerate}
\setcounter{enumi}{1}
\tightlist
\item
  Cooperative better farming society
\end{enumerate}

\begin{itemize}
\tightlist
\item
  The members do not cultivate their land jointly. Each member
  cultivates his own land.
\item
  However, they co-operate with each other for pre-sowing and post
  harvesting operation.
\item
  For instance, they purchase various agricultural inputs like seeds,
  fertilizers, insecticides, services of machinery etc. on cooperative
  basis.
\item
  They sell the crops jointly.
\item
  A cooperative better farming society may also arrange for financial
  assistance for carrying on these activities.
\item
  The members pay for the service rendered to them by the society.
\end{itemize}
\end{frame}

\begin{frame}{}
\protect\hypertarget{section-2}{}
\begin{enumerate}
\setcounter{enumi}{2}
\tightlist
\item
  Cooperative tenant farming society
\end{enumerate}

\begin{itemize}
\tightlist
\item
  This is a society which purchases or leases in land from the
  Government or private individuals and then in turn leases out the land
  to its members.
\item
  Such societies are usually organized by landless farmers.
\item
  The members cultivate the land and pay the rent falling to their
  share, to the society.
\item
  The society also renders various other services to its members and
  charges from, its members for the service so rendered.
\item
  The profits earned by the society are distributed among its members
  according to some agreed formula.
\end{itemize}
\end{frame}

\begin{frame}{}
\protect\hypertarget{section-3}{}
\begin{enumerate}
\setcounter{enumi}{3}
\tightlist
\item
  Cooperative collective farming society
\end{enumerate}

\begin{itemize}
\tightlist
\item
  This type of society involves pooling of their land by the members on
  a permanent basis.
\item
  A member who joins this society cannot ever withdraw his land from the
  society.
\item
  He can only transfer his land to some other person who will now become
  a substitute member of the society. -The functions of this society are
  similar to those performed by a cooperative joint farming society.
\item
  The member get their wages and profits according to the labor and land
  respectively contributed by them.
\item
  It is obvious that such a society is formed in contravention of the
  general principles of cooperative i.e.~voluntary membership with a
  right to withdraw from the society at any time.
\end{itemize}
\end{frame}

\begin{frame}{Advantages of cooperatives in commercial farming}
\protect\hypertarget{advantages-of-cooperatives-in-commercial-farming}{}
\footnotesize

\begin{enumerate}
\item
  \textbf{Increases bargaining strength of the farmers}: Many of the
  defects of the present agricultural marketing system arise because
  often one ignorant and illiterate farmer (as an individual) has to
  face well-organized mass of clever intermediaries. If the farmers join
  hands and for a co-operative, naturally they will be less prone to
  exploitation and malpractices. Instead of marketing their produce
  separately, they will market it together through one agency.
\item
  \textbf{Direct dealing with final buyers}: The co-operatives can
  altogether skip the intermediaries and enter into direct relations
  with the final buyers. This practice will eliminate exploiters and
  ensure fair prices to both the producers and the consumers.
\item
  \textbf{Provision of credit}: The marketing co-operative societies
  provide credit to the farmers to save them from the necessity of
  selling their produce immediately after harvesting. This ensures
  better returns to the farmers.
\end{enumerate}
\end{frame}

\begin{frame}{}
\protect\hypertarget{section-4}{}
\footnotesize

\begin{enumerate}
\setcounter{enumi}{3}
\tightlist
\item
  \textbf{Easier and cheaper transport}: Bulk transport of agricultural
  produce by the societies is often easier and cheaper. Sometimes the
  societies have their own means of transport.
\item
  \textbf{Storage facilities}: The co-operative marketing societies
  generally have storage facilities. Thus the farmers can wait for
  better prices.
\item
  \textbf{Grading and standardization}: This task can be done more
  easily for a co-operative agency than for an individual farmer. For
  this purpose, they can seek assistance from the government or can even
  evolve their own grading arrangements.
\item
  \textbf{Market intelligence}: The co-operatives can arrange to obtain
  data on market prices, demand and supply and other related information
  from the markets on a regular basis and can plan their activities
  accordingly.
\end{enumerate}
\end{frame}

\begin{frame}{}
\protect\hypertarget{section-5}{}
\begin{enumerate}
\setcounter{enumi}{7}
\tightlist
\item
  \textbf{Influencing marketing prices}: Wherever strong marketing
  co-operative are operative, they have bargained for and have achieved,
  better prices for their agricultural produce.
\item
  \textbf{Provision of inputs and consumer goods}: The co-operative
  marketing societies can easily arrange for bulk purchase of
  agricultural inputs, like seeds, manures fertilizers etc. and consumer
  goods at relatively lower price and can then distribute them to the
  members.
\item
  \textbf{Processing of agricultural produce}: The co-operative
  societies can undertake processing activities like crushing seeds,
  ginning and pressing of cotton, etc. In addition to all these
  advantages, the co-operative marketing system can arouse the spirit of
  self-confidence and collective action in the farmers without which the
  programs of agricultural development, howsoever well-conceived and
  implemented, holds no promise to success.
\end{enumerate}
\end{frame}

\begin{frame}{Registration of cooperatives in Nepal}
\protect\hypertarget{registration-of-cooperatives-in-nepal}{}
\footnotesize

A cooperative organization can carry out its functions only after its
registration under the Cooperative act, 2074. For registration of
cooperatives following procedures should be followed.

\begin{enumerate}
\tightlist
\item
  Preliminary meeting: According to the Cooperative act, 2074, there
  should be at least 25 members to form a cooperative society.
  Preliminary meeting must be held before applying for the registration.
  The meeting is held in the presence of 25 members under 1 chairman
  among them. The following things should be discussed in the meeting:
\end{enumerate}

\begin{itemize}
\tightlist
\item
  Commencement of the business
\item
  The name and address of the society
\item
  The objectives of the society
\item
  The value of each share
\item
  Membership fee
\end{itemize}
\end{frame}

\begin{frame}{}
\protect\hypertarget{section-6}{}
\footnotesize

\begin{enumerate}
\setcounter{enumi}{1}
\tightlist
\item
  Filing an application for registration: After preparing and passing
  proposed by laws and working schemes in the preliminary general
  meeting. In application should be submitted to the office of
  registrar, department of cooperatives, and government of Nepal
\end{enumerate}

Following things are mentioned in the application form:

\begin{itemize}
\tightlist
\item
  Proposed name of society
\item
  Address
\item
  Objectives
\item
  Working areas
\item
  Liabilities
\item
  Total share capital
\item
  Total number of shares to be paid
\item
  Two copies of law of proposed society
\item
  Original copy of working scheme
\item
  Copies of citizen certificate
\item
  Application must be signed by chairman
\end{itemize}
\end{frame}

\begin{frame}{}
\protect\hypertarget{section-7}{}
\begin{enumerate}
\setcounter{enumi}{2}
\tightlist
\item
  Receiving the certificate of registration
\end{enumerate}

After filing application for registration, certificate of registration
is to be received. After applying application along with document, they
are submitted at the registration office. Then the registrar checks all
the documents. If the documents are satisfactory then registrar will
issue certificate of registration. After receiving certificate of
registration, the society can operate.
\end{frame}

\begin{frame}{Legal provisions for dissolution of co-operative society}
\protect\hypertarget{legal-provisions-for-dissolution-of-co-operative-society}{}
Cooperative society can be dissolved under following circumstances:

\begin{itemize}
\tightlist
\item
  Two third majority of total number of society can take decisions of
  dissolution.
\item
  Registrar can dissolve it, if application with reasonable clause is
  received.
\item
  Registrar can dissolve it, if the society is found inactive and not
  operating for two years.
\item
  The registrar can dissolve it, if it is found operating against the
  law and objectives of it.
\end{itemize}
\end{frame}




\end{document}
